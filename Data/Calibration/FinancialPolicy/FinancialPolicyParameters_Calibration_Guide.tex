	\documentclass[article,11pt,letterpaper,fleqn]{article}
\usepackage{graphicx,color}
\usepackage{array}
\usepackage{threeparttable}
\usepackage[format=hang,font=normalsize,labelfont=bf]{caption}
\usepackage{colortbl}
\usepackage{multirow}
\usepackage{geometry}
\usepackage{subfigure}
\geometry{letterpaper,tmargin=1in,bmargin=1in,lmargin=1.25in,rmargin=1.25in}
\usepackage{hyperref}
\hypersetup{colorlinks,%
citecolor=red,%
filecolor=red,%
linkcolor=red,%
urlcolor=blue,%
pdftex}
\usepackage{amsmath}
\usepackage{amssymb}
\usepackage{amsthm}
\usepackage{harvard}
\usepackage{tikz}
\usepackage{setspace}
\usepackage{float,graphicx,color}
\usepackage{appendix}
\usepackage{longtable}
\newtheorem*{thm}{Theorem}
\theoremstyle{definition}
\usepackage{lscape}
\numberwithin{equation}{section}
\newcommand{\cn}{\citeasnoun} % shortens command to cite as noun
\newcommand\ve{\varepsilon}


\title{Guide to Calibration of Financial Policy Cost Function Parameters in the OLG Dynamic Scoring Model}
\date{\today}



% make tables with smaller sized font 
\makeatletter
\def\table{\@ifnextchar[{\table@i}{\table@i[\fps@table]}}
\def\table@i[#1]{\@float{table}[#1]\footnotesize}
\makeatother



%\setlength{\topmargin}{-0.4in}
%\setlength{\topskip}{0.3in}    % between header and text
%\setlength{\textheight}{9.0in} % height of main text
%\setlength{\textwidth}{6in}    % width of text
%\setlength{\oddsidemargin}{39pt} %even side margin
%\setlength{\evensidemargin}{39pt} %odd side margin

\begin{document}
\bibliographystyle{aer}
\maketitle



\begin{abstract}
This will be the section in the dynamic scoring model handbook on calibrating the parameters affecting financial policy.
\end{abstract}

\section{Calibrating the Financial Policy Parameters}

Firm financial policy is endogenous, and is determined as a result of decisions that incorporate the benefits of new investment with the costs of financing that investment.  This guide will describe how the parameters directly affecting the costs of external finance are calibrated.  We have two cost functions that relate to external financing costs.  The cost of debt is a function of debt to the firm's stock of capital:

\begin{equation}
\label{eqn:debt_cost}
c(B_{t+1},K_{t}) = \chi_{bk}\left(\frac{B_{t+1}}{K_{t}}\right)^{\ve_{bk}},
\end{equation}

\noindent\noindent The costs associated with raising new equity include are given by:

\begin{equation}
\label{eqn:equity_cost}
\Psi(VN_{t})= \psi_{1}VN_{t} + \psi_{2}VN_{t}^{2}
\end{equation}

\noindent\noindent These costs are reduced form representations of costs associated with asymmetric information in the debt and equity markets and other financial frictions.


 Note that our treatment of sector will correspond to the tax-treatment of the business entity.  Therefore, we consider subchapter S corporations as non-corporate since they do not remit an entity level tax.  See Table \ref{tab:org_form} for this breakdown.  Note that these definitions are in contrast to the methodology used by BEA, where both subchapter C and subchapter S corporations fall into the ``corporate" sector and partnership and proprietorships fall under the non-corporate grouping.  

% Table generated by Excel2LaTeX from sheet 'SectorDefinitions'
\begin{table}[htbp]
  \centering
  \caption{Legal Form of Organization vs. Tax Treatment}
    \begin{tabular}{lll}
    \hline
    \hline
    Entity & Legal Form of Organization & Tax Treatment \\
   \hline
    C Corporation & Corporate & Corporate \\
    S Corporation & Corporate & Non-corporate \\
    Partnership & Non-corporate & n.a. \\
    \ \ \ Share of partnership income & n.a   & Corporate \\
    \ \ \ attributable to corporate partners & &  \\
    \ \ \ Share of partnership income& n.a.  & Non-corporate \\
    \ \ \ attributable to individual partners &  &  \\
    Sole Proprietorship & Non-corporate & Non-corporate \\
    \hline
    \hline
    \end{tabular}%
  \label{tab:org_form}%
\end{table}%


\section{Measuring Debt by Industry}
\label{sec:step3}

We measure total debt from the \href{http://www.federalreserve.gov/apps/fof/FOFTables.aspx}{Financial Accounts of the United States}.  In particular, we use the following tables to capture debt, which we measure separately for corporate financial and nonfinancial businesss, noncorporate business, and household mortgage debt:

\begin{itemize}
\item B.100: Value of owner-occupied houses;
\item L.102: Liabilities of nonfarm nonfinancial corporations, by type of instrument;
\item L.103: Liabilities of nonfarm noncorporate business, by type of instrument;
\item L.104: Liabilities of farm businesses, by type of instrument;
\item L.208: Commercial paper outstanding (financial corporations);
\item L.210: Agency- and GSE- (government-sponsored enterprise) backed securities outstanding (financial corporations);
\item L.212: Corporate bonds outstanding (financial corporations);
\item L.213: Corporate equity outstanding, by sector (nonfinancial, financial);
\item L.215: Bank loans (n.e.c., not elsewhere classified) outstanding (financial corporations);
\item L.216: Other loans and advances outstanding (financial corporations);
\item L.217: Total mortgages (financial corporations);
\item L.218: Home mortgages (households); and
\item L.228: Proprietors� equity in noncorporate business, by sector (farm, nonfarm).
\end{itemize}

To allocate debt across tax treatment, we use SOI Tax Stats Data (see the \href{https://github.com/OpenSourcePolicyCenter/dynamic/blob/master/Data/Calibration/DepreciationParameters/Depreciation_Calibration_Guide.pdf}{Depreciation Calibration Guide} for details on the specific files to use).  The Financial Account Data combine both S corporations and C corporations in the ``corporation" definition.  We thus use SOI data to identify the portion of debt and equity attributable to S corporations. Debt is assigned in proportion to interest deductions. Equity is assigned in proportion to the sum of capital stock, additional paid-in capital, and retained earnings minus treasury stock. The resulting S corporation amounts were subtracted from corporate totals (leaving the amount for C corporations) and added to noncorporate businesses.  We do the same to allocate the noncorporate across sole proprietorships and partnerhships.  We further allocate the amount of debt and equity attributable to corporate partnerships using a similar method.

\textcolor{red}{We should confirm that we can't find equity by industry.}  The Census' \href{http://www.census.gov/econ/qfr/}{Quarterly Financial Reports}. The has industry breaktouts and include balance sheet data.  If totals don't match the Financial Accounts data, we can use an adjustment to hit.

We then allocate this equity across industry (if we don't get it from the QFR data) and tax treatment using the distribution of capital across industry and tax treatment.  The assumption here is that the equity/capital ratio is the same (within industry) across corporate and non-corporate.



\subsection{Owner-Occupied Housing} 

Financial Accounts Data:
B B.100: Value of owner-occupied houses;
B L.218: Home mortgages (households)

For purposes of an analysis of effective tax rates, home- owner debt consists entirely of home mortgages�\$5,909.5 billion, according to the Flow of Funds report. The total value of owner-occupied housing was \$13,701.4 billion, and the average share financed by debt was 43.1 percent. As with the corporate and noncorporate sectors, CBO assumed that the marginal investment in owner- occupied housing, fh, is funded with the same share of debt.

Of course, individual homeowners typically have their highest debt share at the time of purchase. The share of debt is gradually reduced as the mortgage is paid off and as the home appreciates in value. Refinancing interrupts the process, but the trend is still toward a lower share of debt over time. The methodology of effective tax rates, how- ever, is concerned with the debt share over the life of an investment, not with the vari- ation within those years. From that perspective, the average share of debt among all homeowners is a better guide to the debt share of a marginal investment than is the debt share at the time a home is purchased.

\section{Measuring New Equity Issues by Industry}
\label{sec:step3}

To measure equity outstanding, we use the Financial Accounts of the United States, Table L.213: Corporate equity outstanding, by sector (nonfinancial, financial) and Table L.228: Proprietors� equity in noncorporate business, by sector (farm, nonfarm).

\textcolor{red}{We should confirm that we can't find equity by industry.}  The Census' \href{http://www.census.gov/econ/qfr/}{Quarterly Financial Reports}. The has industry breaktouts and include balance sheet data.  If totals don't match the Financial Accounts data, we can use an adjustment to hit.

We then allocate this equity across industry (if we don't get it from the QFR data) and tax treatment using the distribution of capital across industry and tax treatment.  The assumption here is that the equity/capital ratio is the same (within industry) across corporate and non-corporate.


\section{Measuring Capital Stock by Industry}
\label{sec:step3}

Done with the calibration of the depreciation parameters.


\section{Calibrating Parameters}

It is not yet determined exactly how we will use these data to calibrate parameters.  Using a generalized method of moments (GMM) estimator on the first order conditions of the firms is probably the best way to do this.


\section{A Note on Industry Classifications}

For our computational model, we would like to model the industries outlined in Table \ref{tab:prod_ind}. These are mostly at the 2-digit NAICS classification level, with some exceptions for industries that may face special tax treatment.  The data sources do not all share the same level of industry detail.  For example, the BEA Detailed Fixed Asset Tables report fixed assets by asset type and by industry, where industry categories are generally at the 3-digit NAICS level.  IRS data is generally reported at the 2-digit NAICS level, with some items being available at finer levels of aggregation and others at more coarse levels.  BEA's Standard Fixed Asset Tables report fixed asset by industry, but only at a very coarse level.  

% Table generated by Excel2LaTeX from sheet 'ProductionIndustries'
\begin{table}[htbp]
  \centering
  \caption{Production Industries}
    \begin{tabular}{lll}
    \hline
    \hline
    \# & NAICS Code & Industry \\
    \hline
    1     & 11    & Agriculture, Forestry, Fishing and Hunting \\
    2     & 211   & Oil and Gas Extraction \\
    3     & 212 and 213 & Mining and Support Activities for Mining \\
    4     & 22    & Utilities \\
    5     & 23    & Construction \\
    6     & 32411 & Petroleum Refineries \\
    7     & 336   & Transportation Equipment Manufacturing \\
    8     & 3391  & Medical Equipment and Supplies Manufacturing \\
    9     & Other codes in 31-33 & Manufacturing \\
    10    & 42    & Wholesale Trade \\
    11    & 44-45 & Retail Trade \\
    12    & 48-49 & Transportation and Warehousing \\
    13    & 51    & Information \\
    14    & 52    & Finance and Insurance \\
    15    & 53    & Real Estate and Rental and Leasing \\
    16    & 54    & Professional, Scientific, and Technical Services \\
    17    & 55    & Management of Companies and Enterprises \\
    18    & 56    & Administrative and Support and Waste Management and Remediation Services \\
    19    & 61    & Educational Services \\
    20    & 62    & Health Care and Social Assistance \\
    21    & 71    & Arts, Entertainment, and Recreation \\
    22    & 72    & Accommodation and Food Services \\
    23    & 81    & Other Services (except Public Administration) \\
    24    & 92    & Public Administration \\
      \hline
    \hline
    \end{tabular}%
  \label{tab:prod_ind}%
\end{table}%

When moving across these data sources, we try to retain the finest level of detail with regard to industry classification.  In cases where we cannot, we apply the most detailed industry information we can across the sub-classifications.  However, to maintain notational consistency, we refer to the industry with the subscript $m$, even if the industry category level differs.


\bibliography{/Users/jasondebacker/repos/dynamic/Data/Calibration/TaxModelCalibrationReferences}

\end{document}
