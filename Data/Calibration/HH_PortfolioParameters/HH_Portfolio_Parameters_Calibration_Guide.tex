	\documentclass[article,11pt,letterpaper,fleqn]{article}
\usepackage{graphicx,color}
\usepackage{array}
\usepackage{threeparttable}
\usepackage[format=hang,font=normalsize,labelfont=bf]{caption}
\usepackage{colortbl}
\usepackage{multirow}
\usepackage{geometry}
\usepackage{subfigure}
\geometry{letterpaper,tmargin=1in,bmargin=1in,lmargin=1.25in,rmargin=1.25in}
\usepackage{hyperref}
\hypersetup{colorlinks,%
citecolor=red,%
filecolor=red,%
linkcolor=red,%
urlcolor=blue,%
pdftex}
\usepackage{amsmath}
\usepackage{amssymb}
\usepackage{amsthm}
\usepackage{harvard}
\usepackage{tikz}
\usepackage{setspace}
\usepackage{float,graphicx,color}
\usepackage{appendix}
\usepackage{longtable}
\newtheorem*{thm}{Theorem}
\theoremstyle{definition}
\usepackage{lscape}
\numberwithin{equation}{section}
\newcommand{\cn}{\citeasnoun} % shortens command to cite as noun
\newcommand\ve{\varepsilon}


\title{Guide to Calibration of Household Portfolio Preference Parameters in the OLG Dynamic Scoring Model}
\date{\today}



% make tables with smaller sized font 
\makeatletter
\def\table{\@ifnextchar[{\table@i}{\table@i[\fps@table]}}
\def\table@i[#1]{\@float{table}[#1]\footnotesize}
\makeatother



%\setlength{\topmargin}{-0.4in}
%\setlength{\topskip}{0.3in}    % between header and text
%\setlength{\textheight}{9.0in} % height of main text
%\setlength{\textwidth}{6in}    % width of text
%\setlength{\oddsidemargin}{39pt} %even side margin
%\setlength{\evensidemargin}{39pt} %odd side margin

\begin{document}
\bibliographystyle{aer}
\maketitle



\begin{abstract}
This will be the section in the dynamic scoring model handbook on calibrating the parameters affecting the household's asset portfolio.
\end{abstract}

\section{Calibrating the Parameters Affecting Portfolio Decisions}

Households hold total assets, $a$.  They allocate these assets between equity in firms and bonds (from firms of government).  We assume that the equity holdings of a household represent a diversified portfolio of corporate and non-corporate equities across all industries (Note that this will be assumed as a simplifying assumption even if after-tax returns across industries are not the same.).  With no uncertainty in the model (other than household mortality risk), all bonds have the same return, $r_{b}$.   Since firms earn economic rents due to fixed factors of production, the return on equities exceeds the return on bonds (i.e., $r_{e}>r_{b}$).  The equilibrium equity premium in the model is justified through the households' preferences over their portfolio of assets.  Household's are assumed to have constant elasticity of substitution (CES) preferences over debt and equity in their portfolio.  These preferences allows for investor portfolios that are a mix of debt and equity despsite the two assets having differential returns.  Thus, while our model does not include idiosyncratic or aggregate uncertainty, these CES preferences account for the premium paid to the more risky assets through the preference parameters. To match lifecycle portfolio changes, we may consider allowing to the CES preference parameters to vary by age.  Given the CES preferences, the household's total assets, $a_{j,s,t}$ are given by (\textcolor{red}{NOTE that if we include these notations we should change our notation for assets from $b$ to $a$. We need to also think about the notion for equity - I use $e$ below, but we are already using that for effective labor units.}): 

\begin{equation}
\label{eqn:ces_port}
a_{j,s,t}= \left[\gamma_{a,s}^{\frac{-1}{\ve_{a,s}}}b_{j,s,t}^{\frac{1+\ve_{a.s}}{\ve_{a,s}}}+(1-\gamma_{a,s})^{\frac{-1}{\ve_{a,s}}}e_{j,s,t}^{\frac{1+\ve_{a,s}}{\ve_{a,s}}}\right]^{\frac{\ve_{a,s}}{1+\ve_{a,s}}},
\end{equation}

\noindent\noindent $b$ and $e$ are the amount of bonds and stocks held by the household.  The parameter $\ve_{a,s}$ is the elasticity of substitution between bonds and stocks in the asset portfolio of an age $s$ household.  $\gamma_{a,s}$ is the taste parameter in these preferences (\textcolor{red}{I've written this in the most flexible way, where both parameters depend upon age, but perhaps we just want the taste parameter to vary by age the rate of substitution is constant.}).

Our calibration will find values for $\ve_{a,s}$ and  $\gamma_{a,s}$.

We can hopefully use the demands that result from these preferences to identify the model parameters.  In particular, we have:

\begin{equation}
\label{eqn:bond_demand}
b_{j,s,t} = \left(\frac{\rho_{b,j,s,t}}{\rho_{e,j,s,t}}\right)^{\ve_{a,s}}\gamma_{a,s}a_{j,s,t}
\end{equation}     

\begin{equation}
\label{eqn:equity_demand}
e_{j,s,t} = \left(\frac{\rho_{e,j,s,t}}{\rho_{b,j,s,t}}\right)^{\ve_{a,s}}(1-\gamma_{a,s})a_{j,s,t}
\end{equation}     

With data on the after-tax rates of return and asset holds (both debt and equity), we have two unknowns (for each age $s$).   Thus we can use these two equations to solve for those two unknowns.

\section{Data if we assume parameters don't vary by age}
\label{sec:step3}

From the CBO document...

\emph{Identifying Household and Nonprofit Assets as Debt or Equity.} Debt and equity instruments are not distributed the same way among the categories of accounts. Because different proposals affect debt and equity differently, it is important to be able to reflect those differences in the calculation of effective tax rates. Under current law, debt and equity are taxed differently when they fall into the �fully taxable� category.
The source of financing underlying household assets is straightforward in some cases but difficult to ascertain in others. The simplest to identify are the assets held directly by households, but assets also can be held indirectly in nontaxable, pass-through enti- ties, such as mutual funds. Sometimes, the assets must be traced to identify the under- lying source of financing. For example, some households hold part of their assets in private defined-contribution plans. Those plans, in turn, hold some assets in mutual funds. Finally, the mutual funds hold corporate equities and various credit market instruments. The debt�equity split for the mutual fund must be assigned back to the household.

\emph{Equity.} Corporate equities held directly by households and nonprofits in 2002 were reported as \$5,047.8 billion on line 16 of Flow of Funds Table L.100.a (see Table 6). Corporate equities held indirectly by households and nonprofits in 2002 were reported as \$5,003.5 billion on line 8 of Table B.100.e. The vehicles in which equities are held indirectly include bank personal trusts and estates (Table B.100.e, line 9), life insurance companies (line 10), private defined-benefit pension funds (line 12), private defined-contribution pension funds (line 13), state and local government retirement funds (line 14), federal government retirement funds (line 15), and mutual funds (line 16).

Noncorporate equities in 2002 were reported as \$5,139.7 billion on line 22 of Flow of Funds Table L.100. All were held directly. Equity in owner-occupied housing in 2002 was reported at \$7,791.9 billion on line 50 of Table B.100. Like noncorporate business equity, all was held directly by households.


Table 6.
Corporate Equity Held by Households and Nonprofits, 2002
????Billions of Dollars of Equity
??Directly Held
Indirectly Held
Bank personal trusts
Life insurance companies
Private defined-benefit plans Private defined-contribution plans State and local retirement plans Federal retirement plans
Mutual funds
Total
5,047.8
385.0 692.5 535.3
1,076.0
869.8
45.9
1,399.0
 10,051.3
??Source: BoardofGovernorsoftheFederalReserveSystem,FlowofFundsAccountsoftheUnited States (March 10, 2005), Tables B.100.e. and L.100.a, available at www.federalreserve.gov/ releases/z1/20050310.

\emph{Debt.} The Flow of Funds report does not include a table that summarizes debt held both directly and indirectly by households. Instead, CBO compiled that information from various Flow of Funds tables (see Table 7).

Debt held directly by households is reported on line 7 of Table L.100.a as ``credit mar- ket instruments." It is divided into open market paper (line 8), treasury securities (line 9), agency- and GSE-backed securities (line 12), municipal securities (line 13), corporate and foreign bonds (line 14), and mortgages (line 15). Treasury securities are excluded from the analysis because they are never issued by a taxable entity.

Debt held indirectly by households is more difficult to identify. Some is held directly by sectors whose liabilities are all assets of households. Those assets can simply be passed through to households. The sectors for which this is permissible, and the Flow of Funds table on which their credit market instrument holdings are reported, are bank personal trusts (Table L.116), private defined-benefit plans (Table L.119.b), private defined-contribution plans (Table L.119.c), state and local government retirement funds (Table L.120), and federal government retirement funds (Table L.121).

Other debt held indirectly by households is held directly by sectors whose liabilities are only partly assets of households. Those sectors, and the Flow of Funds tables on which their credit market instrument holdings are reported, are life insurance companies (Table L.117), money market funds (Table L.122), mutual funds (Table L.123), agency- and GSE-backed mortgage pools (Table L.126), commercial banks
(Table L.109), savings institutions (Table L.113), and credit unions (Table L.115).
??

Before attributing the credit market instruments held by those sectors to households, the appropriate share was identified. That was a straightforward exercise for several sectors:

\begin{itemize}
\item Table L.117 in the Flow of Funds report indicated that 87.7 percent of life insurance liabilities were either pension fund reserves or life insurance reserves, both of which are considered assets solely of households and nonprofits.
\item All money market fund liabilities are money market fund shares. According to Table L.206, 48.8 percent of those shares were held by households or nonprofits. Therefore, 48.8 percent of each credit market instrument held by money market funds was attributed to the household and nonprofit sector.
\item All mutual fund liabilities are mutual fund shares. Because Table L.214 shows that 63.9 percent of those shares were held by households or nonprofits, 63.9 percent of each credit market instrument held by mutual funds was attributed to the household and nonprofit sector.
\item All agency- and GSE-backed mortgage pool liabilities are agency- and GSE-backed securities. According to Table L.210, 3.6 percent of those shares were held by households and nonprofits. Therefore, 3.6 percent of each credit market instru- ment held by agency- and GSE-backed mortgage pools was attributed to the household and nonprofit sector.
\end{itemize}

Deposits in commercial banks, savings institutions, and credit unions present another challenge in that they are not the only liability of depository institutions. Thus, only a portion of the debt held by depository institutions (67.1 percent) was passed through to other entities; the remainder was assumed to affect the value of corporate equities. Furthermore, according to Table L.205, sectors other than households and nonprofits held 37.0 percent of the passed-through deposits. Therefore, the share of each debt instrument held by depository institutions passed through to households and non- profits was approximately 42.3 percent.

Finally, it was necessary to identify credit market instruments held indirectly by households and nonprofits through more than one level of pass-through entity. In addition to holding credit market instruments, several sectors also hold money market funds, mutual funds, agency- and GSE-backed securities, and deposits in financial institutions: life insurance companies, private defined-benefit plans, private defined- contribution plans, state and local retirement plans, federal retirement plans, and bank and personal trusts.

\section{Data if parameters vary by age}
\label{sec:step3}

If parameters vary by age, we need micro data.  We can turn to the Survey of Consumer finances to get portfolio holdings of debt and equity by age.  We then use Equations \ref{eqn:bond_demand} and \ref{eqn:equity_demand} to estimate $\ve_{a,s}$ and  $\gamma_{a,s}$.

\bibliography{/Users/jasondebacker/repos/dynamic/Data/Calibration/TaxModelCalibrationReferences}

\end{document}
