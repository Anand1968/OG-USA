	\documentclass[article,11pt,letterpaper,fleqn]{article}
\usepackage{graphicx,color}
\usepackage{array}
\usepackage{threeparttable}
\usepackage[format=hang,font=normalsize,labelfont=bf]{caption}
\usepackage{colortbl}
\usepackage{multirow}
\usepackage{geometry}
\usepackage{subfigure}
\geometry{letterpaper,tmargin=1in,bmargin=1in,lmargin=1.25in,rmargin=1.25in}
\usepackage{hyperref}
\hypersetup{colorlinks,%
citecolor=red,%
filecolor=red,%
linkcolor=red,%
urlcolor=blue,%
pdftex}
\usepackage{amsmath}
\usepackage{amssymb}
\usepackage{amsthm}
\usepackage{harvard}
\usepackage{tikz}
\usepackage{setspace}
\usepackage{float,graphicx,color}
\usepackage{appendix}
\usepackage{longtable}
\newtheorem*{thm}{Theorem}
\theoremstyle{definition}
\usepackage{lscape}
\numberwithin{equation}{section}
\newcommand{\cn}{\citeasnoun} % shortens command to cite as noun
\newcommand\ve{\varepsilon}


\title{Guide to Calibration of Firm Depreciation Parameters in the OLG Dynamic Scoring Model}
\date{\today}



% make tables with smaller sized font 
\makeatletter
\def\table{\@ifnextchar[{\table@i}{\table@i[\fps@table]}}
\def\table@i[#1]{\@float{table}[#1]\footnotesize}
\makeatother



%\setlength{\topmargin}{-0.4in}
%\setlength{\topskip}{0.3in}    % between header and text
%\setlength{\textheight}{9.0in} % height of main text
%\setlength{\textwidth}{6in}    % width of text
%\setlength{\oddsidemargin}{39pt} %even side margin
%\setlength{\evensidemargin}{39pt} %odd side margin

\begin{document}
\bibliographystyle{aer}
\maketitle



\begin{abstract}
This will be the section in the dynamic scoring model handbook on calibrating depreciation parameters.
\end{abstract}

\section{Calibrating Deprecation Parameters}

Our model accounts for both physical (also called economic) depreciation rates and allowable deprecation rates as specified under tax law.  The differences between these two rates is an important element of tax policy.  For example., U.S. firms currently benefit from depreciation deductions for income tax purposes that are generally accelerated as compared to the actual economic depreciation.  Thus the net present value of the deprecation deductions for tax purposes exceeds the net present value of the costs the firms incur resulting from the economic deprecation of their assets.  

Note that depreciation rates (both economic and tax) differ by asset type.  For example, automobiles deprecate faster than do residential structures.  Due to computational concerns, our model will not capture the richness of multiple production industries and multiple types of capital.  Instead, we will consider multiple production industries, but only a single type of capital.\footnote{A limitation of this type of analysis is that we cannot model how changes in tax policy affect the types of capital firms own.}  However, our calibration of depreciation rates will account for the fact that different industries and sectors (corporate vs. non-corporate) hold different types of capital in different amounts.  Note that our treatment of sector will correspond to the tax-treatment of the business entity.  Therefore, we consider subchapter S corporations as non-corporate since they do not remit an entity level tax.  See Table \ref{tab:org_form} for this breakdown.  Note that these definitions are in contrast to the methodology used by BEA, where both subchapter C and subchapter S corporations fall into the ``corporate" sector and partnership and proprietorships fall under the non-corporate grouping.  

% Table generated by Excel2LaTeX from sheet 'SectorDefinitions'
\begin{table}[htbp]
  \centering
  \caption{Legal Form of Organization vs. Tax Treatment}
    \begin{tabular}{lll}
    \hline
    \hline
    Entity & Legal Form of Organization & Tax Treatment \\
   \hline
    C Corporation & Corporate & Corporate \\
    S Corporation & Corporate & Non-corporate \\
    Partnership & Non-corporate & n.a. \\
    \ \ \ Share of partnership income & n.a   & Corporate \\
    \ \ \ attributable to corporate partners & &  \\
    \ \ \ Share of partnership income& n.a.  & Non-corporate \\
    \ \ \ attributable to individual partners &  &  \\
    Sole Proprietorship & Non-corporate & Non-corporate \\
    \hline
    \hline
    \end{tabular}%
  \label{tab:org_form}%
\end{table}%


``Capital type" is not well defined since the Bureau of Economic Analysis (BEA) (who produces the National Accounts numbers and provides estimates of economic deprecation rates) uses different categories of capital than does the IRS.  Thus we create a crosswalk between the BEA classification of asset types (which is very detailed) and the various asset classifications used for tax policy.  This crosswalk is then used when calculating the tax deprecation rates by sector and industry.  We discuss the method for doing this below.

To compute the economic deprecation rate by industry and sector, we take a weighted average.  Assume there are $I$ types of capital.  We use the depreciation rate for each of those $I$ types to find the weighted average where the weights are determined by the amount of capital of each type.  Thus the economic depreciation rate for capital in sector $C$ in industry $m$ is given by:

\begin{equation}
\label{eqn:econ_deprec}
\delta_{m,C}=\sum_{i=1}^{I}\delta_{i}\frac{K_{i,m,C}}{K_{m,C}},
\end{equation}

\noindent\noindent where $K_{i,m,C}$ is the amount of capital of type $i$ in sector $C$ in industry $m$ and $K_{m,C}$ is the total amount of capital in sector $C$ in industry $m$.  Economic depreciation rates, $\delta_{i}$, correspond to the BEA's estimated depreciation rates by asset type.  Rates of depreciation under will be calculated analogously: 

\begin{equation}
\label{eqn:tax_deprec}
\delta^{\tau}_{m,C}=\sum_{i=1}^{I}\delta^{\tau}_{i,C}\frac{K_{i,m,C}}{K_{m,C}},
\end{equation}

\noindent\noindent where $\delta^{\tau}_{i,C}$ is the tax depreciation rate  of capital of type $i$ and is given by tax law.  Section \ref{sec:approx_tax_depr_rates} below describes how we approximate tax depreciation methods and recovery periods using a geometric depreciation function.

There are three pieces of data we need to gather to calibrate these depreciation parameters: 1) BEA estimates of capital stock by asset type, production industry, and sector (corporate vs. non-corporate), 2) BEA economic deprecation rates by BEA asset type,  and 3) Tax depreciation rates by IRS asset type.  The steps below guide on through the process of creating depreciation rates by industry and sector that are input into the computational  model.

We have found the \href{http://www.cbo.gov/sites/default/files/12-18-taxrates.pdf}{CBO's ``Computing Effective Tax Rates on Capital Income"} to be an extremely helpful guide for the process of calibrating depreciation parameters.  You will find that many of our choices for imputation methods follow those laid out by the CBO.

\section{Measuring the Capital Stock}
\label{sec:step3}

To calculate the rates of economic depreciation by production industry and sector, $\delta_{i,C}$, we use a weighted average of the rates of economic depreciation across capital types, by industry and sector.  Thus, to calibrate the depreciation parameters, we need data on the capital stock by asset type, industry, and sector, $K_{i,m,C}$.  The capital stock can be decomposed into three categories: fixed assets (equipment, structures, and intellectual property products), inventories, and land.  Thus we have,  $K_{i,m,C}={FA}_{i,m,C}+{INV}_{i,m,C}+{LAND}_{i,m,C}$.

%\begin{tikzpicture}
%    \tikzstyle{every node}=[font=\footnotesize]
%    \node {$K$} [sibling distance=4cm]
%        child { 
%             node {${FA}$}[sibling distance=0.5cm]	
%	   child { 
%	   	node {${FA}_{i,m},i=1,...,I,m=1,...,M$} [sibling distance=1.5cm]	
%		child{ node {${FA}_{i,m,C}$}}
%		child{node {${FA}_{i,m,NC}$}}
%		}
%		child{node{}}
%		child{node{}} 
%        }
%        child { 
%             node {${INV}$} [sibling distance=0.5cm]
%	   child { 
%	   	node {${INV}_{m,C}$ } [sibling distance=0.5cm]
%		} 
%            child { 
%	   	node {${INV}_{m,NC}$ } [sibling distance=0.5cm]
%		} 
%        }
%       child { 
%             node {${LAND}$}[sibling distance=1.5cm]
%	   child { 
%	   	node {${LAND}_{m,C}$ } [sibling distance=0.5cm]
%		child{node{}}		
%		child{node{}}
%		} 
%            child { 
%	   	node {${LAND}_{m,NC}$ } [sibling distance=0.5cm]
%		child{node{}}		
%		child{node{}}
%		} 
%        }
%    ;
%\end{tikzpicture}
%
%
%\begin{tikzpicture}[level/.style={sibling distance=60mm/#1}]
%\node [circle,draw] (z){$K$}
%  child {node [circle,draw] (a) {$\frac{n}{2}$}
%    child {node [circle,draw] (b) {$\frac{n}{2^2}$}
%      child {node {$\vdots$}
%        child {node [circle,draw] (d) {$\frac{n}{2^k}$}}
%        child {node [circle,draw] (e) {$\frac{n}{2^k}$}}
%      } 
%      child {node {$\vdots$}}
%    }
%    child {node [circle,draw] (g) {$\frac{n}{2^2}$}
%      child {node {$\vdots$}}
%      child {node {$\vdots$}}
%    }
%  }
%  child {node [circle,draw] (j) {$\frac{n}{2}$}
%    child {node [circle,draw] (k) {$\frac{n}{2^2}$}
%      child {node {$\vdots$}}
%      child {node {$\vdots$}}
%    }
%  child {node [circle,draw] (l) {$\frac{n}{2^2}$}
%    child {node {$\vdots$}}
%    child {node (c){$\vdots$}
%      child {node [circle,draw] (o) {$\frac{n}{2^k}$}}
%      child {node [circle,draw] (p) {$\frac{n}{2^k}$}
%        child [grow=right] {node (q) {$=$} edge from parent[draw=none]
%          child [grow=right] {node (q) {$O_{k = \lg n}(n)$} edge from parent[draw=none]
%            child [grow=up] {node (r) {$\vdots$} edge from parent[draw=none]
%              child [grow=up] {node (s) {$O_2(n)$} edge from parent[draw=none]
%                child [grow=up] {node (t) {$O_1(n)$} edge from parent[draw=none]
%                  child [grow=up] {node (u) {$O_0(n)$} edge from parent[draw=none]}
%                }
%              }
%            }
%            child [grow=down] {node (v) {$O(n \cdot \lg n)$}edge from parent[draw=none]}
%          }
%        }
%      }
%    }
%  }
%};
%\path (a) -- (j) node [midway] {+};
%\path (b) -- (g) node [midway] {+};
%\path (k) -- (l) node [midway] {+};
%\path (k) -- (g) node [midway] {+};
%\path (d) -- (e) node [midway] {+};
%\path (o) -- (p) node [midway] {+};
%\path (o) -- (e) node (x) [midway] {$\cdots$}
%  child [grow=down] {
%    node (y) {$O\left(\displaystyle\sum_{i = 0}^k 2^i \cdot \frac{n}{2^i}\right)$}
%    edge from parent[draw=none]
%  };
%\path (q) -- (r) node [midway] {+};
%\path (s) -- (r) node [midway] {+};
%\path (s) -- (t) node [midway] {+};
%\path (s) -- (l) node [midway] {=};
%\path (t) -- (u) node [midway] {+};
%\path (z) -- (u) node [midway] {=};
%\path (j) -- (t) node [midway] {=};
%\path (y) -- (x) node [midway] {$\Downarrow$};
%\path (v) -- (y)
%  node (w) [midway] {$O\left(\displaystyle\sum_{i = 0}^k n\right) = O(k \cdot n)$};
%\path (q) -- (v) node [midway] {=};
%\path (e) -- (x) node [midway] {+};
%\path (o) -- (x) node [midway] {+};
%\path (y) -- (w) node [midway] {$=$};
%\path (v) -- (w) node [midway] {$\Leftrightarrow$};
%\path (r) -- (c) node [midway] {$\cdots$};
%\end{tikzpicture}
%


Our main sources for these data are the BEA's \href{http://www.bea.gov/national/FA2004/Details/Index.html}{Detailed Data for Fixed Assets and Consumer Durable Goods} and the BEA's \href{(http://www.bea.gov/iTable/index_FA.cfm}{Standard Fixed Asset Tables}.  From the former, we pull the net stock, current cost for all fixed asset and industry categories for calendar year 2012.  From the later, we gather the stock of fixed assets by industry and sector by drawing upon the current-cost of the net capital stock by industry and form of legal organization, using Tables 4.1, 5.1, and 6.1.  We further draw upon data on the stock of inventories by industry from the \href{http://www.bea.gov/iTable/index_nipa.cfm}{BEA's NIPA Tables}, using Table 5.8.5A.  
\ \\
The BEA does not produce an estimate for the value of the stock of land.  For these data, we turn to the Federal Reserve's \href{http://www.federalreserve.gov/apps/fof/FOFTables.aspx}{Financial Accounts of the United States Data}.  Tables B.100, B.102, and B.103 give us the value for land for households and nonprofit organizations, non-financial corporate businesses, and non-financial non-corporate businesses, respectively.   To derive the value of land from these balance sheet data, we take the difference between the value of real estate and the value of residential structures.  For households, this is the difference between series FL155035015 (Line 4, ``Households; owner-occupied real estate including vacant land and mobile homes at market value") and series FL155012665 (Line 44, ``Households; residential structures, current cost basis"). For nonprofits, the is the difference between FL165035005 (Line 5, ``Nonprofit organizations; real estate at market value").  For non-financial corporate businesses, we find the value of land to be the difference between FL105035005 (Line 3, ``Nonfinancial corporate business; real estate at market value") and FL105012665 (Line 37 , ``Nonfinancial corporate business; residential structures, current cost basis") and FL105013665 (Line 38, ``Nonfinancial corporate business; nonresidential structures, current cost basis").  And for non-financial non-corporate businesses, we use Table B.103 to find the value of land as the difference between FL1150350055 (Line 3, ``Nonfinancial noncorporate business; real estate at market value") and FL115012665 (Line 35, ``Nonfinancial noncorporate business; residential structures, current cost basis") and FL115013665 (Line 36, ``Nonfinancial noncorporate business; nonresidential structures, current cost basis").
\ \\
Because the BEA's decomposition into corporate and non-corporate does not align with tax treatment, we must supplement the BEA data using Internal Revenue Service (IRS) data that allow us identify the share of corporate assets attributable to C and S corporations, as well as the fraction of partnership assets that face corporate tax treatment.  Thus we supplement the BEA data with IRS data.  The next section details how we use these IRS data to measure the capital stock held by different tax entity types.

\subsection{A Note on Industry Classifications}

For our computational model, we would like to model the industries outlined in Table \ref{tab:prod_ind}.\footnote{This excludes the multi-national sector, which we still need to think about.}  These are mostly at the 2-digit NAICS classification level, with some exceptions for industries that may face special tax treatment.  The data sources do not all share the same level of industry detail.  For example, the BEA Detailed Fixed Asset Tables report fixed assets by asset type and by industry, where industry categories are generally at the 3-digit NAICS level.  IRS data is generally reported at the 2-digit NAICS level, with some items being available at finer levels of aggregation and others at more coarse levels.  BEA's Standard Fixed Asset Tables report fixed asset by industry, but only at a very coarse level.  

% Table generated by Excel2LaTeX from sheet 'ProductionIndustries'
\begin{table}[htbp]
  \centering
  \caption{Production Industries}
    \begin{tabular}{lll}
    \hline
    \hline
    \# & NAICS Code & Industry \\
    \hline
    1     & 11    & Agriculture, Forestry, Fishing and Hunting \\
    2     & 211   & Oil and Gas Extraction \\
    3     & 212 and 213 & Mining and Support Activities for Mining \\
    4     & 22    & Utilities \\
    5     & 23    & Construction \\
    6     & 32411 & Petroleum Refineries \\
    7     & 336   & Transportation Equipment Manufacturing \\
    8     & 3391  & Medical Equipment and Supplies Manufacturing \\
    9     & Other codes in 31-33 & Manufacturing \\
    10    & 42    & Wholesale Trade \\
    11    & 44-45 & Retail Trade \\
    12    & 48-49 & Transportation and Warehousing \\
    13    & 51    & Information \\
    14    & 52    & Finance and Insurance \\
    15    & 53    & Real Estate and Rental and Leasing \\
    16    & 54    & Professional, Scientific, and Technical Services \\
    17    & 55    & Management of Companies and Enterprises \\
    18    & 56    & Administrative and Support and Waste Management and Remediation Services \\
    19    & 61    & Educational Services \\
    20    & 62    & Health Care and Social Assistance \\
    21    & 71    & Arts, Entertainment, and Recreation \\
    22    & 72    & Accommodation and Food Services \\
    23    & 81    & Other Services (except Public Administration) \\
    24    & 92    & Public Administration \\
      \hline
    \hline
    \end{tabular}%
  \label{tab:prod_ind}%
\end{table}%

When moving across these data sources, we try to retain the finest level of detail with regard to industry classification.  In cases where we cannot, we apply the most detailed industry information we can across the sub-classifications.  However, to maintain notational consistency, we refer to the industry with the subscript $m$, even if the industry category level differs.


\subsection{Calculating the Capital Stock by Tax Entity From IRS Data}

We use tax data to find values for the capital stock by taxable entity type.  We denote the values by $K^{\tau}_{m,j}={FA}^{\tau}_{m,j}+{INV}^{\tau}_{m,j}+{LAND}^{\tau}_{m,j}$, where the superscript $\tau$ denotes that these values are from tax records (as opposed to BEA calculations which include entities that do not file a income tax return) and the subscripts $m$ and $j$ refer to industry and tax entity type, respectively.  Tax entity types are subchapter C corporations, subchapter S corporations, partnerships, and sole proprietorships.  We discuss our methodology for finding the capital stock attributable to each below.
\ \\
\textcolor{red}{*** ONE THING I'M NOT REALLY SURE ABOUT IS HOW/IF WE SHOULD MEASURE ``INTELLECTUAL PROPERTY PRODUCTS" FROM TAX DATA?  DO WE USE INTANGIBLES?  THIS SEEMS IMPORTANT AFTER THE RECENT, LARGE REVISIONS TO HOW BEA ACCOUNTS FOR THE CAPITAL STOCK****}


\subsubsection{C Corporations}

Tax data on subchapter C corporations come from the data files for the \emph{SOI Tax Stats - Corporation Source Book} for 2011.  The link to those files is \href{http://www.irs.gov/uac/SOI-Tax-Stats-Corporation-Source-Book:-Data-File}{here}.  Specifically, we use the 2011sb1.csv and 2011sb3.csv files to find the aggregate amounts by industry for the following variables from Form 1120 and associated schedules: depreciable assets, accumulated depreciation, land, beginning-of-year inventories, interest paid, capital stock, additional paid-in capital, retained earnings (appropriated and unappropriated), and cost of treasury stock.  Note that the 2011sb1.csv file contains data from all Form 1120 returns (which includes both C and S corporations).  Thus, in calculating aggregates for subchapter C corporations only, we net out the totals by industry and line item for S corporations using the 2011sb3.csv data.

Note that the level of industry detail in 2011sb1.csv and 2011sb3.csv differ, with the former reporting variables as fine as the 6-digit NAICS level and the latter reporting variables at the 2-digit level.  In order to infer S corporation data at a finer level of industry detail, we make the assumption that the each variable is distributed across minor industries within a major industry in the same way for all corporations as it they are for S corporations.  Letting $x_{m1}$ be a variable of interest reported for all corporations in detailed industry $m1$ (e.g., detailed may be a 6-digit NAICS code) from 2011sb1.csv and $x_{m2}$ be the same variable reported for all corporations at the less detailed industry level (e.g., 2-digit NAICS).  We thus assume that the variable $x$ for S corporations can be allocated across detailed industry categories $m1$ as:

\begin{equation}
x_{m1,s}=\frac{x_{m1}}{x_{m2}}\times x_{m2,s},
\end{equation}

\noindent\noindent where $m1\in m2$.  Variables allocated in this way are then used when differencing out data from 2011sb1.csv and 2011sb3.csv to find the amounts for C corporations.

Using these data, we calculate the stock of fixed assets for C corporations in industry $m$ as reported on tax returns as: ${FA}^{\tau}_{m,c}$ as the difference between the aggregate amounts of depreciable assets and accumulated depreciation for that industry.  ${INV}^{\tau}_{m,c}$ and ${LAND}^{\tau}_{m,c}$ and taken from the aggregate values of beginning-of-year inventories and land, as reported on Schedule L of Form 1120.

\subsubsection{S Corporations}

Tax data on subchapter S corporations come from the data files for the \emph{SOI Tax Stats - Corporation Source Book} for 2011.  The link to those files is \href{http://www.irs.gov/uac/SOI-Tax-Stats-Corporation-Source-Book:-Data-File}{here}.  Specifically, we use the 2011sb3.csv file to find the aggregate amounts by industry for the following variables from Form 1120S and associated schedules: depreciable assets, accumulated depreciation, land, beginning-of-year inventories, interest paid, capital stock, additional paid-in capital, retained earnings (appropriated and unappropriated), and cost of treasury stock.  

Using these data, we calculate the stock of fixed assets for S corporations in industry $m$ as reported on tax returns as: ${FA}^{\tau}_{m,s}$ as the difference between the aggregate amounts of depreciable assets and accumulated depreciation for that industry.  ${INV}^{\tau}_{m,s}$ and ${LAND}^{\tau}_{m,s}$ and taken from the aggregate values of beginning-of-year inventories and land, as reported on Schedule L of Form 1120S.

\subsubsection{Partnerships}

For partnerships, we draw upon the \href{http://www.irs.gov/uac/SOI-Tax-Stats-Partnership-Statistics-by-Sector-or-Industry}{SOI Tax Stats - Partnership Statistics by Sector or Industry}.  There are three files we use to get measures of partnership capital in 2012.  From the 12pa01.xls file, we pull aggregate depreciation deductions and beginning of year inventories by industry.  From 12pa03.xls, we collect aggregate values for depreciable assets, accumulated depreciation, and land.  Finally, we use 12pa05.xls to help us allocate the total partnership capital stock between corporate, individual, and tax exempt partners (we discuss this further below).

Using these data, we calculate the stock of fixed assets for S corporations in industry $m$ as reported on tax returns as: ${FA}^{\tau}_{m,p}$ as the difference between the aggregate amounts of depreciable assets and accumulated depreciation for that industry.  ${INV}^{\tau}_{m,p}$ and ${LAND}^{\tau}_{m,p}$ and taken from the aggregate values of beginning-of-year inventories and land, as reported on Schedule L of Form 1065.

\ \\
NOTE THAT FOR THESE DATA, INDUSTRY CODES ARE NOT GIVEN, SO WE'LL NEED TO MAP THE NAMES TO NAICS CODES.
\ \\
 
\textbf{\emph{Allocating Partnership Capital Across Types of Partners}}: Partners in partnerships may be corporations, individuals, partnerships, tax-exempts, or other organizations.  Because these partners face different tax treatment, we need to allocate shares of partnership assets to each of these entity types.  We do this by using ratios of depreciable assets to net income/loss by industry.  We then use these ratios to distribute the share of total assets across partner type using the net income/loss going to partners of a given type in each industry.  The assumption here is that the ratio of assets to income/loss is the same across types of partners within a given industry.  This certainly misses some of the variation in the ownership structure of partnership assets and in the distribution of partnership income, but is a method that allows us to attribute partnership assets across partner types.

File 12pa03.xls provides data on depreciable assets by net income/loss by industry.  These data give totals for all partnerships and then totals separately for partnerships with net profits.  We determine the total from partnerships with net losses as the difference between the total for all partnerships and the total from partnerships with positive net income.  We use these data to find the amount of fixed assets by industry, $\text{FA}^{\tau}_{m,p}$, and calculate the ratios $\frac{\text{FA}^{\tau}_{m,p}}{\text{Net Income}_{m}}$ and $\frac{\text{FA}^{\tau}_{m,p}}{\text{Net Loss}_{m}}$, for each industry $m$.  Then, using 12pa05.xls, we gather the aggregate amounts of net income or losses distributed to partners by partner type and industry.  Note that the data from 12pa05.xls differ in the level of industry detail from the data in 12pa01.xls and 12pa03.xls.  For notational clarity, let $m1$ be the more detailed classifications and $m2$ be the less detailed classifications.  Using these two pieces of information together, we find the total amount of fixed assets by industry and partner type as:  

\begin{equation}
\text{FA}^{\tau}_{m2,p,t}=  \frac{\text{FA}^{\tau}_{m1,p}}{\text{Net Income(Loss)}_{m1}} \times \text{Net Income(Loss)}_{m2,t},
\end{equation}

\noindent\noindent where $m1\in m2$ and $t$ denotes partner type (individual, corporate, partnership, tax-exempt, other).  To identify the amount of fixed assets by more detailed industry classifications, we make the assumption that the distribution of assets by partner type is the same for all minor industries within a major industry:

\begin{equation}
\text{FA}^{\tau}_{m,p,t}=  \frac{\text{FA}^{\tau}_{m2,p,t}}{\text{FA}^{\tau}_{m2,p}} \times \text{FA}^{\tau}_{m1}
\end{equation}

We use the same methodology to allocation inventories and land to find ${INV}^{\tau}_{m,p,t}$ and ${LAND}^{\tau}_{m,p,t}$. 

When allocating capital across tax treatment, we will attribute the capital owned by corporate partners to the corporate sector and the remainder to the non-corporate sector.


\subsubsection{Sole Proprietorships}

We divide sole proprietorships into two groups: non-farm sole proprietors, who file a Schedule C of Form 1040, and farm sole proprietorships, who file Schedule F of Form 1040.  

\textbf{\emph{Non-farm Sole Proprietorships}}:  Our data for non-farm sole proprietorships come from the \href{http://www.irs.gov/uac/SOI-Tax-Stats-Nonfarm-Sole-Proprietorship-Statistics}{SOI Tax Stats - Non-farm Sole Proprietorship Statistics} for 2011.  Specifically, we use the file 11sp01br.xls.  These data do not record the value of depreciable assets or land for sole proprietorships, but they do contain depreciation deductions for sole proprietors.  Thus we impute the value of depreciable assets and land using the assumption that the ratio of depreciable assets to depreciation deductions and the ratio of land to depreciation deductions are the same within particular industry for sole proprietorships and partnerships.  Specifically, we find the stock of fixed assets for sole proprietors to be: 

\begin{equation}
{FA}^{\tau}_{sp}=\frac{\text{Depreciable Assets}_{m,p}}{\text{Depreciation Deductions}_{m,p}}\times \text{Depreciation Deductions}_{m,sp},
\end{equation}

\noindent\noindent where $m$ denotes industry and the subscripts $p$ and $sp$ represent partnership and sole proprietorship, respectively.  And the stock of land held by sole proprietorships to be: 

\begin{equation}
{LAND}^{\tau}_{sp}=\frac{\text{LAND}_{m,p}}{\text{Depreciation Deductions}_{m,p}}\times \text{Depreciation Deductions}_{m,sp}
\end{equation}

Inventories are reported on Schedule C, thus we take the value of beginning-of-year inventories directly from the 11sp01br.xls files as our value of ${INV}^{\tau}_{m,sp}$.

NOTE THAT FOR THESE DATA, INDUSTRY CODES ARE NOT GIVEN, SO WE'LL NEED TO MAP THE NAMES TO NAICS CODES.

\textbf{\emph{Farm Sole Proprietorships}}:  The SOI do not provide detailed data on farm sole proprietors.  Thus for these businesses, we use \href{http://www.agcensus.usda.gov/Publications/2012/Full_Report/Volume_1,_Chapter_1_US/st99_1_067_067.pdf}{Table 67 of the \emph{2012 Census of Agriculture}} (\emph{COA}).  The \emph{COA} reports the values of land and structures (together) and the value of machinery and equipment.  These values are reported separately by type of organization (e.g, sole proprietorship, partnership).  To find the value of depreciable assets that is comparable to those for non-farm sole proprietors as reported in tax data, we must adjust these data so that we have a separate accounting of land and structures.  We use tax data to help us to impute this decomposition.  

Let $R_{sp}$ be the value of land and structures held by sole proprietor farms in the \emph{COA} and let $Q_{sp}$ be the value of machinery and equipment held by sole proprietor farms in the \emph{COA}.  Let $R_{p}$ and $Q_{p}$ be the analogous values for farm partnerships in the \emph{COA}.  By an accounting identity, it must be the case that $R_{i}+Q_{i}={FA}_{i}+{LAND}_{i}$ for entity of type $i\in{sp,p}$.  We thus find the ratio of land to capital held by partnerships in the agriculture industry; $\frac{\text{LAND}^{\tau}_{ag,p}}{{\text{LAND}^{\tau}_{ag,p}}+{FA}^{\tau}_{ag,p}}$, where the subscript $ag$ denotes the industry used is agriculture and the subscript $p$ denotes partnership returns. Next, this ratio is multiplied by the value for land and structures, $R_{p}$, and machinery and equipment. $Q_{p}$ for partnerships in the \emph{COA}.  The result is an imputation for the value of land held by farm partnerships: 

\begin{equation}
\text{LAND}_{p}= \frac{\text{LAND}^{\tau}_{ag,p}}{\text{LAND}^{\tau}_{ag,p}+{FA}^{\tau}_{ag,p}}\times (R_{p}+Q_{p})
\end{equation}

To then get an imputation for the value of land held by farm sole proprietorships, we assume that the distribution in the value of land per acre is the same for farm sole proprietorships as it is for farm partnerships.  That is, $\frac{\text{LAND}_{p}}{A_{p}}=\frac{\text{LAND}_{sp}}{A_{sp}}$, where $A_{p}$ and $A_{sp}$ denote the acreage held by farm partnerships and farm sole proprietorships, as reported in the \emph{COA}.  We use this assumption to solve for ${LAND}_{sp}$, given our imputed value for ${LAND}_{p}$ and data on $A_{p}$ and ${A}_{sp}$.  

Finally, we solve for the imputed value of fixed assets held by farm sole proprietorships as: 

\begin{equation}
{FA}_{sp}=R_{sp}+Q_{sp}-\text{LAND}_{sp}
\end{equation}

We then add the values of ${FA}_{sp}$ and ${LAND}_{sp}$ to the values for fixed assets and land that we found for non-farm sole proprietorships in the agriculture industry, ${FA}^{\tau}_{ag,sp}$ and ${FA}^{\tau}_{ag,sp}$.

\textcolor{red}{What do we do about inventories for farm proprietorships?  The \emph{COA} has no information on total inventories (at least in Table 67).   Do we just assume it equals zero?  CBO doesn't mention...}


\subsection{Allocating the BEA Capital Stock Across Corporate and Non-corporate Tax Treatment} 

As noted above, while the BEA does separately account for the capital stock in the corporate and non-corporate sectors, this does not map directly into corporate and non-corporate tax treatment, which is what we want to model.  In particular, the BEA includes S corporations in their corporate sector and includes capital that may be owned by corporate partners in its non-corporate sector.  

Since IRS data will differ from BEA data for a number of reasons (e.g., due measurement and reporting error and non-taxable entities), we use ratios derive from the dollar amounts of capital owned by type of taxable entity. 

\subsubsection{Fixed Assets}
We apportion fixed assets of type $i$ held in industry $m$ across tax treatment, $C\in(\text{corporate},\text{non-corporate})$ by the following methodology.

First, while the BEA provides estimates of the total value of fixed assets owned by nonprofits, households, and non-tax entities, the BEA does not split out the ownership by entity type at the industry or asset type level.  We thus us the ratio of assets reported on tax returns to the value of total BEA assets by industry to come up with a measure of the fraction of assets by industry that are owned by an entity that does not file a business entity tax return (i.e., a Form 1120, 1120S, 1065, 1040 Schedule C, or 1040 Schedule F).  We denote the value of assets of type $i$ held by entities in industry $m$ that file a business tax return as $\widetilde{FA}_{i,m}$, where:

\begin{equation}
\widetilde{FA}_{i,m}={FA}_{i,m}\times \frac{{FA}^{\tau}_{m}}{{FA}_{m}}
\end{equation}

\ \\
\begin{center}
Put table here with $\frac{{FA}^{\tau}_{m}}{{FA}_{m}}$ ratios.
\end{center}
\ \\

\noindent\noindent  ${FA}_{i,m}$ come from the BEA's Detailed Fixed Asset Tables and ${FA}_{m}=\sum_{i}{FA}_{i,m}$.  ${FA}^{\tau}_{m}=\sum_{j}{FA}^{\tau}_{j}$, where $j\in\{c,s,p,sp\}$ is the tax entity type.  The assumption underlining the imputation is that the distribution of asset types across taxable and non-taxable business entities is the same within an industry.

Next, we impute the value of fixed assets by asset type, industry, and tax entity type for taxable business entities.  We denote the amount of fixed asset $i$ held by entities with corporate tax treatment in industry $m$ as $\widehat{FA}_{i,m,C}$, and the fraction of that asset type in industry $m$ held by taxable entities with non-corporate tax treatment as $\widehat{FA}_{i,m,NC}$.  Our assumption for this imputation is that the distribution of asset types within an industry is the same across each type of tax entity.  Therefore we have: 

\begin{equation}
\widehat{FA}_{i,m,C}=\widetilde{FA}_{i,m} \times \frac{{FA}^{\tau}_{m,C}}{{FA}^{\tau}_{m}},
\end{equation} 


\noindent\noindent where ${FA}^{\tau}_{m,C}={FA}^{\tau}_{m,c}+{FA}^{\tau}_{m,p,corp}$, ${FA}^{\tau}_{m,NC}={FA}^{\tau}_{m,s}+{FA}^{\tau}_{m,p}+{FA}^{\tau}_{m,sp}-{FA}^{\tau}_{m,p,corp}$, and ${FA}^{\tau}_{m}={FA}^{\tau}_{m,c}+{FA}^{\tau}_{m,s}+{FA}^{\tau}_{m,p}+{FA}^{\tau}_{m,sp}$.

\ \\
\begin{center}
Put table here with $\frac{{FA}^{\tau}_{m,C}}{{FA}^{\tau}_{m}}$ ratios - percentage of assets by industry/tax treatment.
\end{center}
\ \\

To get the total amount of assets held by the non-corporate sector, we must add to the assets held by non-corporate taxable business entities, the amount held by non-profits and tax-exempts.  Thus we find the total amount of fixed assets held by those receiving non-corporate tax treatment to be ${FA}_{i,m,NC}=\widehat{FA}_{i,m,NC}+({FA}_{i,m}-\widetilde{FA}_{i,m})$.\footnote{I think we want to keep the assets of these non-taxable entities since they do produce goods and services. I think we'll just be adjusting for them with the tax rate we place on that non corporate sector.  That is, to the extent that tax-exempts make up a larger portion, the effective marginal tax rate is lower on that sector.}  The amount going to those with corporate tax treatment is just ${FA}_{i,m,NC}=\widehat{FA}_{i,m,C}$.

NOTE THAT THERE ARE SOME DIFFERENCES IN THE BEA AND SOI INDUSTRY CLASSIFICATIONS.  WHEN NECESSARY USE WE'LL JUST ASSUME THE MINOR INDUSTRY HAS THE SAME DISTRIBUTION OF CAPITAL TYPE/OWNERSHIP ENTITY TYPE AS THE MORE MAJOR INDUSTRY.

\ \\
\textcolor{red}{**We then need to confirm that the total fixed assets apportioned equals total fixed assets measured by the BEA (e.g. in Standard Fixed Asset Table 4.1).  That is, $\sum_{i}\sum_{m}\sum_{C}{FA}_{i,m,C}=FA$. It should, but if not, we can think about an adjustment...}
\ \\


\subsubsection{Inventories}

A similar methodology allocates inventories across industry and tax treatment:  

\begin{equation}
\widetilde{INV}_{m}={INV}_{m}\times \frac{{INV}^{\tau}_{m}}{{INV}_{m}}
\end{equation}

\ \\
\begin{center}
Put table here with $\frac{{INV}^{\tau}_{m}}{{INV}_{m}}$ ratios 
\end{center}
\ \\

\noindent\noindent and 

\begin{equation}
\widehat{INV}_{m,C}=\widetilde{INV}_{m} \times \frac{{INV}^{\tau}_{m,C}}{{INV}^{\tau}_{m}},
\end{equation} 

\ \\
\begin{center}
Put table here with $\frac{{INV}^{\tau}_{m,C}}{{INV}^{\tau}_{m}}$ ratios - percentage of inventories by industry/tax treatment.
\end{center}
\ \\

\noindent\noindent where ${INV}_{m}$ come from the NIPA Table 5.8.5A and aggregates across tax entity types are calculated in the same way as they are for fixed assets.  We then have ${INV}_{i,m,NC}=\widehat{INV}_{m,NC}+({INV}_{m}-\widetilde{INV}_{m})$ and ${INV}_{m,NC}=\widehat{INV}_{m,C}$.  Note that the BEA only reports inventories by very coarse industry categories.  So our assumption is that within those more major industries, the ownership of assets is distributed in the same way as described by the IRS data.

\ \\
\textcolor{red}{CAN WE FIND SOME MORE DETAILED DATA FOR INVENTORIES - I.E. MORE DISAGGREGATED INDUSTRY?  CORP VS NONCORP?}
\ \\

\subsubsection{Land}

Financial Accounts of the United States data give us the total value of land held by households, nonprofits, non-corporate, and corporate businesses.  The definition of non-corporate here includes sole proprietorships, partnerships, and individuals with rental income. The definition of corporate in the Financial Accounts include both subchapter S and subchapter C corporations.  Thus we allocate land across the the two type of corporations and across industry as:

\begin{equation}
{LAND}_{m,c}={LAND}_{c}\times \frac{{LAND}^{\tau}_{m,c}}{{LAND}^{\tau}_{m,s}+{LAND}^{\tau}_{m,c}}
\end{equation}

\begin{equation}
{LAND}_{m,s}={LAND}_{c}\times \frac{{LAND}^{\tau}_{m,s}}{{LAND}^{\tau}_{m,s}+{LAND}^{\tau}_{m,c}}
\end{equation}

\noindent\noindent We use a similar methodology to allocate land across partnerships and sole proprietorships:

\begin{equation}
{LAND}_{m,p}={LAND}_{nc}\times \frac{{LAND}^{\tau}_{m,p}}{{LAND}^{\tau}_{m,p}+{LAND}^{\tau}_{m,sp}}
\end{equation}

\begin{equation}
{LAND}_{m,sp}={LAND}_{nc}\times \frac{{LAND}^{\tau}_{m,sp}}{{LAND}^{\tau}_{m,p}+{LAND}^{\tau}_{m,sp}}
\end{equation}

\ \\
\textcolor{red}{Do we have to net out Sch E stuff from the Financial Account data since we don't know the value of the land for Sch E in our tax data?  If so, how?  Probably easiest to just ignore since Sch E and Sch C filer basically very similar - plus we had to use the partnership to impute sch c land anyway...}
\ \\

\ \\
\textcolor{red}{I think we just ignore that going to HHs and nonprofits, but not sure...}
\ \\

\ \\
\begin{center}
Put tables with ratios as for other types of capital here.
\end{center}
\ \\

Check that total land equals the total from the Financial Account data (it should)...

\section{Depreciation Rates by Asset Type}
\label{sec:depr_rates}

The model uses two depreciation rates: economic depreciation rates and tax depreciation rates.  Economic depreciation rates vary across types of assets, but are not policy parameters.  Rather, rates of economic depreciation are determined by technology and the laws of the physical world.  Tax deprecation rates also vary across asset types, but are policy parameters.  Below, we show how we calibrate the tax depreciation parameters to current law tax policies, but model users may adjust these parameters and we leave as much flexibility as possible for these to be adjusted.  It is perhaps worth noting that policies like bonus depreciation and expensing will be handled through other model parameters, and not directly affect the rate of tax depreciation parameters (perhaps we can refer the reader to a specific section of the model documentation here). 

\subsection{Rates of Economic Deprecation}
Our source for rates of economic deprecation are the \href{http://bea.gov/national/FA2004/Tablecandtext.pdf}{BEA Depreciation Estimates}.  We pull the rates for each asset type, $\delta_{i}$.


\subsection{Current Law Tax Depreciation Rates}
\label{sec:tax_depr_rates}

We find tax deprecation schedules from \href{http://www.irs.gov/pub/irs-pdf/p946.pdf}{IRS Publication 946}, Table B-1.   This table gives the recovery period by class of asset.   \textcolor{red}{HOW DO WE KNOW WHAT THE METHOD IS?  SAME FOR ALL?}  We then create a crosswalk between the IRS asset classes and the BEA asset types using the descriptions of assets.  This crosswalk is available in the IRS\_BEA\_fixedasset\_crosswalk.xls file.  

\ \\
\begin{center}
Table with all possible methods/recovery periods under tax law.
\end{center}
\ \\


\subsection{Approximating Tax Depreciation Rates}
\label{sec:approx_tax_depr_rates}

Our computational model cannot account for the methods of depreciation used for tax purposes exactly because these require knowledge the vintages of capital.  To keep the state space of the model parsimonious, we do not track the amount of capital of each vintage.  We thus approximate the tax depreciation schedule using a geometric deprecation rate that is equivalent in terms of the net present value of depreciation deductions.  The process for creating this equivalence is as follows.  

Under straight-line depreciation, the remaining depreciable value of \$1 invested at any time $y$ is given by: 

\begin{equation}
V(y) =  1-\frac{y}{Y}, 
\end{equation}

\noindent\noindent where $Y$ is the recovery period of the asset.  With a declining balance method of deprecation, the remaining depreciable value of \$1 invested at any time $y$ is given by:

\begin{equation}
V(y) =  e^{-\beta y}, 
\end{equation}

\noindent\noindent where $\beta$ is the rate of decline in value.  Under the declining balance method of depreciation, this rate determined by the rate of acceleration of the straight-line deprecation method for an asset with a recovery period of $Y$ years.  Letting $b$ denote the degree of acceleration of straight-line depreciation, we have $\beta=\frac{b}{Y}$.  For example, for a 200\% declining balance method of an asset with a recovery period of 5 years, $\beta =\frac{2}{5}=40\%$.

To determine when it is advantageous for a filer to switch from the declining balance method to the straight-line method, we must find the point at which the slope of the declining balance method falls below the slope of the straight-line method.  It is at this point that the depreciation deductions from the straight line method exceed those of the declining balance method.  The slope of the remaining depreciable value under the declining balance method is given by:

\ \\
\begin{center}
Do we want a figure like CBO shows where the two curves (for DB and SL) intersect?
\end{center}
\ \\

\begin{equation}
\sigma_{db} = \frac{dV}{dy}=-\beta e^{-\beta y}
\end{equation}

\noindent\noindent The slope of the straight line function depends upon the depreciable basis remaining at the switch.  Therefore, the slope of the depreciable basis for the straight line method is give by:

\begin{equation}
\sigma_{sl} =  \frac{dV}{dy}=\frac{e^{-\beta Y^{*}}}{Y^{*}-Y},
\end{equation}

\noindent\noindent where $Y^{*}$ is the optimal time to switch. We can thus solve for $Y^{*}$ as the point in time at which the slope of the two functions are equal.  The $Y^{*}$ that solves this is given by:

\begin{equation}
Y^{*}=Y\left(1-\frac{1}{b}\right)
\end{equation}

We can now find the present value of depreciation deductions under a declining balance with switch to straight line depreciation method.  To do this, we find the integrals over the two methods for their respective portions of the recovery life.  We find the present value of deprecation deductions, $z_{dbsl}$, to be:

\begin{equation}
z_{dbsl}=\int_{0}^{Y^{*}}\beta e^{-(\beta+r)y}dy+\int_{Y^{*}}^{Y}\frac{e^{-\beta Y^{*}}}{Y^{*}-Y}e^{-ry}dy
\end{equation}

\noindent\noindent which, when integrated, yields:

\begin{equation}
z_{dbsl}=\frac{\beta}{\beta+r}\left[1-e^{-(\beta+r)Y^{*}}\right]+\frac{e^{-\beta Y^{*}}}{(Y-Y^{*})r}\left[e^{-rY^{*}}-e^{-rY}\right]
\end{equation}

\noindent\noindent To find the geometric rate of depreciation with the same net present value of depreciation deductions as under a declining balance with with to straight line, with an acceleration rate of $b$ and a recovery life of $Y$, we solve for, $\delta^{\tau}$:

\begin{equation}
\frac{\delta^{\tau}}{\delta^{\tau}+r}=\frac{\beta}{\beta+r}\left[1-e^{-(\beta+r)Y^{*}}\right]+\frac{e^{-\beta Y^{*}}}{(Y-Y^{*})r}\left[e^{-rY^{*}}-e^{-rY}\right]
\end{equation}

\noindent\noindent Letting $Z$ be the right hand side of the equation above, we thus find $\delta^{\tau}=\frac{r}{(1/Z-1)}$.\footnote{We can also easily solve for the geometric rate when the the tax method is just DB or SL.}

\section{Economic Depreciation Rates by Industry and Sector}
\label{sec:econ_rates_sector}

\ \\
\textcolor{red}{We almost certainly want to calculate the capital stock for several years and create find the average shares of capital by type over those years - so that things the current point in the business cycle doesn't affect the depreciation parameter estimates (e.g. inventories fluctuate a lot over the business cycle).  Note that we'll still use a single base year for other parts of the calibration (e.g. the starting point of the US economy).}
\ \\

To get the economic deprecation rate by industry and sector, we use a weighted average of the economics depreciation rates for each asset type in that industry/sector.  Assume there are $I$ types of fixed assets.  We use the depreciation rate for each of those $I$ types find the weighted average where the weights are determined by the amount of capital of each type.  Following CBO's ``Computing Effective Tax Rates on Capital Income", we assign a economic and tax depreciation rates of zero to inventories and land. Thus the economic depreciation rate for capital in sector $C$ in industry $m$ can be give by:

\begin{equation}
\label{eqn:econ_deprec}
\delta_{m,C}=\frac{\sum_{i=1}^{I}\delta_{i}{FA}_{i,m,C}+{INV}_{m,C}+{LAND}_{m,C}}{K_{m,C}} 
\end{equation}



 

\ \\
\begin{center}
Table of econ deprec rates by industry/sector.
\end{center}
\ \\


\section{Tax Depreciation Rates by Industry and Sector}
\label{sec:tax_rates_sector}

 To get the tax rate of depreciation by industry and sector: 

\begin{equation}
\label{eqn:tax_deprec}
\delta^{\tau}_{m,C}=\frac{\sum_{i=1}^{I}\delta^{\tau}_{i}{FA}_{i,m,C}+{INV}_{m,C}+{LAND}_{m,C}}{K_{m,C}},
\end{equation}

\noindent\noindent where $\delta^{\tau C}_{i}$ is the tax depreciation rate of fixed asset of type $i$ and is given by tax law and our geometric rate approximation described in Section \ref{sec:approx_tax_depr_rates}.

\ \\
\begin{center}
Table of tax deprec rates by industry/sector.
\end{center}
\ \\


\bibliography{TaxModelCalibrationReferences}

\end{document}
