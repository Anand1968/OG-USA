	\documentclass[article,11pt,letterpaper,fleqn]{article}
\usepackage{graphicx,color}
\usepackage{array}
\usepackage{threeparttable}
\usepackage[format=hang,font=normalsize,labelfont=bf]{caption}
\usepackage{colortbl}
\usepackage{multirow}
\usepackage{geometry}
\usepackage{subfigure}
\geometry{letterpaper,tmargin=1in,bmargin=1in,lmargin=1.25in,rmargin=1.25in}
\usepackage{hyperref}
\hypersetup{colorlinks,%
citecolor=red,%
filecolor=red,%
linkcolor=red,%
urlcolor=blue,%
pdftex}
\usepackage{amsmath}
\usepackage{amssymb}
\usepackage{amsthm}
\usepackage{harvard}
\usepackage{setspace}
\usepackage{float,graphicx,color}
\usepackage{appendix}
\usepackage{longtable}
\newtheorem*{thm}{Theorem}
\theoremstyle{definition}
\usepackage{lscape}
\numberwithin{equation}{section}
\newcommand{\cn}{\citeasnoun} % shortens command to cite as noun
\newcommand\ve{\varepsilon}


%\author{Authors here\thanks{Thanks here.}}
\title{Calibration of Household Consumption Parameters in the OLG Dynamic Scoring Model}
\date{\today}


% make tables with smaller sized font 
\makeatletter
\def\table{\@ifnextchar[{\table@i}{\table@i[\fps@table]}}
\def\table@i[#1]{\@float{table}[#1]\footnotesize}
\makeatother



%\setlength{\topmargin}{-0.4in}
%\setlength{\topskip}{0.3in}    % between header and text
%\setlength{\textheight}{9.0in} % height of main text
%\setlength{\textwidth}{6in}    % width of text
%\setlength{\oddsidemargin}{39pt} %even side margin
%\setlength{\evensidemargin}{39pt} %odd side margin

\begin{document}
\bibliographystyle{aer}
\maketitle



\begin{abstract}
This note outlines the process to calibrate the parameters of the households' preferences over consumption goods.
\end{abstract}

\section{Step 1: Acquire the Consumption Data}
\label{sec:step1}

Download the Consumer Expenditure Survey (CEX), years 2000-2013.  You can find the data here: \href{http://www.bls.gov/cex/pumdhome.htm}{http://www.bls.gov/cex/pumdhome.htm}.  You will want to use Stata to manipulate the data, so please download the Stata data files.  Also get the codebook(s) that accompany the data.

\section*{Step 2: Format the Data}
\label{sec:step2}

Once you have each year's file, please append them together in one pooled cross-sectional dataset.  

Next, put all the dollar amounts in constant, year 2013 dollars.  Do this using the consumer price index (CPI).  You can find the CPI deflators here: \\ \href{http://research.stlouisfed.org/fred2/series/CPIAUCSL}{http://research.stlouisfed.org/fred2/series/CPIAUCSL} (using the deflator for all items is fine for now).  When doing this, you'll want to loop over the a list of variables in Stata to make the code more efficient.

I'm not sure of the data, but if there is not a variable for total household income (e.g., the sum of wage and capital income), please create that.

\section*{Step 3: Create Consumption Categories}
\label{sec:step3}

Next, create consumption good categories.  Let the categories be those in Table \ref{tab:cons_goods} below.  Here, you'll want to create a ``cross-walk" document.  This will be a list of the detailed consumption goods in the CEX and which of the consumption goods categories we look at that they map into.

% Table generated by Excel2LaTeX from sheet 'ConsumptionGoodsCategories'
\begin{table}[!h]
  \centering
  \caption{Consumption Goods Categories}
    \begin{tabular}{lll}
    \hline
    \hline
   \# & Consumption Good Category \\
    \hline
    1     & Food  \\
    2     & Alcohol \\
    3     & Tobacco \\
    4     & Household fuels and utilities \\
    5     & Shelter \\
    6     & Furnishings \\
    7     & Appliances \\
    8     & Apparel \\
    9     & Public transportation \\
    10    & New and used cars, fees, and maintenance \\
    11    & Cash contributions and personal care (personal services) \\
    12    & Financial services \\
    13    & Reading and entertainment (recreation) \\
    14    & Household operations (nondurables) \\
    15    & Gasoline and motor oil \\
    16    & Health care \\
    17    & Education \\
     \hline
    \hline
    \end{tabular}%
  \label{tab:cons_goods}%
\end{table}%

\section*{Step 4: Tabulate the Data}
\label{sec:step4}

To see that things are looking like we might expect, do the following tabulations with the data:
\begin{enumerate}
\item Find the average dollar amount spent on each consumption category by calendar year.  Plot these trends.
\item Find the fraction of income spent on each consumption category by calendar year.  Plot these trends.
\item Find the average dollar amount spent on each consumption category by age of head of household.  Plot these life-cycle profiles.
\item Find the fraction of income spent on each consumption category by age of head of household.  Plot these life-cycle profiles.
\item Find the average dollar amount spent on each consumption category by income percentile (so create groups of people for each percentile of household income).  Plot how consumption varies by income.
\item Find the fraction of income spent on each consumption category by income percentile (so create groups of people for each percentile of household income).  Plot how consumption varies by income.
\end{enumerate}

\section*{Step 5: Estimate Consumption Parameters}
\label{sec:step5}

The calibration of the parameters of the composite consumption good is outlined in \cn{FR1993}.  The process of estimating the parameters of the the constant elasticity of substitution (CES) function for preferences over corporate and non-corporate goods and for estimating the parameters of the Stone-Geary function describing preferences over goods from different consumption categories should be directly analogous (with differences being the number of industries considered and the vintage of the data used).

\section*{Step 6: Estimate Transition Matrix to Map Production Goods Income Consumption Goods}
\label{sec:step6}

Note that these categories do not map directly into the production sectors we'll use in the model.  To map production goods into consumption goods we use a fixed coefficient model summarized by the ``transition matrix" $Z$.  To estimate $Z$, we'll use the Survey of Current Business, ``Make and Use Tables".



\section{Notes}
\label{sec:notes}

Steps 1-4 should be relatively straight forward, but I'm happy to answer any questions (\href{mailto:jason.debacker@gmail.com}{jason.debacker@gmail.com}, 770-289-0340).  When we are at Step 5, you'll need to contact me and I can provide more details/guidance.

\bibliography{TaxModelCalibrationReferences}

\end{document}
