	\documentclass[article,11pt,letterpaper,fleqn]{article}
\usepackage{graphicx,color}
\usepackage{array}
\usepackage{threeparttable}
\usepackage[format=hang,font=normalsize,labelfont=bf]{caption}
\usepackage{colortbl}
\usepackage{multirow}
\usepackage{geometry}
\usepackage{subfigure}
\geometry{letterpaper,tmargin=1in,bmargin=1in,lmargin=1.25in,rmargin=1.25in}
\usepackage{hyperref}
\hypersetup{colorlinks,%
citecolor=red,%
filecolor=red,%
linkcolor=red,%
urlcolor=blue,%
pdftex}
\usepackage{amsmath}
\usepackage{amssymb}
\usepackage{amsthm}
\usepackage{harvard}
\usepackage{tikz}
\usepackage{setspace}
\usepackage{float,graphicx,color}
\usepackage{appendix}
\usepackage{longtable}
\newtheorem*{thm}{Theorem}
\theoremstyle{definition}
\usepackage{lscape}
\numberwithin{equation}{section}
\newcommand{\cn}{\citeasnoun} % shortens command to cite as noun
\newcommand\ve{\varepsilon}


\title{Guide to Calibration of Household Consumption Parameters in the OLG Dynamic Scoring Model}
\date{\today}



% make tables with smaller sized font 
\makeatletter
\def\table{\@ifnextchar[{\table@i}{\table@i[\fps@table]}}
\def\table@i[#1]{\@float{table}[#1]\footnotesize}
\makeatother



%\setlength{\topmargin}{-0.4in}
%\setlength{\topskip}{0.3in}    % between header and text
%\setlength{\textheight}{9.0in} % height of main text
%\setlength{\textwidth}{6in}    % width of text
%\setlength{\oddsidemargin}{39pt} %even side margin
%\setlength{\evensidemargin}{39pt} %odd side margin

\begin{document}
\bibliographystyle{aer}
\maketitle



\begin{abstract}
This will be the section in the dynamic scoring model handbook on calibrating parameters of household utility over consumption goods.
\end{abstract}

\section{Consumer Expenditure Survey}

To calibrate the parameters defining household preferences over consumption goods, we use the Consumer Expenditure Survey (CEX).  There are two types of surveys that make up the CEX; the diary surveys and interview surveys.  We use the latter.  

\subsection{Interview Surveys}

Interview Surveys are done quarterly and ask the respondents retrospective questions on expenditures over the prior three-month period.  Households are interviewed for this survey in four consecutive quarters.  We use two waves of data.  One wave includes quarterly surveys from 2012Q1-2012Q4 (thus covering expenditures made from 2011Q4-2012Q3). The second includes surveys from 2012Q2-2013Q1 (covering expenditures from 2012Q1-2012Q4).  This leave us observations of XXXX households for as many as four quarters each.  We aggregate the consumption expenditures over these four quarters to get annual consumption amounts.

From these data we pull variables measuring expenditures on consumption goods by category, household total income (UCC 980000), age of the head of household (AGE\_REF), whether the household received food stamps (900150), household size (FAM\_SIZE), dwelling size (ROOMSQ+BATHRMQ+HLFBATHQ), number of earners in household (NO\_EARNR), and whether the household rents or owns it's housing (HOMEOWN).

\subsection{Sample Selection}

From the two waves of surveys we use, we drop respondents who participate in the survey for less than a year (i.e., have less than four quarters of data).  We further exclude from our sample respondents who received food stamps and those with incomplete income reporting (i.e., for whom we do not observe reported amounts of total income).  These exclusions leave us with observations on 128,769 households.\footnote{Note: current dataset has monthly observations for each consumption code item per household. The total number of observations in the dataset is 30 585 740 after we clean up the items noted in this section. This number will drop a bit after the consumption categories are created.}

\subsection{Creating Consumption Good Categories}

Our model specifies 17 different consumption categories, which well into CEX consumption categories.  Table \ref{tab:cons_categories} describes in detail how we use the CEX data to construct expenditures for these 17 categories.  Note that we excludes  from our data any items noted as ``Assets" or ``Addenda" or otherwise not represented as any of the ISTUB files.    Also note that consumption codes change from year to year, some are added, some are removed and some are amended so one must check throughly to ensure  all relevant consumption codes have been added. The list below is based only on the 2013 ISTUB file. Any particular codes not found in this file may be available in the other files between 2000-2012. 

% Table generated by Excel2LaTeX from sheet 'Sheet1'
\begin{table}[h!]
  \centering
  \caption{Consumption Goods Categories}
    \begin{tabular}{rlll}
    \hline
    \hline
    \multicolumn{1}{c}{Category} &  \multicolumn{1}{c}{Description} &  \multicolumn{1}{c}{CEX variables} &  \multicolumn{1}{c}{Additional categories} \\
    & & &  \multicolumn{1}{c}{from ``Miscellaneous"} \\
    \hline
    1     & Food  & FOODTOTL &  \\
    2     & Alcohol & ALCBEVG &  \\
    3     & Tobacco & TOBACCO &  \\
    4     & Household fuels and utilities & UTILS &  \\
    5     & Shelter & 910050 (for homeowner), &  \\
     &  & RNTDWELL (for renter) \\
    6     & Furnishings & HHFURNSH - MAJAPPL &  \\
    7     & Appliances & MAJAPPL &  \\
    8     & Apparel & APPAREL &  \\
    9     & Public transportation & PUBTRANS &  \\
    10    & New and used cars,  & VEHPURCH + VEHOTHXP &  \\
    & fees, and maintenance & & \\
    11    & Cash contributions and  & PERSCARE + CASHCONT & +680140+680901 \\
    & personal care (personal services) & & \\
    12    & Financial services & INSPENSN & 680210+680220+680902+710110+\\
     & & & 005420+005520+005620+880210+ \\
     & & & 620112 \\
    13    & Reading and entertainment (recreation) & READING + ENTRTAIN & +680904+680905+790600+620926 \\
    14    & Household operations (nondurables) & HHOPER & +620115+900002+680110 \\
    15    & Gasoline and motor oil & GASOIL &  \\
    16    & Health care & HEALTH &  \\
    17    & Education & EDUCATN &  \\
    \hline
    \hline
    \end{tabular}%
  \label{tab:cons_categories}%
\end{table}%


\subsection{Adjustments for Durables and Shelter}

While the consumption categories from the CEX are easily mapped into the categories used in our model, there are some other features of the CEX data that necessitate adjustments.  

\subsubsection{Durables}

Durable consumption goods provide consumers with a stream of consumption services for the life of the good.  By construction, the CEX measures only the initial outlay on durable goods, and not the stream of services from these goods.  The effect is to make the consumption of durables look more lumpy than it actually is.  To mitigate this, we follow \cn{King1979} and \cn{FR1993} to create values for the stream of services from durables.  This is done by averaging expenditures on durables by household income and other household characteristics.  These average values by household type are then used instead of reported expenditures on durables.  \textcolor{red}{Note that for these adjustments to durables and shelter, please create a new variable, e.g., shelter\_adj, so that the original variable is kept.  However, use the new variable that is constructed with the averages over groups as described, when calculating total consumption and when estimating the regression models described below.}

For appliances, we average annual expenditures on appliances (CEX variable MAJAPPL) by income group and number of household members.  Table \ref{tab:impute_appliances} gives the mean amounts for annual expenditures on appliances by household type.

% Table generated by Excel2LaTeX from sheet 'Impute Appliances'
\begin{table}[h!]
  \centering
  \caption{Average Annual Expenditures on Appliances, by Income and Family Size, 2012-2013}
    \begin{tabular}{rrrr}
    \hline
    \hline
    Income (dollars) & One or two members & Three or four members & Five or more members \\
    \hline
    \$1 under \$5,000 &       &       &  \\
    \$5,000 under \$10,000 &       &       &  \\
    \$10,000 under \$15,000 &       &       &  \\
    \$15,000 under \$20,000 &       &       &  \\
    \$20,000 under \$25,000 &       &       &  \\
    \$25,000 under \$30,000 &       &       &  \\
    \$30,000 under \$40,000 &       &       &  \\
    \$40,000 under \$50,000 &       &       &  \\
    \$50,000 under \$75,000 &       &       &  \\
    \$75,000 under \$100,000 &       &       &  \\
    \$100,000 and over &       &       &  \\
    \hline
    \hline
    \end{tabular}%
  \label{tab:impute_appliances}%
\end{table}%

To impute the flow of consumption from furniture, we average annual expenditures on furniture (constructed from CEX variables HHFURNSH - MAJAPPL) by income group, whether the occupant rents or owns the dwelling, and number of rooms in the dwelling.  Table \ref{tab:impute_furniture} gives the mean amounts for annual expenditures on furniture by household type.

% Table generated by Excel2LaTeX from sheet 'Impute Furniture'
\begin{table}[h!]
  \centering
  \caption{Average Annual Expenditures on Furniture, by Income, Dwelling Size, and Tenure, 2012-2013}
    \begin{tabular}{rrrr}
    \hline
    \hline
    Income (dollars) & Up to three rooms & Four or five rooms & Six or more rooms \\
    \hline
    \multicolumn{4}{c}{Owners} \\
    \$1 under \$5,000 &       &       &  \\
    \$5,000 under \$10,000 &       &       &  \\
    \$10,000 under \$15,000 &       &       &  \\
    \$15,000 under \$20,000 &       &       &  \\
    \$20,000 under \$25,000 &       &       &  \\
    \$25,000 under \$30,000 &       &       &  \\
    \$30,000 under \$40,000 &       &       &  \\
    \$40,000 under \$50,000 &       &       &  \\
    \$50,000 under \$75,000 &       &       &  \\
    \$75,000 under \$100,000 &       &       &  \\
    \$100,000 and over &       &       &  \\
    \multicolumn{4}{c}{Renters} \\
    \$1 under \$5,000 &       &       &  \\
    \$5,000 under \$10,000 &       &       &  \\
    \$10,000 under \$15,000 &       &       &  \\
    \$15,000 under \$20,000 &       &       &  \\
    \$20,000 under \$25,000 &       &       &  \\
    \$25,000 under \$30,000 &       &       &  \\
    \$30,000 under \$40,000 &       &       &  \\
    \$40,000 under \$50,000 &       &       &  \\
    \$50,000 under \$75,000 &       &       &  \\
    \$75,000 under \$100,000 &       &       &  \\
    \$100,000 and over &       &       &  \\
    \hline
    \hline
    \end{tabular}%
  \label{tab:impute_furniture}%
\end{table}%

We impute the flow of consumption from motor vehicles by averaging annual expenditures on the durable component of motor vehicles (CEX variable VEHPURCH) by income group and number of earners.  Table \ref{tab:impute_motor} gives the mean amounts for annual expenditures on motor vehicles by household type.

% Table generated by Excel2LaTeX from sheet 'Impute Motor Vehicles'
\begin{table}[h!]
  \centering
  \caption{Average Annual Expenditures on Motor Vehicles, by Income and Number of Earners, 2012-2013}
    \begin{tabular}{rrrr}
    \hline
    \hline
    Income (dollars) & No earners & One earner & Two or more earners \\
    \hline
    \$1 under \$5,000 &       &       &  \\
    \$5,000 under \$10,000 &       &       &  \\
    \$10,000 under \$15,000 &       &       &  \\
    \$15,000 under \$20,000 &       &       &  \\
    \$20,000 under \$25,000 &       &       &  \\
    \$25,000 under \$30,000 &       &       &  \\
    \$30,000 under \$40,000 &       &       &  \\
    \$40,000 under \$50,000 &       &       &  \\
    \$50,000 under \$75,000 &       &       &  \\
    \$75,000 under \$100,000 &       &       &  \\
    \$100,000 and over &       &       &  \\
    \hline
    \hline
    \end{tabular}%
  \label{tab:impute_motor}%
\end{table}%



\subsubsection{Shelter}

As noted in Table \ref{tab:cons_categories}, we create the shelter consumption category using the rental equivalent value of owner occupied housing.  This better approximates the flow of consumption services from owner-occupied housing than do mortgage payments.  However, since the rent includes the cost of taxes and property maintenance and repairs, we ignore those homeowner related expenses to avoid double counting.

\subsection{Underreporting}

\cn{Gieseman1987} finds evidence that survey evidence of consumption is inaccurate, especially for categories of consumption for which there is some stigma, such as alcohol and tobacco consumption.  He finds the amount of underreporting across consumption categories by comparing the CEX control totals to the control totals from the Personal Consumption Expenditure component of the National Income and Product Accounts (NIPA).  We follow \cn{King1979} and \cn{FR1993} and scale up the CEX reported expenditure amounts to control for this underreporting.  The scalars used are summarized in Table \ref{tab:CEX_NIPA}, which reports the CEX expenditure totals as a proportion of the NIPA totals.  Note that scaling up each household expenditure by the scalar for each category is technically only appropriate if respondents tend to underreport expenditures rather than omit them altogether.  Still, this procedure should be helpful in getting us closer to the true amount of consumption by category.

To find these shares, we first calculate aggregate expenditure amounts for each consumption category using the population weights in the CEX.  Next, we find the aggregate amounts for each category as reporting in the NIPA tables.  Finally, we use these two sets of numbers to calculate the ratio for each consumption category.

\textcolor{red}{I think the best source for NIPA consumption data by category is from NIPA Table 2.4.5U} (\href{http://www.bea.gov/iTable/iTable.cfm}{http://www.bea.gov/iTable/iTable.cfm}).  You'll need to use some judgement to map the categories in this table into those we create from the CEX.  Please make clear what you do here.

% Table generated by Excel2LaTeX from sheet 'CEX vs NIPA'
\begin{table}[h!]
  \centering
  \caption{Reported 2012 CEX Expenditure Totals as a Proportion of NIPA Estimated Totals}
    \begin{tabular}{lrr}
    \hline
    \hline
    Category & Description & Value \\
    \hline
    1     & Food  &  \\
    2     & Alcohol &  \\
    3     & Tobacco &  \\
    4     & Household fuels and utilities &  \\
    5     & Shelter &  \\
    6     & Furnishings &  \\
    7     & Appliances &  \\
    8     & Apparel &  \\
    9     & Public transportation &  \\
    10    & New and used cares, fees, and maintenance &  \\
    11    & Cash contributions and personal care (personal services) &  \\
    12    & Financial services &  \\
    13    & Reading and entertainment (recreation) &  \\
    14    & Household operations (nondurables) &  \\
    15    & Gasoline and motor oil &  \\
    16    & Health care &  \\
    17    & Education &  \\
    \hline
    \hline
    \end{tabular}%
  \label{tab:CEX_NIPA}%
\end{table}%


\section{Estimating the Consumption Parameters}

Given the Stone-Geary preferences over different consumption goods, demand for good $i$ by an individual of age $t$ can be written as:

\begin{equation}
\label{eqn:c_demand}
c_{it}=b_{it} + \frac{\beta_{it}}{p_{i}}\left(\tilde{p}_{t}\tilde{c}_{t}\sum_{j=1}^{N}p_{i}b_{it}\right),
\end{equation}

\noindent\noindent where $c_{it}$ is consumption of good $i$ at age $t$, $b_{it}$ is the minimum required consumption of good $i$ at age $t$, $\beta_{it}$ is the share parameter for good $i$ at age $t$, $p_{i}$ is the price of good $i$, $\tilde{c}_{t}$ is the amount of composite consumption good consumed at age $t$, and $\tilde{p}_{t}$ is the price of the age $t$ composite consumption good.  Note that in the CEX, we only observe total expenditures (i.e., price times quantity) and not prices of consumption goods.  Thus, we multiply both sides of Equation \ref{eqn:c_demand} by $p_{i}$ to obtain an equation defining expenditures across consumption categories:

\begin{equation}
\label{eqn:cp_demand}
p_{i}c_{it}=exp_{it}=p_{i}b_{it} + \beta_{it}\left(m_{t}\sum_{j=1}^{N}p_{i}b_{it}\right)
\end{equation}

\noindent\noindent Here, $exp_{it}$ is the total expenditure on good $i$ at age $t$ and $m_{t}=\tilde{p}_{t}\tilde{c}_{t}$ is defined as total consumption expenditures at age $t$.  Note that we observe both $exp_{it}$ and $m_{t}$ in the CEX.  However, we do not observe $p_{i}b_{it}$.  We thus further rearrange the equation to find: 

\begin{equation}
\label{eqn:exp_eqn}
exp_{it}=\kappa_{it} + \beta_{it}m_{t}+\varepsilon_{it}, \ \ \varepsilon_{it}\sim N(0,\sigma^{2}),
\end{equation}

\noindent\noindent where $\kappa_{it}$ is given by:

\begin{equation}
\label{eqn:kappa}
\kappa_{it}=p_{i}b_{it} - \beta_{it}\sum_{j=1}^{N}p_{i}b_{it}
\end{equation}

\noindent\noindent  We estimate Equation \ref{eqn:exp_eqn} by ordinary least squares (OLS), run separately for each of the 17 consumption good categories and 12 age groups (given in Table \ref{tab:ages}).  Regressions are run using the population weights for each observation.  With the parameter estimates in hand, we then use the intercept terms, $\kappa_{it}$, the share parameters, $\beta_{it}$, and Equation \ref{eqn:kappa} to solve for the $p_{t}b_{it}$.  In doing this, note that we have $17\times 12$ equations (the 17 consumption categories times the 12 age groups).  However, since the share parameters $\beta_{it}$ must sum to one (i.e., $\sum_{t=1}^{T}\beta_{it}=1$), we only have $16\times 12$ unique equations.  Thus we need an additional identifying equation for each age group.  \cn{King1979} tries several and finds similar results for each. Thus we follow \cn{King1979} and \cn{FR1993} and  use the identifying assumption that $\sum_{j=1}^{N}p_{i}b_{it}=\$32,000$.  The amount of total minimum required purchases across each age group is found by taking the value of \$8,000 from \cn{FR1993} and adjusting it for growth and inflation by multiplying it by the ratio of nominal GDP in 2012 (\$16,163.2 billion) to nominal GDP in 1984 (\$4,040.7 billion). \textcolor{red}{This minimum amount seems high, so we might want to try some other assumptions like assuming that the minimum required expenditures on alcohol are zero.}

% Table generated by Excel2LaTeX from sheet 'Age Groups'
\begin{table}[h!]
  \centering
  \caption{Age Groups Used in Estimation}
    \begin{tabular}{rr}
    \hline
    \hline
    Group & Ages \\
    \hline
    1     & 20-24 \\
    2     & 25-29 \\
    3     & 30-34 \\
    4     & 35-39 \\
    5     & 40-44 \\
    6     & 45-49 \\
    7     & 50-54 \\
    8     & 55-59 \\
    9     & 60-64 \\
    10    & 65-69 \\
    11    & 70-74 \\
    12    & 75+ \\
    \hline
    \hline
    \end{tabular}%
  \label{tab:ages}%
\end{table}%


Note that the solution to Equation \ref{eqn:kappa} gives us $p_{i}b_{it}$, not the parameter we are interested in, $b_{it}$.  To uncover $b_{it}$ we normalize the units of each consumption good to be the amount of good that can be purchased with one dollar.  Thus $p_{i}=1, \ \ \forall i=1,...,N$ and $p_{i}b_{it}=b_{it}$.  This normalization is consistent with our benchmark equilibrium where we normalize the prices of each consumption good to be one.

This exercise yields us estimates for $17\times12\times2$ parameter estimates (i.e., $\{\beta_{it},b_{it}\}_{i=1,t=1}^{N,T}$).



%% may want to put something in here as Fullerton does about OLS efficient, unbiased, assumptions needed, and about seemingly unrelated regressions

%% Note that may want to estimate separately for married, single status if we put that in theory model


\section{Parameter Estimates}

Tables with estimates?  Maybe too many to present...  Could separate by age group as in \cn{FR1993}, Tables 5-6 and 5-7.

% Table generated by Excel2LaTeX from sheet 'Parameter Compare'
\begin{table}[h!]
  \centering
  \caption{Comparison of Linear Expenditure System Parameters for 2012-2013 with King's Parameters from 1972-73 and Fullerton and Roger's from 1984-85}
    \begin{tabular}{rlrrrrrrrr}
    \hline
    \hline
          &       & \multicolumn{3}{c}{Share ($\beta_{it}$)} & \multicolumn{5}{c}{Minimum expenditure ($p_{i}b_{it}$)} \\
    \hline
       &  & & & &Adjusted,  & Actual, & Adjusted, & Actual,  \\
    Category & Description & 1972-73 & 1984-85 & 2012-13 & 1972-73 & 1984-85 & 1984-85 &  2012-13 & 2012-13 \\
    1     & Food  & 0.1508 & 0.1565 &       & 816   & 555   & 1832  &       &  \\
    2     & Alcohol & 0.0273 & 0.0293 &       & -3    & 22    & 73    &       &  \\
    3     & Tobacco & 0.0073 & 0.0058 &       & 122   & 71    & 234   &       &  \\
    4     & Household fuels and utilities & 0.0928 & 0.0406 &       & 282   & 307   & 1013  &       &  \\
    5     & Shelter & 0.1440 & 0.1448 &       & 842   & 724   & 2388  &       &  \\
    6     & Furnishings & 0.0278 & 0.0239 &       & 122   & 76    & 251   &       &  \\
    7     & Appliances & 0.0079 & 0.0257 &       & 99    & 144   & 375   &       &  \\
    8     & Apparel & 0.0889 & 0.1141 &       & 9     & -32   & -106  &       &  \\
    9     & Public transportation & 0.0156 & 0.0293 &       & -15   & -57   & -188  &       &  \\
    10    & New and used cares, fees,  & 0.0788 & 0.0869 &       & 338   & 435   & 1435  &       &  \\
     & and maintenance & & & & & & &  \\
    11    & Cash contributions and& 0.2046 & 0.1595 &       & -581  & -319  & -1052 &       &  \\
         &  personal care (personal services)  & & & & & & &  \\
    12    & Financial services & 0.0843 & 0.0396 &       & 59    & 93    & 306   &       &  \\
    13    & Reading and entertainment (recreation) & 0.0745 & 0.0533 &       & -114  & 28    & 92    &       &  \\
    14    & Household operations (nondurables) & 0.0211 & 0.0095 &       & 241   & 154   & 509   &       &  \\
    15    & Gasoline and motor oil & 0.029 & 0.0341 &       & 237   & 229   & 754   &       &  \\
    16    & Health care & -     & 0.0273 &       & -     & 142   & 470   &       &  \\
    17    & Education & -     & 0.0197 &       & -     & 41    & -135  &       &  \\
    \hline
    \hline
    \end{tabular}%
  \label{tab:est_compare}%
\end{table}%


\bibliography{cons_calib_bib}

\end{document}