	\documentclass[article,11pt,letterpaper,fleqn]{article}
\usepackage{graphicx,color}
\usepackage{array}
\usepackage{threeparttable}
\usepackage[format=hang,font=normalsize,labelfont=bf]{caption}
\usepackage{colortbl}
\usepackage{multirow}
\usepackage{geometry}
\usepackage{subfigure}
\geometry{letterpaper,tmargin=1in,bmargin=1in,lmargin=1.25in,rmargin=1.25in}
\usepackage{hyperref}
\hypersetup{colorlinks,%
citecolor=red,%
filecolor=red,%
linkcolor=red,%
urlcolor=blue,%
pdftex}
\usepackage{amsmath}
\usepackage{amssymb}
\usepackage{amsthm}
\usepackage{harvard}
\usepackage{setspace}
\usepackage{float,graphicx,color}
\usepackage{appendix}
\usepackage{longtable}
\newtheorem*{thm}{Theorem}
\theoremstyle{definition}
\usepackage{lscape}
\numberwithin{equation}{section}
\newcommand{\cn}{\citeasnoun} % shortens command to cite as noun
\newcommand\ve{\varepsilon}


\author{Authors here\thanks{Thanks here.}}
\title{Notes on the Piketty/Wealth Tax Project}
\date{\today}


% make tables with smaller sized font 
\makeatletter
\def\table{\@ifnextchar[{\table@i}{\table@i[\fps@table]}}
\def\table@i[#1]{\@float{table}[#1]\footnotesize}
\makeatother



%\setlength{\topmargin}{-0.4in}
%\setlength{\topskip}{0.3in}    % between header and text
%\setlength{\textheight}{9.0in} % height of main text
%\setlength{\textwidth}{6in}    % width of text
%\setlength{\oddsidemargin}{39pt} %even side margin
%\setlength{\evensidemargin}{39pt} %odd side margin

\begin{document}
\bibliographystyle{aer}
\maketitle



\begin{abstract}
This paper will look at the effects of a wealth tax on the distribution of income and wealth and on economic performance.
\end{abstract}



\section{Mechanisms of a wealth tax}
\label{sec:mechanics}

Taxing wealth should provide a disincentive for households to accumulate wealth.  A tax on wealth is equivalent to a very high (over 100\%) tax on capital (we may also want to consider a tax on bequests like in Piketty and Saez (\emph{Econometrica}, 2013)). Because the receipts from the wealth tax will be redistributed in a lump sum manner, the policy will also have a direct effect on the distribution of wealth and income.  The major mechanisms are:
\begin{itemize}
\item Higher wealth tax $\implies$ less $K$
	\begin{itemize}
	\item This harms everyone since there is less output
	\item Lower $K$ $\implies$ lower $MPL$, so wages fall (thus disproportionately affecting those who rely more on labor income)
	\item Lower $K$ $\implies$ higher $MPK$, so wages fall (thus disproportionately affecting those who rely more on labor income)
		\begin{itemize}
		\item I don't know if the result holds here, but Panousi (\emph{JME}, 2012) finds that in general equilibrium a capital tax can increase $K$ because while the tax lowers the need for precautionary savings, the higher return on capital can actually increase the capital stock.
		\end{itemize}
	\end{itemize}
\item The tax on wealth makes it more costly to consumption smooth (both from precautionary savings due to income shocks and to move income around in lifetime due to the lifecycle profile of earnings)
	\begin{itemize}
	\item But these problems are at least somewhat mitigated by the fact that the receipts from the wealth tax are to be redistributed lump sum.
	\end{itemize}
\item Effects on the distribution of wealth:
	\begin{itemize}
	\item Even if bequests are distributed lump sum over the entire population, the tax can still speed up that process of reallocating capital across the economy.
	\item If bequests go to those of a similar type, the wealth tax is especially helpful in reducing inequality.
	\end{itemize}
\end{itemize}

The main trade-off is thus efficiency vs. equity.  A higher wealth tax equalizes the distribution of income and wealth, but at the cost of a lower capital stock and a less productive economy.  In fact, it may be the case that the wealth tax disproportionately affects those with less wealth who rely more on labor income due to the effects on the wage and interest rates.

Note that with heterogeneous returns to capital income, as in Guvenen et. al (unpublished, 2014), we could see substantially different effects of the wealth tax on market efficiency.

\section{What does stochastic income buy us?}

\begin{itemize}
\item More savings because of income risk.
\item Wealth tax imposes more costs because makes precautionary savings more costly 
	\begin{itemize}
	\item But this may not be an issue since the wealth tax is redistributed lump sum (I'm not sure it solve the problem completely since the transfer is the same to all and some might get a bigger income shock and so have more need for savings).
	\end{itemize}
\item As noted above, stochastic returns to capital might be of more interest in the model, but are much harder to estimate.
\end{itemize}

In short, stochastic income buys us some realism and forces savings up, but doesn't give us a lot.  Also, we might still want an income fixed effect so that some people have higher expected lifetime incomes.  This would be helpful in how we think about the transfer of bequests as well.  If income is completely stochastic - which type is an agent and which young agents are the same type?  Do we just use the type in the last year of earnings?


\section{Modeling notes}

Things we might want to think about:
\begin{itemize}
\item Do we want an ability ``fixed effect" - so that some households have higher expected lifetime earnings than others?
\item Do we want a borrowing constraint - so that agents can save, but not borrow (or only borrow to some limit)?
\end{itemize}

\section{Output for the paper}

What kinds of things do we want to look at (to be put in graphs/tables in the paper)?

\begin{itemize}
\item Baseline case (i.e. calibrated economy, no wealth tax)
	\begin{itemize}
	\item How does wealth and income get concentrated (plot transition path) given different assumptions about where bequests go
		\begin{itemize}
		\item NOTE: What is the SS in the baseline case where bequests go to agents of the same type (with differences in expected lifetime incomes)?  Doesn't wealth/income just get more and more concentrated?  I think it does, but should should still settle on a stationary distribution where almost all (but not all, since others will have precautionary savings) wealth is held by the highest ability type.
		\end{itemize}
	\item How important are labor and capital income in describing the cross-sectional variance in household income over the transition path? (e.g. do we see that labor income is important early on, but that capital income become more important later (as Piketty suggests)).
	\end{itemize}
\item Cases with wealth tax (experiment with several of different degrees of progressivity):
	\begin{itemize}
	\item Dynamics of wealth accumulation/income inequality (plot transition patch for concentration of wealth/income (e.g. fraction held by/going to top 10\%, top 1\%))
	\item Calculate utilitarian social welfare under the baseline and wealth taxes
	\item Find the expected utility for each ability type under the wealth taxes and baseline and compare
		\begin{itemize}
		\item How does the wealth tax differentially affect people in different parts of the ability distribution?
		\item Do the distributional effects differ over the transition path?
		\end{itemize}
	\item Plot the marginal products of capital and labor over the transition path
	\item Can we say something about the preferred wealth tax for someone who is ``behind the veil"?
	\end{itemize}
\end{itemize}

\section{A wealth tax in a simple 3-period model}

It should act just like a capital tax (see Saez capital tax notes in references).  But we could also consider a tax on bequests (I'm not sure what exactly Piketty talks about in his book or what Auerbach and Hassett will want to focus on).  We should put some thought into this.

\section{Estimating the stochastic ability process}

We will estimate the stochastic ability process using two datasources.  We want to use tax data due to the absence of top coding in these data and the ability to identify the tails of the distribution.  However, tax data do not provide hours of work, which are necessary to back our ability/productivity.  Thus, we will use the CPS to impute hours for individuals in the tax data.  The process will go something like this:

\begin{enumerate}
\item From the CPS, run following regressions (separately for each year 1987-2010 (maybe 2011) ):  

\begin{equation}
\label{eqn:hours_reg}
\begin{split}
hours_{it} = &  \alpha_{1}ln(labor\_income_{it}) + \alpha_{2}ln(labor\_income_{it})^{2} + \alpha_{3}ln(spouse\_labor_income_{it}) + \\
& \alpha_{4}ln(spouse\_labor_income_{it})^{2} + \alpha_{5}ln(selfemp\_income_{it}) + \alpha_{6}ln(spouse\_selfemp\_income_{it}) + \\
& \alpha_{7}children_{it} + \alpha_{8}age_{it} + \alpha_{9}age\_spouse_{it} + \varepsilon_{it}
\end{split}
\end{equation}
	\begin{itemize}
	\item We'll let age enter as a series of dummy variables
	\item Children is just a dummy variable for children under 18 in the household
	\item The logs of self-employment income may be problematic.  To deal with this perhaps we can run regressions separately for those who have self-employment income.
	\item We'll need to be careful to choose income variables that correspond to those in the tax data.
	\end{itemize}
\item Use the coefficients from the regression in Equation \ref{eqn:hours:reg} to predict hours for each individual in the Continuous Work History Sample (CWHS) (the panel of tax data using a 1/5000 random sample of filers)
\item  Find ``ability" (equivalently, hourly wages) by  dividing income by the imputed hours
	\begin{itemize}
	\item We'll probably do this both for labor income (i.e. wages and salaries from Form 1040) and then also for earned income (wages and salaries plus business income)
	\end{itemize}
\item Run the regression: $ability_{it} = a_{i} + age_{it} + \eta_{it}$ (may want to use $ln(ability)$ so that error term is more likely to be normally distributed)
	\begin{itemize}
	\item Age is entered as dummy variables so that we can get mean earnings by age (to construct a life-cycle profile of earnings)
	\item We may or may not estimate the fixed effect, $a_{i}$.  
	\item If we do, we'll want to estimate the distribution of this fixed effect.
	\end{itemize}
\item Find the transition matrix for the fraction moving from/to each percentile of $\eta_{it}$ from one year to the next (and the mean values for each percentile bin) 
	\begin{itemize}
	\item We'll average this over a few years, say 2000-2010 or 2011.
	\end{itemize}
\end{enumerate}

\end{document}
