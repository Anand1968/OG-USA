\documentclass[letterpaper,12pt]{article}

  \usepackage{threeparttable}
  \usepackage{geometry}
  \geometry{letterpaper,tmargin=1in,bmargin=1in,lmargin=1.25in,rmargin=1.25in}
  \usepackage[format=hang,font=normalsize,labelfont=bf]{caption}
  \usepackage{amsmath}
  \usepackage{multirow}
  \usepackage{array}
  \usepackage{delarray}
  \usepackage{amssymb}
  \usepackage{amsthm}
  \usepackage{lscape}
  \usepackage{natbib}
  \usepackage{setspace}
  \usepackage{float,color}
  \usepackage[pdftex]{graphicx}
  \usepackage{hyperref}
  \usepackage{placeins}
  \hypersetup{colorlinks,linkcolor=red,urlcolor=blue,citecolor=red}
  \theoremstyle{definition}
  \newtheorem{theorem}{Theorem}
  \newtheorem{acknowledgement}[theorem]{Acknowledgement}
  \newtheorem{algorithm}[theorem]{Algorithm}
  \newtheorem{axiom}[theorem]{Axiom}
  \newtheorem{case}[theorem]{Case}
  \newtheorem{claim}[theorem]{Claim}
  \newtheorem{conclusion}[theorem]{Conclusion}
  \newtheorem{condition}[theorem]{Condition}
  \newtheorem{conjecture}[theorem]{Conjecture}
  \newtheorem{corollary}[theorem]{Corollary}
  \newtheorem{criterion}[theorem]{Criterion}
  \newtheorem{definition}{Definition} % Number definitions on their own
  \newtheorem{derivation}{Derivation} % Number derivations on their own
  \newtheorem{example}[theorem]{Example}
  \newtheorem{exercise}[theorem]{Exercise}
  \newtheorem{lemma}[theorem]{Lemma}
  \newtheorem{notation}[theorem]{Notation}
  \newtheorem{problem}[theorem]{Problem}
  \newtheorem{proposition}{Proposition} % Number propositions on their own
  \newtheorem{remark}[theorem]{Remark}
  \newtheorem{solution}[theorem]{Solution}
  \newtheorem{summary}[theorem]{Summary}
  \numberwithin{equation}{section}
  \bibliographystyle{aer}
  \newcommand\ve{\varepsilon}
  \renewcommand\theenumi{\roman{enumi}}
  \providecommand{\norm}[1]{\lVert#1\rVert}

\begin{document}

%titlepage
\begin{titlepage}
  \title{Dyanmic General Equilibrim Tax Scoring with Micro Tax Simulations
         \thanks{} }

  \author{ Richard W. Evans\footnote{Brigham Young University, Department of Economics, 167 FOB, Provo, Utah 84602, (801) 422-8303, \href{mailto:revans@byu.edu}{revans@byu.edu}.} \\[-2pt]
         \and
         Kerk L. Phillips\footnote{Brigham Young University, Department of Economics, 166 FOB, Provo, Utah 84602, \href{mailto:kerk_phillips@byu.edu}{kerk\_phillips@byu.edu}.} \\[-2pt]}
  \date{June 2014 \\
        \scriptsize{(version 14.06.a)}\\
        \small Preliminary and incomplete; please do not cite.}
  \maketitle
  \begin{abstract}
  \small{This paper ...

  \vspace{0.3in}

  \textit{keywords:} dynamic general equilibrium, taxation, numerical simulation, computational techniques, simulation modeling.

  \vspace{0.3in}

  \textit{JEL classifications:} C63, C68, E62, H24, H25, H68}
  \end{abstract}
  \thispagestyle{empty}
\end{titlepage}

\begin{spacing}{1.5}


\section{Introduction}\label{Sec_Intro}

\section{Details of the Macro Model}\label{Sec_Macro}

  \subsection{Deomgraphics}

  \subsection{Households}

  \subsection{Firms}

  \subsection{Market Clearing}

  \subsection{Solution and Simulation}
    

\newpage

\section{Incorporating Feedbacks with Micro Tax Simulations}\label{SecMicro}

  Follow this algorthim:
  \begin{itemize}
    \item Period 1
    \begin{itemize}
      \item Use current IRS public use sample.
      \item Run the following within-period routine
      \begin{itemize}
        \item Do the static tax analysis of this sample, save the results
        \item Summarize the public use sample by aggregating into bins over age and earnings ability
        \item Use this as a starting point for the dynamic macro model
        \item Get values for fundamental interest rates and effective wages for next period
      \end{itemize}
    \end{itemize}
  \item Period 2
    \begin{itemize}
      \item “Age” the public use data demographically by one year.
      \item Let wages and interest rates rise by the amounts predicted in the macro model.
      \item Rerun the within-period routine
    \end{itemize}
  \item Iterate over periods until end of forecast period is reached.
  \end{itemize}

\section{Calibration}
  \subsection{Tax Bend Points}
      We use IRS data which summarizes individual tax returns for 2011 by 19 income categories and 4 filing statuses.  For each filing status we fit the mapping from reported income into adjusted gross income (AGI) using a sufficiently high-order polynomial.  We then use this function to solve for the income level which corresponds to each of the five bend points in the tax code for each filing type.
      \begin{table}[ht]
        \caption{AGI and Income Bend Points}
        \label{Calib_Bend_Tab1}
        \centering
        AGI Bend Points
        \begin{tabular}{|r|r|r|r|r|} \hline 
          Tax rate & Married Joint & Married Separate & Head of Household & Single \\ \hline 
          10\% & 17,400 & 8700 & 12,400 & 8700 \\ \hline 
          15\% & 70,700 & 35,350 & 47,350 & 35,350 \\ \hline 
          25\% & 142,700 & 71,350 & 122,300 & 85,650 \\ \hline 
          28\% & 217,450 & 108,725 & 198,050 & 178,650 \\ \hline 
          33\% & 388,350 & 194,175 & 388,350 & 388,350 \\ \hline 
        \end{tabular}
        \\
        Corresponding Reported Income Bendpoints
        \begin{tabular}{|r|r|r|r|r|} \hline 
          Tax rate & Married Joint & Married Separate & Head of Household & Single \\ \hline 
          0\%  & 5850  & 91 & 756 & 1435 \\ \hline 
          10\% & 22,932 & 8591 & 12,911 & 9956 \\ \hline 
          15\% & 75,181 & 34,592 & 47,023 & 36,021 \\ \hline 
          25\% & 145,866 & 69,768 & 120,200 & 85,244 \\ \hline 
          28\% & 219,162 & 106,245 & 194,176 & 176,270 \\ \hline 
          33\% & 386,798 & 189,674 & 380,043 & 381,524 \\ \hline 
        \end{tabular}
      \end{table}

      We then fit a bivariate probabililty density function over income and filing type from the data.  For each bendpoint we calculate the probability density at that bendpoint and use these as weights in a weighted average over filing types to generate an aggregate bendpoint.
      \begin{table}[ht]
        \caption{Aggregated Bend Points}
        \label{Calib_Bend_Tab2}
        \centering
        \begin{tabular}{|r|r|} \hline 
          Tax rate & Bend Point \\ \hline 
          0\% & 2889 \\ \hline 
          10\% & 15,116 \\ \hline 
          15\% & 52,580 \\ \hline 
          25\% & 114,552 \\ \hline 
          28\% & 196,201 \\ \hline 
          33\% & 380,657 \\ \hline 
        \end{tabular}
      \end{table}

\section{Conclusion}\label{SecConclusion}


\end{spacing}


\newpage
\renewcommand{\theequation}{A.\arabic{section}.\arabic{equation}}
                                                 % redefine the command that creates the equation number
\renewcommand{\thedefinition}{A.\arabic{section}.\arabic{definition}}
                                                 % redefine the command that creates the definition number
\renewcommand{\thesection}{A-\arabic{section}}   % redefine the command that creates the section number

\setcounter{equation}{0}                         % reset counter
\setcounter{section}{0}                          % reset section number
\section*{TECHNICAL APPENDIX}

\setcounter{definition}{0}
\setcounter{equation}{0}                         % reset counter




\newpage
\bibliography{AEIBYUDyn}


\end{document}
