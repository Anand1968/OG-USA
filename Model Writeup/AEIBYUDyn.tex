\documentclass[letterpaper,12pt]{article}

  \usepackage{threeparttable}
  \usepackage{geometry}
  \geometry{letterpaper,tmargin=1in,bmargin=1in,lmargin=1.25in,rmargin=1.25in}
  \usepackage[format=hang,font=normalsize,labelfont=bf]{caption}
  \usepackage{amsmath}
  \usepackage{multirow}
  \usepackage{array}
  \usepackage{delarray}
  \usepackage{amssymb}
  \usepackage{amsthm}
  \usepackage{lscape}
  \usepackage{natbib}
  \usepackage{setspace}
  \usepackage{float,color}
  \usepackage[pdftex]{graphicx}
  \usepackage{hyperref}
  \usepackage{placeins}
  \hypersetup{colorlinks,linkcolor=red,urlcolor=blue,citecolor=red}
  \theoremstyle{definition}
  \newtheorem{theorem}{Theorem}
  \newtheorem{acknowledgement}[theorem]{Acknowledgement}
  \newtheorem{algorithm}[theorem]{Algorithm}
  \newtheorem{axiom}[theorem]{Axiom}
  \newtheorem{case}[theorem]{Case}
  \newtheorem{claim}[theorem]{Claim}
  \newtheorem{conclusion}[theorem]{Conclusion}
  \newtheorem{condition}[theorem]{Condition}
  \newtheorem{conjecture}[theorem]{Conjecture}
  \newtheorem{corollary}[theorem]{Corollary}
  \newtheorem{criterion}[theorem]{Criterion}
  \newtheorem{definition}{Definition} % Number definitions on their own
  \newtheorem{derivation}{Derivation} % Number derivations on their own
  \newtheorem{example}[theorem]{Example}
  \newtheorem{exercise}[theorem]{Exercise}
  \newtheorem{lemma}[theorem]{Lemma}
  \newtheorem{notation}[theorem]{Notation}
  \newtheorem{problem}[theorem]{Problem}
  \newtheorem{proposition}{Proposition} % Number propositions on their own
  \newtheorem{remark}[theorem]{Remark}
  \newtheorem{solution}[theorem]{Solution}
  \newtheorem{summary}[theorem]{Summary}
  \numberwithin{equation}{section}
  \bibliographystyle{aer}
  \newcommand\ve{\varepsilon}
  \renewcommand\theenumi{\roman{enumi}}
  \providecommand{\norm}[1]{\lVert#1\rVert}

\begin{document}

%titlepage
\begin{titlepage}
  \title{Dyanmic General Equilibrim Tax Scoring with Micro Tax Simulations
         \thanks{} }

  \author{ Richard W. Evans\footnote{Brigham Young University, Department of Economics, 167 FOB, Provo, Utah 84602, (801) 422-8303, \href{mailto:revans@byu.edu}{revans@byu.edu}.} \\[-2pt]
         \and
         Kerk L. Phillips\footnote{Brigham Young University, Department of Economics, 166 FOB, Provo, Utah 84602, \href{mailto:kerk_phillips@byu.edu}{kerk\_phillips@byu.edu}.} \\[-2pt]}
  \date{June 2014 \\
        \scriptsize{(version 14.06.a)}\\
        \small Preliminary and incomplete; please do not cite.}
  \maketitle
  \begin{abstract}
  \small{This paper ...

  \vspace{0.3in}

  \textit{keywords:} dynamic general equilibrium, taxation, numerical simulation, computational techniques, simulation modeling.

  \vspace{0.3in}

  \textit{JEL classifications:} C63, C68, E62, H24, H25, H68}
  \end{abstract}
  \thispagestyle{empty}
\end{titlepage}

\begin{spacing}{1.5}


\section{Introduction}\label{SecIntro}

\section{Details of the Macro Model}\label{SecMacro}

  We use a model based heavily on \citet{ZodrowDiamond:2013}.

  \subsection{Households}

  \begin{equation}
  LU_t(a,\gamma) = \frac{1}{1-1/\sigma_U} \left[ \sum_{s=t}^{t+T-a-1} \frac{U_s(a,\gamma)^{1-1/\sigma_U}}{(1+\rho)^{s-t}} + \frac{1}{(1+\rho)^{T-a-1}} \alpha_B(\gamma)B_{t+T-a-1}(a,\gamma)^{1-1/\sigma_U} \right]
  \end{equation}
  where $LU_t(a,\gamma)$ is utility for a household of age $a$ and ability level $\gamma$ in period $t$, $U_t(a,\gamma)$ is within-period utility for a household of age $a$ and ability level $\gamma$ in period $t$, $B_{t+T-a-1}(a,\gamma)$ is the bequest left by a houshold of age $a$ and ability level $\gamma$ when it dies in period $t+T-a-1$.  $\sigma_U$ is the intertemporal elasticity of substitution for utility across periods, and $rho$ is the pure rate of time preference.

  Within-period utility depends on consumptions of composite goods $CH$ and leisure $LE$.
  \begin{equation}
  U_s(a,\gamma) = \left[ \alpha_u^{1/\sigma_u} CH_s(a,\gamma)^{1-1/\sigma_u} + (1-\alpha_u)^{1/\sigma_u} LE_s(a,\gamma)^{1-1/\sigma_u}\right]^{\frac{\sigma_u}{\sigma_u-1}}
  \end{equation}

  Composite goods are made up of housing goods $HR$ and non-housing goods $CN$.
  \begin{equation}
  CH_s(a,\gamma) = \left[ \alpha_H^{1/\sigma_H} CN_s(a,\gamma)^{1-1/\sigma_H} + (1-\alpha_H)^{1/\sigma_H} HR_s(a,\gamma)^{1-1/\sigma_H}\right]^{\frac{\sigma_H}{\sigma_H-1}}
  \end{equation}

  Non-housing goods are made up of those produced by the corporate secotor $C$ and non-corporate sector $N$.
  \begin{equation}
  CN_s(a,\gamma) = \left[ \alpha_N^{1/\sigma_N} [C_s(a,\gamma)-b_s^C(a,\gamma)]^{1-1/\sigma_N} + (1-\alpha_N)^{1/\sigma_N} [N_s(a,\gamma)-b_s^N(a,\gamma)]^{1-1/\sigma_N} \right]^{\frac{\sigma_N}{\sigma_N-1}}
  \end{equation}

  Housing goods are made up of owner-occupied housing $H$ and rental housing $R$.
  \begin{equation}
  HR_s(a,\gamma) = \left[ \alpha_R{1/\sigma_R} [H_s(a,\gamma)-b_s^H(a,\gamma)]^{1-1/\sigma_R} + (1-\alpha_R)^{1/\sigma_R} [R_s(a,\gamma)-b_s^R(a,\gamma)]^{1-1/\sigma_R} \right]^{\frac{\sigma_R}{\sigma_R-1}}
  \end{equation}

  \subsection{Firms}

  \subsection{Market Clearing}

\section{Incorporating Feedbacks with Micro Tax Simulations}\label{SecMicro}

\section{Conclusion}\label{SecConclusion}


\end{spacing}


\newpage
\renewcommand{\theequation}{A.\arabic{section}.\arabic{equation}}
                                                 % redefine the command that creates the equation number
\renewcommand{\thedefinition}{A.\arabic{section}.\arabic{definition}}
                                                 % redefine the command that creates the definition number
\renewcommand{\thesection}{A-\arabic{section}}   % redefine the command that creates the section number

\setcounter{equation}{0}                         % reset counter
\setcounter{section}{0}                          % reset section number
\section*{TECHNICAL APPENDICES}

\setcounter{definition}{0}
\setcounter{equation}{0}                         % reset counter




\newpage
\bibliography{AEIBYUDyn}


\end{document}
