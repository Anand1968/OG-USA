\documentclass[letterpaper,12pt]{article}

  \usepackage{threeparttable}
  \usepackage{geometry}
  \geometry{letterpaper,tmargin=1in,bmargin=1in,lmargin=1.25in,rmargin=1.25in}
  \usepackage[format=hang,font=normalsize,labelfont=bf]{caption}
  \usepackage{amsmath}
  \usepackage{multirow}
  \usepackage{array}
  \usepackage{delarray}
  \usepackage{amssymb}
  \usepackage{amsthm}
  \usepackage{lscape}
  \usepackage{natbib}
  \usepackage{setspace}
  \usepackage{float,color}
  \usepackage[pdftex]{graphicx}
  \usepackage{hyperref}
  \usepackage{placeins}
  \hypersetup{colorlinks,linkcolor=red,urlcolor=blue,citecolor=red}
  \theoremstyle{definition}
  \newtheorem{theorem}{Theorem}
  \newtheorem{acknowledgement}[theorem]{Acknowledgement}
  \newtheorem{algorithm}[theorem]{Algorithm}
  \newtheorem{axiom}[theorem]{Axiom}
  \newtheorem{case}[theorem]{Case}
  \newtheorem{claim}[theorem]{Claim}
  \newtheorem{conclusion}[theorem]{Conclusion}
  \newtheorem{condition}[theorem]{Condition}
  \newtheorem{conjecture}[theorem]{Conjecture}
  \newtheorem{corollary}[theorem]{Corollary}
  \newtheorem{criterion}[theorem]{Criterion}
  \newtheorem{definition}{Definition} % Number definitions on their own
  \newtheorem{derivation}{Derivation} % Number derivations on their own
  \newtheorem{example}[theorem]{Example}
  \newtheorem{exercise}[theorem]{Exercise}
  \newtheorem{lemma}[theorem]{Lemma}
  \newtheorem{notation}[theorem]{Notation}
  \newtheorem{problem}[theorem]{Problem}
  \newtheorem{proposition}{Proposition} % Number propositions on their own
  \newtheorem{remark}[theorem]{Remark}
  \newtheorem{solution}[theorem]{Solution}
  \newtheorem{summary}[theorem]{Summary}
  \numberwithin{equation}{section}
  \bibliographystyle{aer}
  \newcommand\ve{\varepsilon}
  \renewcommand\theenumi{\roman{enumi}}
  \providecommand{\norm}[1]{\lVert#1\rVert}

\begin{document}

%titlepage
\begin{titlepage}
  \title{Step-by-Step Derivations of a Model for Dyanmic General Equilibrim Tax Scoring with Micro Tax Simulations
         \thanks{} }

  \author{ Richard W. Evans\footnote{Brigham Young University, Department of Economics, 167 FOB, Provo, Utah 84602, (801) 422-8303, \href{mailto:revans@byu.edu}{revans@byu.edu}.} \\[-2pt]
         \and
         Kerk L. Phillips\footnote{Brigham Young University, Department of Economics, 166 FOB, Provo, Utah 84602, \href{mailto:kerk_phillips@byu.edu}{kerk\_phillips@byu.edu}.} \\[-2pt]}
  \date{June 2014 \\
        \scriptsize{(version 14.06.a)}\\
        \small Preliminary and incomplete; please do not cite.}
  \maketitle
  \begin{abstract}
  \small{This paper ...

  \vspace{0.3in}

  \textit{keywords:} dynamic general equilibrium, taxation, numerical simulation, computational techniques, simulation modeling.

  \vspace{0.3in}

  \textit{JEL classifications:} C63, C68, E62, H24, H25, H68}
  \end{abstract}
  \thispagestyle{empty}
\end{titlepage}

\begin{spacing}{1.5}


\section{Introduction}\label{Sec_Intro}

\section{Details of the Macro Model}\label{Sec_Macro}

  We use a model based initially on that from \citet{EvansPhillips2014} and incorporate many of the features of \citet{ZodrowDiamond:2013} which we refer to hereafter as the DZ model.

  \subsection{Baseline Model - Model 1}\label{SubSec_Macro_Base}
    For our first baseline model we take \citet{EvansPhillips2014} and add a leisure-labor decision, while removing the switching of ability from period to period.  Hence all workers remain the same type throughout their lifetime.  Agents live for $S$ periods and exogenously retire in period $R$.  This is a perfect foresight model.  Both households and firms exist in a unit measure.  All firms are idendical, but households are distinguished by age and ability.

    \subsubsection{Households}
      Housholds maximize utility as given in the equation below.
      \begin{equation}\label{Macro_Base_Utility}
      U_{ist} = \sum_{u=0}^{S-s}\beta^u u(c_{i,s+u,t+u},\ell_{i,s+u,t+u});\text{ where }u(c,\ell) = \frac{c^{1-\gamma}-1}{1-\gamma} - \eta \frac{\ell^{1-\xi}-1}{1-\xi} \nonumber
      \end{equation} 
      $U_{ist}$ is the remaining lifetime utility of a household with ability level $i$ of age $s$ in period $t$.  $c$ denotes consumption of goods and $\ell$ denotes labor supplied to the market.

      The household faces the following set of budget constraints.
      \begin{align}
      w_t \ell_{ist} n_i & \ge c_{ist} + k_{i,s+1,t+1} \text{     for } s=1,\forall i \label{Macro_Base_BC1}\\
      w_t \ell_{ist} n_i + (1+r_t-\delta)k_{ist} & \ge c_{ist} + k_{i,s+1,t+1} \text{     for } 1<s<S,\forall i \label{Macro_Base_BC2}\\
      w_t \ell_{ist} n_i + (1+r_t-\delta)k_{ist} & \ge c_{ist} \text{     for } s=S,\forall i \label{Macro_Base_BC3}
      \end{align}
      $k_{ist}$ is the holdings of capital by household of type $i$ coming due in period $t$ when the household is age $s$. $w$ is the wage rate, $r$ denotes the return on savings, $n$ denotes the effective labor productivity of the houshold.

      The Euler equations from this maximization problem are given below.
      \begin{align}
      c_{ist}^{-\gamma} & = \beta c_{i,s+1,t+1}^{-\gamma}(1+r_{t+1}-\delta) \text{     for } 1\le s<S,\forall i \label{Macro_Base_Euler1}\\
      c_{ist}^{-\gamma} w_t & = \eta \ell_{ist}^{-\xi}, \forall s,i \label{Macro_Base_Euler2}
      \end{align}

    \subsubsection{Firms}
      Firms produce using a Cobb-Douglas production function each period and maximize profits as shown below:
      \begin{equation}
      \Pi_t = K_t^\alpha (e^{gt}L_t)^{1-\alpha} - r_tK_t - w_tL_t \label{Macro_Base_Profit} \nonumber
      \end{equation}

      The profit maximizing conditions are:
      \begin{align}
      r_t & = \alpha K_t^{\alpha-1}(e^{gt}L_t)^{1-\alpha} \label{Macro_Base_FirmFOC1}\\
      w_t & = (1-\alpha) K_t^{\alpha}e^{(1-\alpha)gt}L_t^{-\alpha} \label{Macro_Base_FirmFOC2}
      \end{align}

    \subsubsection{Market Clearing}
      Market-clearing conditions require the following:
      \begin{align}
      K_t & = \sum_{s=2}^S \sum_{i=1}^I \phi_i k_{ist} \label{Macro_Base_Clear1}\\
      L_t & = \sum_{s=1}^S \sum_{i=1}^I \phi_i \ell_{ist} \label{Macro_Base_Clear2}
      \end{align}
      $\phi_i$ is the proportion of type $i$ in the total population of workers.

    \subsubsection{Solution and Simulation}
      This model can be simulated using either the TPI or AMF method described in \citet{EvansPhillips2014}.

      Assuming there are $I$ abilitiy types and $S$ cohorts alive in any period, equations \eqref{Macro_Base_BC1} through \eqref{Macro_Base_Clear2} define a dynamic system of $4+3IS - S$ equations in the variables: $K_t$, $L_t$, $w_t$, $r_t$, $\{c_{ist}\}_{s=1}^S$, $\{\ell_{ist}\}_{s=1}^S$ and $\{k_{ist}\}_{s=2}^S$.

      The parameters of the model are $\alpha$, $\beta$, $\delta$, $\gamma$, $\xi$, $g$, $\{n_i\}$ and $\{\phi_i\}$

    \newpage

  \subsection{Adding Taxes on the Household - Model 2}\label{SubSec_Macro_HHTax}
    \subsubsection{Households}
      The social security payroll tax paid or benefit received is caclualted as follows.
      \begin{equation}\label{Macro_HHTax_PayrollTax}
      T^P_{ist} = \left\{ \begin{matrix} \tau_P w_t \ell_{ist} n_i & \text{if } w_t \ell_{ist} n_i < \chi_P, s<R \\
                                         \tau_P \chi_P & \text{if } w_t \ell_{ist} n_i \ge \chi_P, s<R \\
                                         -\theta w_t n_i & \text{if } s\ge R
                                         \end{matrix} \right.
      \end{equation}
      $\tau_P$ is the payroll tax rate and $\chi_P$ is the payroll tax ceiling.

      Income is $w_t \ell_{ist} n_i + (r_t-\delta) k_{ist}$.  Define $D\{w\ell n+(r-\delta)b,\Omega\}$ as the exemptions and benefits claimed as a function of income and other variables, $\Omega$.  Adjusted gross income is $X_{ist} \equiv w_t \ell_{ist} n_i + (r_t-\delta) k_{ist} - D\{w_t \ell_{ist} n_i + (r_t-\delta) k_{ist},\Omega_{ist}\} -\tau_\delta \delta k_{ist}$.  The final term is a capital depreciation allowance at rate $\tau_\delta$.  We have fit this $D$ function to the data for 2011 using a polynomial function.  Income tax paid is defined as follows.
      \begin{equation}\label{Macro_HHTax_IncomeTax}
      T^I_{ist} = \left\{ \begin{matrix} 0 & \text{if } X_{ist} < \chi_1 \\
                  \tau_1 (X_{ist} - \chi_1) & \text{if }  \chi_1 \le X_{ist} < \chi_2 \\
                  \tau_1 \chi_1 + \tau_2 (X_{ist} - \chi_2) & \text{if }  \chi_2 \le X_{ist} < \chi_3 \\
                  \tau_1 \chi_1 + \tau_2 (\chi_3 - \chi_2) + \tau_3 (X_{ist} - \chi_3) & \text{if }  \chi_3 \le X_{ist} < \chi_4 \\
                  \vdots \\
                  \tau_1 \chi_1 + \tau_2 (\chi_3 - \chi_2) +\dots+ \tau_N (X_{ist} - \chi_N) & \text{if }  \chi_N \le X_{ist} \\
                  \end{matrix} \right.
      \end{equation}
      $\tau_i$ is the marginal tax rate in bracket $i$, the bend points between brackets are denoted $\chi_i$.

      The consumption tax rate is denoted $\tau_c$

      The household faces the following set of budget constraints.
      \begin{align}
      c_{ist} & = (1-\tau_c)\left[w_t \ell_{ist} n_i - k_{i,s+1,t+1} - T^p_{ist} - T^i_{ist} \right] \label{Macro_HHTax_BC1}\\
      c_{ist} & = (1-\tau_c)\left[w_t \ell_{ist} n_i + (1+r_t-\delta) k_{ist} - k_{i,s+1,t+1} - T^p_{ist} - T^i_{ist} \right] \label{Macro_HHTax_BC2}\\
      c_{ist} & = (1-\tau_c)\left[w_t \ell_{ist} n_i + (1+r_t-\delta) k_{ist} - T^p_{ist} - T^i_{ist} \right] \label{Macro_HHTax_BC3}
      \end{align}

    \subsubsection{Government}
      Government collects the following amounts of tax revenue each period.
      \begin{equation} \label{Macro_HHTax_GovRev}
      R_t = \sum_s \sum_i \phi_i \left( T^p_{ist} + T^i_{ist} + \tfrac{\tau_c}{1-\tau_c} c_{ist} \right)
      \end{equation}

    \subsubsection{Solution and Simulation}
      The model now consists of $5 + 5IS - I$ equations with the addtion of \eqref{Macro_HHTax_PayrollTax}, \eqref{Macro_HHTax_IncomeTax} and \eqref{Macro_HHTax_GovRev}, and the substitution of \eqref{Macro_HHTax_BC1} - \eqref{Macro_HHTax_BC3} for \eqref{Macro_Base_BC1} - \eqref{Macro_Base_BC3}.

      The variables are $K_t$, $L_t$, $w_t$, $r_t$, $\{c_{ist}\}_{s=1}^S$, $\{\ell_{ist}\}_{s=1}^S$, $\{k_{ist}\}_{s=2}^S$, $\{T^p_{ist}\}_{s=1}^S$, $\{T^i_{ist}\}_{s=1}^S$ and $R_t$.

      The parameters of the model are $\alpha$, $\beta$, $\delta$, $\gamma$, $\xi$, $g$, $\{n_i\}$, $\{\phi_i\}$, $\tau_p$, $\chi_p$, $\{\tau_n\}_{n=1}^N$, $\{\chi_n\}_{n=1}^N$ and $\tau_c$.

    \newpage

  \subsection{Adding Taxes on Firms - Model 3}\label{SubSec_Macro_CorpTax}
    We allow firms to acquire capital by renting it as above, or by accumulating their own capital and paying dividends, or by issuing bonds.

    We assume both firms and households pay a percent quadratic capital adjustment cost of $\psi(K_{t+1}) = \tfrac{\kappa}{2} \left(K_{t+1}-K_t\right)^2$.  Both housholds and firms receive a depreciation allowance at the rate $\tau_\delta$ and an investment credit at the rate $\tau_{\Delta k}$

    \subsubsection{Households}
      In addtion to capital ($k_{ist}$), households now also hold bonds in the amount $b_{ist}$ and equities in the amount $q_{ist}$.  Interest income and dividends are taxed as regular income, but capital gains are taxed separately ($T^q$).  

      The household's problem can bewritten in the following recursive form:
      \begin{equation}
      \begin{split}
      & V^h_s(k_{ist},b_{ist},q_{ist}) = \\
      & \max_{k_{t+1},b_{t+1},q_{t+1},\ell_t} \frac{c_{ist}^{1-\gamma}-1}{1-\gamma} - \eta \frac{\ell_{ist}^{1-\xi}-1}{1-\xi} + \beta V^h_{s+1}(k_{i,s+1,t+1},b_{i,s+1,t+1},q_{i,s+1,t+1}) \nonumber
      \end{split}
      \end{equation}
    
      The typical household budget constraint is:
      \begin{equation} \label{Macro_CorpTax_BC}
      c_{ist} = (1-\tau_c)\begin{bmatrix} w_t \ell_{ist} n_i + (1+r_t-\delta)k_{ist} + (1+i_t)b_{ist} + p_t q_{ist} \\
        - \tfrac{\kappa}{2}\left(k_{i,s+1,t+1} - k_{ist}\right)^2 - k_{i,s+1,t+1} - b_{i,s+1,t+1} - p_t q_{i,s+1,t+1} \\
        - T^p_{ist} - T^i_{ist} - T^q_{ist} + \tau_\delta \delta k_{ist} + \tau_{\Delta k} (k_{i,s+1,t+1} - k_{ist}) \end{bmatrix} \\
      \end{equation}

      Income subject to taxation ($I_{ist}$) and AGI ($X_{ist}$) are:
      \begin{align}
      I_{ist} & = w_t \ell_{ist} n_i + (r_t-\delta) k_{ist} + i_t b_{ist} + \pi_t q_{ist} \label{Macro_CorpTax_Income} \\
      X_{ist} & = I_{ist} - D\{I_{ist},\Omega_{ist}\} -\tau_\delta \delta k_{ist} \label{Macro_CorpTax_AGI}
      \end{align}

      With the following tax code formulas.
      \begin{align}
      T^P_{ist} & = \left\{ \begin{matrix} \tau_P w_t \ell_{ist} n_i & \text{if } w_t \ell_{ist} n_i < \chi_P, s<R \\
                                         \tau_P \chi_P & \text{if } w_t \ell_{ist} n_i \ge \chi_P, s<R \\
                                         -\theta w_t n_i & \text{if } s\ge R
                                         \end{matrix} \right. \\
      T^I_{ist} & = \left\{ \begin{matrix} 0 & \text{if } X_{ist} < \chi_1 \\
                  \tau_1 (X_{ist} - \chi_1) & \text{if }  \chi_1 \le X_{ist} < \chi_2 \\
                  \tau_1 \chi_1 + \tau_2 (X_{ist} - \chi_2) & \text{if }  \chi_2 \le X_{ist} < \chi_3 \\
                  \tau_1 \chi_1 + \tau_2 (\chi_3 - \chi_2) + \tau_3 (X_{ist} - \chi_3) & \text{if }  \chi_3 \le X_{ist} < \chi_4 \\
                  \vdots \\
                  \tau_1 \chi_1 + \tau_2 (\chi_3 - \chi_2) +\dots+ \tau_N (X_{ist} - \chi_N) & \text{if }  \chi_N \le X_{ist} \\
                  \end{matrix} \right. \\
      T^q_{ist} & = \tau_q \left(\tfrac{p_t}{p_{t-1}}-1 \right) q_{ist}
      \end{align}

      As above, $\tau_i$ is the marginal tax rate in bracket $i$, the bend points between brackets are denoted $\chi_i$, and the consumption tax rate is denoted $\tau_c$

      The first-order conditions from the household's problem are:
      \begin{align}
      \begin{split}
      & c_{ist}^{-\gamma}\left[(1-\tau_c)\left(-1+\tau_\delta \delta + \tau_{\Delta k} - \kappa \left|k_{i,s+1,t+1} - k_{ist} \right|\right)\right] \\
      & + \beta \frac{\partial V^h_{s+1}(k_{i,s+1,t+1},b_{i,s+1,t+1},q_{i,s+1,t+1})}{\partial k_{i,s+1,t+1}} = 0 
      \end{split} \nonumber \\
      & c_{ist}^{-\gamma}[(1-\tau_c)(-1)] + \beta \frac{\partial V^h_{s+1}(k_{i,s+1,t+1},b_{i,s+1,t+1},q_{i,s+1,t+1})}{\partial b_{i,s+1,t+1}} = 0 \nonumber \\
      & c_{ist}^{-\gamma}[(1-\tau_c)(-p_t)] + \beta \frac{\partial V^h_{s+1}(k_{i,s+1,t+1},b_{i,s+1,t+1},q_{i,s+1,t+1})}{\partial q_{i,s+1,t+1}} = 0 \nonumber
      \end{align}

      The envelope conditions are:
      \begin{align}
      \begin{split}
      & \frac{\partial V^h_{s+1}(k_{i,s+1,t+1},b_{i,s+1,t+1},q_{i,s+1,t+1})}{\partial k_{i,s+1,t+1}} \\
      & = c_{ist}^{-\gamma}(1-\tau_c)\left[ 1 + (1-\tau_j)r_t + (\tau_\delta -1)\delta + \tau_{\Delta k} \right]
      \end{split} \nonumber \\  
      \begin{split}
      & \frac{\partial V^h_{s+1}(k_{i,s+1,t+1},b_{i,s+1,t+1},q_{i,s+1,t+1})}{\partial b_{i,s+1,t+1}} \\
      & = c_{ist}^{-\gamma}(1-\tau_c)\left[ 1+(1-\tau_j)i_t \right]
      \end{split} \nonumber \\
      \begin{split}
      &\frac{\partial V^h_{s+1}(k_{i,s+1,t+1},b_{i,s+1,t+1},q_{i,s+1,t+1})}{\partial q_{i,s+1,t+1}} \\
      & = c_{ist}^{-\gamma}(1-\tau_c)\left[ p_t - \tau_q(p_t-p_{t-1} + (1-\tau_j)\pi_t) \right]
      \end{split} \nonumber
      \end{align}

      The Euler equations are:
      \begin{align}
      \begin{split}
      & c_{ist}^{-\gamma}\left[(1-\tau_c)\left(-1+\tau_\delta \delta + \tau_{\Delta k} - \kappa \left|k_{i,s+1,t+1} - k_{ist} \right|\right)\right] \\
      & = \beta c_{i,s+1,t+1}^{-\gamma}(1-\tau_c)\left[ 1 + (1-\tau_j)r_{t+1} + (\tau_\delta -1)\delta + \tau_{\Delta k} \right] \text{     for } E\le s<S,\forall i
      \end{split} \label{Macro_CorpTax_Euler1}\\
      \begin{split}
      & c_{ist}^{-\gamma} = \beta c_{i,s+1,t+1}^{-\gamma} \left[ 1+(1-\tau_j)i_{t+1} \right] \text{     for } 1\le s<S,\forall i 
      \end{split} \label{Macro_CorpTax_Euler2}\\
      \begin{split}
      & c_{ist}^{-\gamma} = \beta c_{i,s+1,t+1}^{-\gamma}\left[ 1 + (1 - \tau_q)(\tfrac{p_{t+1}}{p_t}-1) + (1-\tau_j)\tfrac{\pi_t}{p_t} \right] \text{     for } 1\le s<S,\forall i 
      \end{split} \label{Macro_CorpTax_Euler3}\\
      & c_{ist}^{-\gamma} w_t(1-\tau_j - \tau_p) = \eta \ell_{ist}^{-\xi}, \forall s,i \label{Macro_CorpTax_Euler4}
      \end{align}

    \subsubsection{Firms}
      The firm's intertemporal profits are now:
      \begin{equation} \label{Macro_CorpTax_LifeProf}
      \Pi_t = \sum_{u=0}^\infty d_{ut}\pi_{t+u} \nonumber
      \end{equation}
      where
      \begin{equation}
      d_{ut} \equiv \left\{\begin{matrix} 1 & \text{if } u = 0 \label{Macro_CorpTax_DiscFact}\\
                                \prod_{j=1}^u \tfrac{1}{1+i_{t+u+j}} & \text{otherwise} \end{matrix} \right. \nonumber
      \end{equation}

      We can write its problem as a dynamic program.
      \begin{equation} \label{Macro_CorpTax_FirmDynProg}
      V^f(B_t,H_t) = \max \pi_t + \tfrac{1}{1+i_{t+1}} V^f(B_{t+1},H_{t+1}) \nonumber
      \end{equation}
      Profits are defined as: 
      \begin{equation} \label{Macro_CorpTax_Profit}
      \pi_t = (1-\tau_f)\begin{bmatrix} (K_t+H_t)^\alpha(e^{gt}L_t)^{1-\alpha} - r_t K_t - w_t L_t - (1+i_t) B_t \\
                                + B_{t+1} + (1-\delta) H_t - H_{t+1} - H_{t+1}\psi\left\{\tfrac{H_{t+1}}{H_{t}} \right\} \\
                                + \tau_\delta \delta H_t + \tau_{\Delta k} (H_{t+1} - H_t)\end{bmatrix} 
      \end{equation}
      We also have the following constraint which indicates that new bonds are used to fincance either capital rentals or expansion of the capital stock:
      \begin{equation}
      B_{t+1} = K_t + H_{t+1} - \tfrac{\kappa}{2}(H_{t+1}-H_t)^2 - \tau_{\Delta k} (H_{t+1} - H_t) \nonumber
      \end{equation}

      FOCs with respect to $K_t, L_t$ and $H_{t+1}$ are:
      \begin{align}
      \alpha (K_t+H_t)^{\alpha-1}(e^{gt}L_t)^{1-\alpha} - r_t - 1 +\tfrac{1}{1+i_{t+1}}V^F_B(B_{t+1},H_{t+1}) & = 0 \nonumber \\  
      (1-\alpha) (K_t+H_t)^{-\alpha}e^{(1-\alpha)gt}L_t^{-\alpha} - w_t & = 0 \nonumber \\
      1 -1 - \kappa \left|H_{t+1} - H_{t} \right| - \tau_{\Delta k} +\tfrac{1}{1+i_{t+1}}V^F_H(B_{t+1},H_{t+1}) & = 0 \nonumber  
      \end{align}

      Envelope conditions for $H_t$ and $B_t$ are:
      \begin{align}
      V^F_B(B_{t},H_{t}) & = -(1+t_{t+1}) \nonumber \\
      V^F_B(B_{t},H_{t}) & = (K_t+H_t)^{\alpha-1}(e^{gt}L_t)^{1-\alpha} + 1 - \delta \nonumber 
      \end{align}

      Combining the above gives:
      \begin{align}
      r_t & = (K_t+H_t)^{\alpha-1}(e^{gt}L_t)^{1-\alpha} \label{Macro_CorpTax_FirmCond1}\\
      w_t & = (1-\alpha) (K_t+H_t)^{-\alpha}e^{(1-\alpha)gt}L_t^{-\alpha} \label{Macro_CorpTax_FirmCond2}\\
      1 + r_{t+1} - \delta & = (1+i_{t+1}) \left( \kappa \left|H_{t+1} - H_{t}\right| + \tau_{\Delta k} \right) \label{Macro_CorpTax_FirmCond3}
      \end{align}

    \subsubsection{Market Clearing}
      Market-clearing conditions which determine $r_t, w_t, i_t$ and $p_t$ are:
      \begin{align}
      K_t & = \sum_{s=2}^S \sum_{i=1}^I \phi_i k_{ist} \label{Macro_CorpTax_Clear1}\\
      L_t & = \sum_{s=1}^S \sum_{i=1}^I \phi_i \ell_{ist} \label{Macro_CorpTax_Clear2}\\
      B_t & = \sum_{s=1}^S \sum_{i=1}^I \phi_i b_{ist} \label{Macro_CorpTax_Clear3}\\
      1 & = \sum_{s=1}^S \sum_{i=1}^I \phi_i q_{ist} \label{Macro_CorpTax_Clear4}
      \end{align}  

    \subsubsection{Government}
      Government collects the following amounts of tax revenue each period.
      \begin{equation} \label{Macro_CorpTax_GovRev}
      \begin{split}
      R_t & = \sum_s \sum_i \phi_i \left[ T^p_{ist} + T^i_{ist} + T^q_{ist} + \tfrac{\tau_c}{1-\tau_c} c_{ist}\right] - \tau_\delta (H_{t+1} + K_{t+1} - H_t - K_t) \\
      & - \tau_\delta \delta (H_t + K_t)
      \end{split}
      \end{equation}

    \subsubsection{Solution and Simulation}
      This version of the model has $9+9IS-3I$ equations defined by replacing \eqref{Macro_HHTax_BC1} - \eqref{Macro_HHTax_BC3} with \eqref{Macro_CorpTax_BC}; replacing \eqref{Macro_Base_Euler1} \& \eqref{Macro_Base_Euler2} with \eqref{Macro_CorpTax_Euler1} - \eqref{Macro_CorpTax_Euler4}; adding \eqref{Macro_CorpTax_Income}, \eqref{Macro_CorpTax_AGI}, \eqref{Macro_CorpTax_Profit}; replacing \eqref{Macro_Base_FirmFOC1} \& \eqref{Macro_Base_FirmFOC2} with \eqref{Macro_CorpTax_FirmCond1} - \eqref{Macro_CorpTax_FirmCond3}; and replacing \eqref{Macro_HHTax_GovRev} with \eqref{Macro_CorpTax_GovRev}.  

      The variables are $K_t$, $L_t$, $w_t$, $r_t$, $\{c_{ist}\}_{s=1}^S$, $\{\ell_{ist}\}_{s=1}^S$, $\{k_{ist}\}_{s=2}^S$, $\{T^p_{ist}\}_{s=1}^S$, $\{T^i_{ist}\}_{s=1}^S$, $R_t$, $\{b_{ist}\}_{s=2}^S$, $\{q_{ist}\}_{s=2}^S$, $\{I_{ist}\}_{s=1}^S$, $\{X_{ist}\}_{s=1}^S$, $H_t$, $B_t$ $\pi_t$ and $i_t$.

      The parameters of the model are $\alpha$, $\beta$, $\delta$, $\gamma$, $\xi$, $g$, $\{n_i\}$, $\{\phi_i\}$, $\tau_p$, $\chi_p$, $\{\tau_n\}_{n=1}^N$, $\{\chi_n\}_{n=1}^N$, $\tau_c$, $\tau_q$, $\tau_\delta$, $\tau{\Delta k}$ and $\kappa$.

  \newpage

  \subsection{Adding Demographics - Model 4}\label{SubSec_Macro_Demog}
    Our next step is to add demographic movements to the model.  Rather than keeping the population distribution constant we allow it to evolve naturally over time.

    We denote the number of households of type $i$ age $s$ in period $t$ as $N_{ist}$.  We assume that workers all become economically active at age $E$.  For each age cohort a fraction $\rho_s$ survive from period $t$ into $t+1$.  For each age cohort a fraction $f_s$ of new children are added to next period's age 1 cohort.  The distribution of these children over ability is assumed to be the previously described $\{\phi_i\}_{i=1}^I$  This gives us the following exogenous laws of motion for the population.
    \begin{align}
    N_{i1,t+1} & = \sum_s f_s N_{ist} \phi_i \\
    N_{i,s+1,t+1} & = N_{ist} \rho_s
    \end{align}

    Since workers become economically active at age $E$, rather than at age 1, we need to keep track of $I-E+1$ cohorts' capital stocks, bond holdings and equity holdings.  We assume that people in cohorts 1 throgh $E$ do not work or consume (alternatively their conssumption is included in that of their parents until age $E$).

    We define the total working age population in period $t$ as below.
    \begin{equation}
    N_t = \sum_{s=E}^S \sum_{i=1}^I N_{ist}
    \end{equation}

    This model has unintentional bequests since households will die before age $S$.  We assume their holdings of capital, bonds, and equity shares are distributed lump sum at the begining of the next peroid.  The amounts of each of these lump-sum transfers is shown below.

    \begin{align}
    T^k_t & = \frac{1}{N_t}\sum_{s=E+1}^S \sum_{i=1}^I N_{ist} (1-\rho_i) k_{i,s+1,t+1} \\
    T^b_t & = \frac{1}{N_t}\sum_{s=E+1}^S \sum_{i=1}^I N_{ist} (1-\rho_i) b_{i,s+1,t+1} \\
    T^q_t & = \frac{1}{N_t}\sum_{s=E+1}^S \sum_{i=1}^I N_{ist} (1-\rho_i) q_{i,s+1,t+1}
    \end{align}

    We add these lump sum transfers to the budget constraint and generate \eqref{Macro_Demog_BC}, whcih we substitute in place of \eqref{Macro_CorpTax_BC}
    \begin{equation} \label{Macro_Demog_BC}
    c_{ist} = (1-\tau_c)\begin{bmatrix} w_t \ell_{ist} n_i + (1+r_t-\delta)k_{ist} + (1+i_t)b_{ist} + p_t q_{ist} \\
      - \tfrac{\kappa}{2}\left(k_{i,s+1,t+1} - k_{ist}\right)^2 - k_{i,s+1,t+1} - b_{i,s+1,t+1} - p_t q_{i,s+1,t+1} \\
      +\tau_\delta \delta k_{ist} + \tau_{\Delta k} (k_{i,s+1,t+1} - k_{ist}) \\
      - T^p_{ist} - T^i_{ist} - T^q_{ist} + T^k_t + T^b_t + p_tT^q_t\end{bmatrix} \\
    \end{equation}

    We also add additional discounting to the household's problem and rewrite \eqref{Macro_CorpTax_Euler1} - \eqref{Macro_CorpTax_Euler3} as below:
    \begin{align}
    \begin{split}
    & c_{ist}^{-\gamma}\left[(1-\tau_c)\left(-1+\tau_\delta \delta + \tau_{\Delta k} - \kappa \left|k_{i,s+1,t+1} - k_{ist} \right|\right)\right] \\
    & = \beta \rho_s c_{i,s+1,t+1}^{-\gamma}(1-\tau_c)\left[ 1 + (1-\tau_j)r_{t+1} + (\tau_\delta -1)\delta + \tau_{\Delta k} \right] \text{     for } E\le s<S,\forall i
    \end{split} \label{Macro_Demog_Euler1}\\
    \begin{split}
    & c_{ist}^{-\gamma} = \beta \rho_s c_{i,s+1,t+1}^{-\gamma} \left[ 1+(1-\tau_j)i_{t+1} \right] \text{     for } E\le s<S,\forall i 
    \end{split} \label{Macro_Demog_Euler2}\\
    \begin{split}
    & c_{ist}^{-\gamma} = \beta \rho_s c_{i,s+1,t+1}^{-\gamma}\left[ 1 + (1 - \tau_q)(\tfrac{p_{t+1}}{p_t}-1) + (1-\tau_j)\tfrac{\pi_t}{p_t} \right] \text{     for } E\le s<S,\forall i 
    \end{split} \label{Macro_Demog_Euler3}
    \end{align}

    Market-clearing conditions \eqref{Macro_CorpTax_Clear1} - \eqref{Macro_CorpTax_Clear2} are replaced with the following.
    \begin{align}
    K_t & = \sum_{s=E+1}^S \sum_{i=1}^I N_{ist} k_{ist} \label{Macro_Demog_Clear1}\\
    L_t & = \sum_{s=E}^S \sum_{i=1}^I N_{ist} \ell_{ist} \label{Macro_Demog_Clear2}\\
    B_t & = \sum_{s=E+1}^S \sum_{i=1}^I N_{ist} b_{ist} \label{Macro_Demog_Clear3}\\
    1 & = \sum_{s=E+1}^S \sum_{i=1}^I N_{ist} q_{ist} \label{Macro_Demog_Clear4}
    \end{align}  

    We have added $4 + IS$ new variables to the model: $T^k_t$, $T^q_t$, $T^q_t$, $N_t$ and $\{N_{ist}\}_{s=1}^S$.

    We have also eliminated $9(E-1)$ variables: $\{c_{ist}\}_{s=E-1}^S$, $\{\ell_{ist}\}_{s=E-1}^S$, $\{k_{ist}\}_{s=E}^S$, $\{T^p_{ist}\}_{s=E-1}^S$, $\{T^i_{ist}\}_{s=E-1}^S$, $\{b_{ist}\}_{s=E}^S$, $\{q_{ist}\}_{s=E}^S$, $\{I_{ist}\}_{s=E-1}^S$ and $\{X_{ist}\}_{s=E-1}^S$.

    This gives a total of $13 + 9(S-E+1)I - 3I + IS$ varaibles.

    We have also added the following parameters: $\{\rho_s\}_{s=1}^S$ and $\{f_s\}_{s=E}^S$

    *** TO DO ***
    \begin{itemize}
    \item Add immigration - average into the cohort/type bin
    \item Let mortality and fertility vary by ability type as well as age.
    \item Allow for trends in these parameters
    \end{itemize}

  \newpage


  \subsection{Adding Open Economy Interest Rate Responsiveness - Model 5}\label{SubSec_Macro_Open}

  \subsection{Adding Mulitiple Industries - Model 6}\label{SubSec_Macro_Multi}

  \subsection{Adding Ability Switching - Model 7}\label{SubSec_Macro_Switch}

\section{Conclusion}\label{SecConclusion}


\end{spacing}


\newpage
\renewcommand{\theequation}{A.\arabic{section}.\arabic{equation}}
                                                 % redefine the command that creates the equation number
\renewcommand{\thedefinition}{A.\arabic{section}.\arabic{definition}}
                                                 % redefine the command that creates the definition number
\renewcommand{\thesection}{A-\arabic{section}}   % redefine the command that creates the section number

\setcounter{equation}{0}                         % reset counter
\setcounter{section}{0}                          % reset section number
\section*{TECHNICAL APPENDIX}

\setcounter{definition}{0}
\setcounter{equation}{0}                         % reset counter




\newpage
\bibliography{AEIBYUDyn}


\end{document}
