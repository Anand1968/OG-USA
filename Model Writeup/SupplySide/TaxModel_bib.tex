@InBook{DZ2013,
  author={Zodrow, George R. and Diamond, John W.},
  title={{Dynamic Overlapping Generations Computable General Equilibrium Models and the Analysis of Tax Policy: The Diamond�Zodrow Model}},
  publisher={Elsevier},
  year=2013,
  month={December},
  volume={1},
  number={},
  series={Handbook of Computable General Equilibrium Modeling},
  edition={},
  chapter={0},
  pages={743-813},
  keywords={Computable general equilibrium; overlapping generations; dynamic modeling; tax reform simulation; in},
  abstract={We examine the use of dynamic overlapping generations (OLG) computable general equilibrium (CGE) models to analyze the economic effects of tax reforms, using as a paradigm our Diamond�Zodrow (DZ) model. Such models are especially well-suited to analyzing both the short-run transitional and the long-run dynamic macroeconomic effects of tax reforms, including the time paths of reform-induced changes in labor supply, saving, and investment, as well as the redistributional effects of reforms across and within generations. We begin with a brief overview of the use of OLG-CGE models in the analysis of tax reform, focusing on the seminal contribution of Auerbach and Kotlikoff (1987). We then consider a variety of extensions of this work, including the multiple-good, multiple-individual model constructed by Fullerton and Rogers (1993), as well as the addition of open economy factors, human capital accumulation and uncertainty. Many of the applications of these models have focused on changes in capital income taxation or, more generally, the replacement of an income tax system that fully taxes capital income with a consumption-based tax system that exempts normal returns to capital, and we focus on such reforms. We describe in considerable detail the DZ model, which is characterized by 55 cohorts, 12 income groups within each cohort, four production sectors and explicit calculation of reform-induced changes in asset values. We conclude by describing numerous applications of the DZ model, ranging from incremental reforms of the income tax system including deficit-financed tax cuts to �fundamental tax reforms� that involve replacing the income tax with a consumption-based tax system, to the implementation of a value-added tax imposed in addition to the income tax as a means of reducing current deficits and the national debt in the US.},
  url={http://ideas.repec.org/h/eee/hacchp/v1y2013icp743-813.html}
}


@Article{BM1993,
  author={Ballard, Charles L. and Medema, Steven G.},
  title={{The marginal efficiency effects of taxes and subsidies in the presence of externalities : A computational general equilibrium approach}},
  journal={Journal of Public Economics},
  year=1993,
  volume={52},
  number={2},
  pages={199-216},
  month={September},
  keywords={},
  abstract={No abstract is available for this item.},
  url={http://ideas.repec.org/a/eee/pubeco/v52y1993i2p199-216.html}
}



@Article{Hayashi1982,
  author={Hayashi, Fumio},
  title={{Tobin's Marginal q and Average q: A Neoclassical Interpretation}},
  journal={Econometrica},
  year=1982,
  volume={50},
  number={1},
  pages={213-24},
  month={January},
  keywords={},
  abstract={No abstract is available for this item.},
  url={http://ideas.repec.org/a/ecm/emetrp/v50y1982i1p213-24.html}
}


@BOOK{ FG1993,
AUTHOR    = {Don Fullerton and Diane Lim Rogers}, 
TITLE     = {Who Bears the Lifetime Tax Burden?},
PUBLISHER = {The Brookings Institution},
YEAR      = {1993},
Volume    = {},
Series    = {},
Address   = {},
Edition   = {},
Month     = {},
Note      = {},
Key       = {}
}


