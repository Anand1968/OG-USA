\chapter{Households}
\index{Households%
@\emph{Households}}%

  \section{Demographics}
    A measure $\omega_{1,t}$ of individuals with heterogeneous working ability $e \in\mathcal{E}\subset\mathbb{R}_{++}$ is born in each period $t$ and live for $E+S$ periods, with $S\geq 4$.\footnote{Theoretically, the model exposition of the model works without loss of generality for $S\geq 3$. However, because we are calibrating the ages outside of the economy to be one-fourth of $S$ (e.g., ages 21 to 100 in the economy, and ages 1 to 20 outside of the economy), we need $S$ to be at least 4.} The population of age-$s$ individuals in any period $t$ is $\omega_{s,t}$. Households are termed ``youth'' and out of the market during ages $1\leq s\leq E$. The households enter the workforce and economy in period $E+1$ and remain in the workforce until they unexpectedly die or live until age $s=E+S$.\footnote{We model the population with households age $s\leq E$ outside of the workforce and economy in order most closely match the empirical population dynamics.} The population of agents of each age in each period $\omega_{s,t}$ evolves according to the following function,
    \begin{equation}\label{EqPopLawofmotion}
      \begin{split}
        \omega_{1,t+1} &= \sum_{s=1}^{E+S} f_s\omega_{s,t}\quad\forall t \\
        \omega_{s+1,t+1} &= (1 + i_s - \rho_s)\omega_{s,t}\quad\forall t\quad\text{and}\quad 1\leq s \leq E+S-1
      \end{split}
    \end{equation}
    where $f_s\geq 0$ is an age-specific fertility rate, $i_s$ is an age-specific immigration rate, $\rho_s$ is an age specific mortality hazard rate,\footnote{The parameter $\rho_s$ is the probability that a household of age $s$ dies before age $s+1$.} and $1+i_s-\rho_s$ is constrained to be nonnegative. The total population in the economy $N_t$ at any period is simply the sum of individuals in the economy, the population growth rate in any period $t$ from the previous period $t-1$ is $g_{n,t}$, $\tilde{N}_t$ is the working age population, and $\tilde{g}_{n,t}$ is the working age population growth rate in any period $t$ from the previous period $t-1$.
    \begin{equation}\label{EqPopDef}
      N_t\equiv\sum_{s=1}^{E+S} \omega_{s,t} \quad\forall t
    \end{equation}
    \begin{equation}\label{EqPopGrowth}
      g_{n,t+1} \equiv \frac{N_{t+1}}{N_t} - 1 \quad\forall t
    \end{equation}
    \begin{equation}\label{EqPopWkDef}
      \tilde{N}_t\equiv\sum_{s=E+1}^{E+S} \omega_{s,t} \quad\forall t
    \end{equation}
    \begin{equation}\label{EqPopWkGrowth}
      \tilde{g}_{n,t+1} \equiv \frac{\tilde{N}_{t+1}}{\tilde{N}_t} - 1 \quad\forall t
    \end{equation}


  \section{Households}
    The consumer's maximization problem is:
    \begin{equation}\label{EqUtilMax}
      \begin{split}
        &U_{j,s,t} = \sum_{u=0}^{E+S-s}\beta^u\left[\prod_{v=s-1}^{s+u-1}(1-\rho_v)\right] u\left(c_{j,s+u,t+u},n_{j,s+u,t+u},b_{j,s+u+1,t+u+1}\right) \\
        &\text{where}\quad \rho_{s-1}=0 \\
        &\text{and} \quad u\left(c_{j,s,t},n_{j,s,t},b_{j,s+1,t+1}\right) = \frac{\left(c_{j,s,t}\right)^{1-\sigma} - 1}{1-\sigma} ... \\
        &\qquad\qquad + e^{g_y t(1-\sigma)}\chi^n_s\left(b\left[1 - \left(\frac{n_{j,s,t}}{\tilde{l}}\right)^\upsilon\right]^\frac{1}{\upsilon} + k\right) + \rho_s\chi^b\frac{\left(b_{j,s+1,t+1}\right)^{1-\sigma} - 1}{1-\sigma} \\
        &\quad\quad\quad\quad\quad\quad\quad\quad\quad\quad\quad\quad\quad\quad\quad\quad\quad\quad\quad\forall j,t\quad\text{and}\:E+1\leq s\leq E+S
      \end{split}
    \end{equation}

    Households choose consumption of the composite consumption good, $c_{j,s+u,t+u}$, labor supply, $n_{j,s+u,t+u},$ and asset holdings, $b_{j,s+u+1,t+u+1}$ to maximize the expected, discounted, lifetime utility subject to their per-period budget constraint.  Total consumption of the composite good is made up of discretionary consumption, $\tilde{c}_{j,s,t}$ and minimum required purchases of each consumption good, $\bar{c}_{i,s}$.  Thus the consumer's choice is over $\tilde{c}_{j,s,t}$, which together with the minimum required purchases equal determine total composite consumption: $c_{j,s,t}=\tilde{c}_{j,s,t}+\sum_{i=1}^{I}c_{i,s}$.  It is therefore the case that there minimum required purchases affect the household's ability to smooth consumption over time.  We discuss the composite consumption good in more detail in Section \ref{sec:subutil}.  This composite good is age dependent, thus each the price of the composite consumption good varies with age $s$.  We denote the gross-of-tax price of the composite consumption good for households of age $s$ in period $t$ as $p_{s,t}$ and the gross-of-tax price for good $i$ at time $t$ as $p_{i,t}$. Thus the households' per period budget constraint can be given by the following:
    
    \begin{equation}\label{EqBC}
      \begin{split}
        \sum_{i=1}^{I} p_{i,t}c_{i,j,s,t} + p_{s,t}\tilde{c}_{s,t} + b_{j,s+1,t+1} \leq \left(1 + r_t\right) b_{j,s,t} + w_t e_{j,s}&n_{j,s,t} + \frac{BQ_{j,t}}{\lambda_j\tilde{N}_t} - T_{j,s,t} \\
        \quad\text{where}\quad b_{j,s,1} = 0 \\
        &\text{for} \quad E+1\leq s \leq E+S \quad \forall j,t
      \end{split}
    \end{equation}

    We set up a Lagrangian and solve by taking derivatives with respect to $\{c_{j,s,t},n_{j,s,t+u},b_{j,s,t+1}\}$ for all $j,s$ and $t$.

      The necessary condition with respect to the discretionary consumption of the composite consumption good $\tilde{c}_{j,s,t}$ is:
    \begin{equation}\label{Eqcfoc}
      \begin{split}
      \frac{\partial U}{\partial \tilde{c}_{j,s+u,t+u}}  = \beta^u\left[\prod_{v=s-1}^{s+u-1}(1-\rho_v)\right] c_{j,s+u,t+u}^{-\sigma} - \beta^u\left[\prod_{v=s-1}^{s+u-1}(1-\rho_v)\right]  \lambda_{t+u} p_{s+u,t+u} = 0
        \end{split}
    \end{equation}

    With respect to labor:
    \begin{equation}\label{Eqnfoc}
      \begin{split}
      \frac{\partial U}{\partial n_{j,s+u,t+u}} & = \beta^u\left[\prod_{v=s-1}^{s+u-1}(1-\rho_v)\right] e^{g_y (t+u)(1-\sigma)}\chi^n_{s}\biggl(\frac{b}{\tilde{l}}\biggr)\biggl(\frac{n_{j,s+u,t+u}}{\tilde{l}}\biggr)^{v-1}\Biggl[1 - \biggl(\frac{n_{j,s+u,t+u}}{\tilde{l}}\biggr)\Biggr]^{\frac{1-v}{v}} \\
      & -  \beta^u\left[\prod_{v=s-1}^{s+u-1}(1-\rho_v)\right]\lambda_{t+u} \left( w_{+u} e_{j,s+u} - \frac{\partial T_{j,s+u,t+u}}{\partial n_{j,s+u,t+u}} \right)= 0
        \end{split}
    \end{equation}

    With respect to savings:
    \begin{equation}\label{Eqbfoc}
      \begin{split}
      \frac{\partial U}{\partial b_{j,s+u+1,t+u+1}} & = \beta^u\left[\prod_{v=s-1}^{s+u-1}(1-\rho_v)\right] \rho_s\chi^b\bigl(b_{j,s+u+1,t+u+1}\bigr)^{-\sigma} - \beta^u\left[\prod_{v=s-1}^{s+u-1}(1-\rho_v)\right] \lambda_{t+u}  \\\ & - \beta^{u+1}\left[\prod_{v=s-1}^{s+u}(1-\rho_v)\right] \lambda_{t+u+1} \left( 1 + r_{t+u+1} - \frac{\partial T_{j,s+u+1,t+u+1}}{\partial b_{j,s+U+1,t+u+1}} \right)= 0
      \end{split}
    \end{equation}

    We can solve each of these for $\lambda_{t+u}$ to get the following.
    \begin{equation}
      \begin{split}
      \lambda_{t+u} = \frac{c_{j,s+u,t+u}^{-\sigma}}{ p_{s,t+u}} \nonumber
      \end{split}
    \end{equation}

    \begin{equation}
      \begin{split}
      \lambda_{t+u} = \frac{e^{g_y (t+u)(1-\sigma)}\chi^n_{s}\biggl(\frac{b}{\tilde{l}}\biggr)\biggl(\frac{n_{j,s+u,t+u}}{\tilde{l}}\biggr)^{v-1}\Biggl[1 - \biggl(\frac{n_{j,s+u,t+u}}{\tilde{l}}\biggr)\Biggr]^{\frac{1-v}{v}}}{ w_{+u} e_{j,s+u} - \frac{\partial T_{j,s+u,t+u}}{\partial n_{j,s+u,t+u}} }  \nonumber
      \end{split}
    \end{equation}

    \begin{equation}
      \begin{split}
      \lambda_{t+u} = \rho_s\chi^b\bigl(b_{j,s+u+1,t+u+1}\bigr)^{-\sigma} - \beta (1-\rho_{s+u}) \lambda_{t+u+1} \left( 1 + r_{t+u+1} - \frac{\partial T_{j,s+u+1,t+u+1}}{\partial b_{j,s+u+1,t+u+1}} \right)
        \end{split}  \nonumber
    \end{equation}

    These then reduce to the following 2 Euler equations for each $j,s,$ and $t$:

    Marginal utility of consumption of the composite good $i$ is compared to the marginal utility of labor:
    \begin{equation}\label{EqcEuler}
      \begin{split}
      & \frac{ c_{j,s+u,t+u}^{-\sigma}}{p_{s+u,t+u}} \\
      & = \frac{ e^{g_y (t+u)(1-\sigma)}\chi^n_{s}\biggl(\frac{b}{\tilde{l}}\biggr)\biggl(\frac{n_{j,s+u,t+u}}{\tilde{l}}\biggr)^{v-1}\Biggl[1 - \biggl(\frac{n_{j,s+u,t+u}}{\tilde{l}}\biggr)\Biggr]^{\frac{1-v}{v}} } { w_{t+u} e_{j,s+u} - \frac{\partial T_{j,s+u,t+u}}{\partial n_{j,s+u,t+u}} }
       \end{split}
    \end{equation}

    Intertemporal Euler equation for savings, including the utility effects of bequests:
    \begin{equation}\label{EqbEuler}
      \begin{split}
      & \frac{ c_{j,s+u,t+u}^{-\sigma}}{p_{s+u,t+u}} = \rho_s\chi^b\bigl(b_{j,s+U+1,t+U+1}\bigr)^{-\sigma}  - \frac{ \beta(1-\rho_{s+u}) c_{j,s+u+1,t+u+1}^{-\sigma}} { p_{s+u+1,t+u+1}} \times \left( 1 + r_{t+u+1} - \frac{\partial T_{j,s+U+1,t+U+1}}{\partial b_{j,s+U+1,t+u+1}} \right)
      \end{split}
    \end{equation}

       
    \subsection{Household's Subutility Function}\label{sec:subutil}
    
    Household preferences over the composite consumption good are modeled as a Stone-Geary function. The aggregate discretionary consumption of the composite good is defined as follows.
    \begin{equation} \label{Eqcagg}
        \tilde{c}_{j,s,t}  = \prod_{i=1}^I \left( c_{i,j,s,t} - \bar c_{i,s} \right) ^{\alpha_{i,s}} 
    \end{equation}

Where, $c_{i,j,s,t}$ are consumption of good $i$ by household of type $j$, age $s$, at time $t$.  There are $I$ total goods and $\bar{c}_{i,s}$ represents the minimum consumption amount for each good at each age.  The parameters $\alpha_{i,s}$ are the share parameters (and $\sum_{i=1}^{I} \alpha_{i,s}=1$).  They correspond to the share of income, after minimum expenditure amounts, that are spent on each good at each age.  Allowing the minimum consumption amounts and the share parameters to vary by age helps to incorporate life-cyle profiles of consumption into the model.  For example, we do not explicitly model household formation decisions, but they will be some of the effects of changes in household composition over the life-cycle are obtained through the parameters of the Stone-Geary function.  For example, the minimum required expenditure on shelter may be higher in the middle of the life-cycle when household size is larger.  The minimum consumption amounts also mean that the composition of consumption will vary with income, even though all households have the same utility function.

The consumer chooses $c_{i,j,s,t}$ to maximize Equation \ref{Eqcagg} subject to the budget constraint:

    \begin{equation} \label{eqn:cons_budgetcons}
        \sum_{i=1}^{I} p_{i,t}(c_{i,j,s,t}-\bar{c}_{i,s})  = p_{s,t}\tilde{c}_{j,s,t}
    \end{equation}

\noindent where $p_{i,t}$ is the gross of tax price of good $i$ at time $t$ and $p_{s,t}$ is the gross of tax price of the the discretionary component of the composite consumption good consumed by those of age $s$ at time $t$.  Maximization of \ref{Eqcagg} subject to \ref{eqn:cons_budgetcons} yields:

    \begin{equation} \label{eqn:cons_lagrangian}
       \mathcal{L} =  \max_{\{c_{i,j,s,t}\}_{i=1}^{I}}  \prod_{i=1}^I \left( c_{i,j,s,t} - \bar c_{i,s} \right) ^{\alpha_{i,s}}  + \lambda \left(p_{s,t}\tilde{c}_{j,s,t} - \sum_{i=1}^{I} p_{i,t}(c_{i,j,s,t}-\bar{c}_{i,j,s,t})\right)
    \end{equation}
    
    Which as $I$ FOCs (for each $j$, $s$, $t$):
    
      \begin{equation} \label{eqn:cons_FOC}
      \begin{split}
       & \frac{\partial \mathcal{L}}{\partial c_{i,j,s,t}} = \frac{\alpha_{i,s} \prod_{i=1}^I \left( c_{i,j,s,t} - \bar c_{i,s} \right) ^{\alpha_{i,s}}}{(c_{i,j,s,t}-\bar{c}_{i,s})}-\lambda p_{i,t} = 0, \forall \ i  \\
       & \implies  \frac{\alpha_{i,s} \prod_{i=1}^I \left( c_{i,j,s,t} - \bar c_{i,s} \right) ^{\alpha_{i,s}}}{(c_{i,j,s,t}-\bar{c}_{i,s})} = \lambda p_{i,t}, \forall \ i \\
       & \implies  \frac{\alpha_{i,s} \prod_{i=1}^I \left( c_{i,j,s,t} - \bar c_{i,s} \right) ^{\alpha_{i,s}}}{ p_{i,t}(c_{i,j,s,t}-\bar{c}_{i,s})} = \lambda, \forall \ i \\
       & \implies \frac{\alpha_{i,s}}{p_{i,t}(c_{i,j,s,t}-\bar{c}_{i,s})}=\frac{\alpha_{j,s}}{p_{k,t}(c_{k,j,s,t}-\bar{c}_{k,s})}, \forall \ i,k \\
       & \implies c_{i,j,s,t}= \frac{\alpha_{i,s} p_{k,t}(c_{k,j,s,t}-\bar{c}_{k,s})}{\alpha_{k,s} p_{i,t}} + \bar{c}_{i,s}
       \end{split}
    \end{equation}
    
    Now substitute the last line of \ref{eqn:cons_FOC} into the budget constraint (Equation \ref{eqn:cons_budgetcons}):
    
          \begin{equation} \label{eqn:cons_solve}
      \begin{split}
       & p_{s,t}\tilde{c}_{j,s,t} = \sum_{i=1}^{I}p_{i,t}(c_{i,j,s,t}-\bar{c}_{i,s}) \\
       & \implies  p_{s,t}\tilde{c}_{j,s,t} = \sum_{i=1}^{I}p_{i,t}\left[ \frac{\alpha_{i,s} p_{k,t}(c_{k,j,s,t}-\bar{c}_{k,s})}{\alpha_{k,s} p_{i,t}} + \bar{c}_{i,s}- \bar{c}_{i,s}\right] \\
       & \implies  p_{s,t}\tilde{c}_{j,s,t} = \sum_{i=1}^{I}\left[ \frac{\alpha_{i,s} p_{k,t}(c_{k,s}-\bar{c}_{k,s})}{\alpha_{k,s}}\right] \\
       & \implies  p_{s,t}\tilde{c}_{j,s,t} = \frac{ p_{k,t}(c_{k,j,s,t}-\bar{c}_{k,s})}{\alpha_{k,s}} \underbrace{\sum_{i=1}^{I}\alpha_{i,s}}_{=1} \\	
        & \implies  p_{s,t}\tilde{c}_{j,s,t} = \frac{ p_{k,t}(c_{k,j,s,t}-\bar{c}_{k,s})}{\alpha_{k,s}} \\
        & \implies  \frac{ p_{k,t}(c_{k,j,s,t}-\bar{c}_{k,s})}{\alpha_{k,s}}  = p_{s,t}\tilde{c}_{j,s,t}   \\	
        & \implies  c_{k,j,s,t}  = \frac{\alpha_{k,s} p_{s,t}\tilde{c}_{j,s,t}}{p_{k,t}} + \bar{c}_{k,s},  \forall \ k  \\	
       \end{split}
    \end{equation}
    
    Thus, consumption of each good $i$, $c_{i,j,s,t}$ is given by the the amount of minimum consumption plus the share of total expenditures remaining after making the minimum expenditures on all goods (this is called the ``supernumerary" expenditure).  To solve the model for the amount of each consumption good we use the amount of composite consumption for the household from the solution to the intertemporal utility maximization problem in \ref{EqUtilMax}, which gives us $\tilde{c}_{j,s,t}$. We then take the $p_{i,t}$ (which is derived from the solution to the firms' problems, the zero profit condition on firms, and the the bridge between production and consumption goods) and the estimated parameters $\{\alpha_{i,s}\}$ and $\{bar{c}_{i,s}\}$ (we discuss calibration of consumption parameters in Chapter \ref{}).  Finally, we derive the prices of the age $s$ composite consumption good in period $t$, $p_{s,t}$ as: 
    
              \begin{equation} \label{eqn:composite_price}
      \begin{split}
      & \tilde{c}_{j,s,t} = \prod_{i=1}^{I}(c_{i,j,s,t}-\bar{c}_{i,s})^{\alpha_{i,s}} \\
      &\implies \tilde{c}_{j,s,t} = \prod_{i=1}^{I}\left( \frac{\alpha_{i,s} p_{s,t}\tilde{c}_{j,s,t}}{p_{i,t}} + \bar{c}_{i,s}-\bar{c}_{i,s}\right)^{\alpha_{i,s}} \\
      &\implies \tilde{c}_{j,s,t} = \prod_{i=1}^{I} \left( \frac{\alpha_{i,s} p_{s,t}\tilde{c}_{j,s,t}}{p_{i,t}} \right)^{\alpha_{i,s}} \\
      &\implies \tilde{c}_{j,s,t} =  p_{s,t}\tilde{c}_{j,s,t} \prod_{i=1}^{I}\left( \frac{\alpha_{i,s}}{p_{i,t}} \right)^{\alpha_{i,s}} \\
      &\implies \frac{p_{s,t}\tilde{c}_{j,s,t}}{\tilde{c}_{j,s,t}} =  \prod_{i=1}^{I}\left( \frac{p_{i,t}}{\alpha_{i,s}} \right)^{\alpha_{i,s}} \\
       &\implies p_{s,t} =  \prod_{i=1}^{I}\left( \frac{p_{i,t}}{\alpha_{i,s}} \right)^{\alpha_{i,s}} \\
       \end{split}
    \end{equation}
    
    With these parameters and endogenous variables, we then use \ref{eqn:cons_solve} to find the $c_{i,j,s,t}$.
    
    \subsection{Relating Consumption and Production Goods}\label{sec:prod_cons_map}
    
    Our model contains $I$ consumption goods and $M$ production goods.  We denote the quantity of production good $m$ in period $t$ as $X_{m,t}$.  We related the output of the production sectors and the consumption goods using a fixed coefficient model. That is, each consumption good is made up of a mix of the outputs of different production sectors.  The weights that determine the mix for each consumption goods are given in the matrix $Z$.  Element $z_{i,m}$ of the matrix $Z$ corresponds to the percentage contribute of the output of sector $m$ in the production of good $i$.  The total supply of good $i$ in the economy at time $t$ is thus given by: 
    
             \begin{equation} \label{eqn:mix_cons}
             c_{i,t} = \sum_{m=1}^{M}z_{i,m}X_{m,t} 
    	\end{equation}
	
	And thus the price of a unit of consumption good $i$ at time $t$ is:
	
             \begin{equation} \label{eqn:mix_cons_price}
             p_{i,t} = \sum_{m=1}^{M}z_{i,m}p_{m,t}, 
    	\end{equation}
    
    Where $p_{m}$ is the price of output of production sector $m$ at time $t$.
    
    \subsection{Preferences for Corporate vs. Noncorporate Goods}\label{sec:pref_corp_noncorp}
    
    Production sectors may contain corporate and non-corporate producers, each facing different tax treatment.  If the output from corporate and non-corporate entities are perfect substitutes, then if the producers have the same production technology, consumers will end up consuming only the output from the sector with lowest after tax cost of producing.  \citet{GK1989} propose a model where different production sectors face different technologies, which can give rise to an equilibrium where both the corporate and non-corporate sector produce the same good.  We take a different track, following \citet{FR1993} we allow production technologies to vary across industry, but not across sectors within industry.  To have both sectors produce output in equilibrium, we proposed that output across sectors are not perfect substitutes.  For example, food outside the home from a corporate, chain restaurant chain is not the same as food outside the home from a small, family-owned restaurant.  Specifically, we define consumer preferences such that demand for the composite production good (combing output from the corporate and non-corporate sector) for production sector $m$ at time $t$, $X_{m,t}$, is a constant elasticity of substitution (CES) function of the output from the corporate and non-corporate sectors, $X_{m,t,C}$ and $X_{m,t,NC}$, respectively:
    
                  \begin{equation} \label{eqn:comp_output}
             X_{m,t} = \left[\gamma_{m}^{\frac{1}{\ve_{3}}}X_{m,t,C}^{\frac{(\ve_{3}-1)}{\ve_{3}}}+(1-\gamma_{m})^{\frac{1}{\ve_{3}}}X_{m,t,NC}^{\frac{(\ve_{3}-1)}{\ve_{3}}}+\right]^{\frac{\ve_{3}}{(\ve_{3}-1)}}, 
    	\end{equation}
	
	\noindent where $\ve_{3}$ is the elasticity of substitution between corporate and non-corporate output and is assumed to be constant across industries.  The share parameter in the CES function, $\gamma_{m}$ is allowed to vary across industry and will be identified by the fraction of corporate produced output across industries.  The CES function thus explains the existence of corporate and non-corporate production within each industry as well as the different shares out corporate output across industries.  Because of these preferences, changes in corporate and non-corproate tax treatment will have differential impacts across consumers of different ages and income levels.  
	Consumers choose $X_{m,t,C}$ and $X_{m,t,NC}$ to maximize \ref{eqn:comp_output} subject to:
	
	 \begin{equation} \label{eqn:comp_output_cons}
             p_{m,t}X_{m,t} = p_{m,t,C}X_{m,t,C}+p_{m,t,NC}X_{m,t,NC}, 
    	\end{equation}
	
	
\noindent where $p_{m,t,C}$ and $p_{m,t,NC}$ are the prices of output from the corporate and non-corporate firms in production industry $m$, respectively.  Note that these prices are determined through the firm's profit maximization problem and the zero economic profit condition for firms. The constrained optimization problem consumers face is: 
    
 \begin{equation} \label{eqn:comp_output_lagrangian}
	\begin{split}
	 \mathcal{L} = \left[\gamma_{m}^{\frac{1}{\ve_{3}}}X_{m,t,C}^{\frac{(\ve_{3}-1)}{\ve_{3}}}+(1-\gamma_{m})^{\frac{1}{\ve_{3}}}X_{m,t,NC}^{\frac{(\ve_{3}-1)}{\ve_{3}}}+\right]^{\frac{\ve_{3}}{(\ve_{3}-1)}} + \lambda\left(p_{m,t}X_{m,t} - p_{m,t,C}X_{m,t,C}+p_{m,t,NC}X_{m,t,NC}\right)
  	\end{split}
\end{equation}
    
    FOCs are:
    
\begin{equation} \label{eqn:comp_output_foc_C}
	\begin{split}
       	&  \frac{\partial \mathcal{L}}{\partial X_{m,t,C}} = \gamma_{m}^{\frac{1}{\ve_{3}}} X_{m,t,C}^{\frac{-1}{\ve_{3}}} \left[\gamma_{m}^{\frac{1}{\ve_{3}}}X_{m,t,C}^{\frac{(\ve_{3}-1)}{\ve_{3}}}+(1-\gamma_{m})^{\frac{1}{\ve_{3}}}X_{m,t,NC}^{\frac{(\ve_{3}-1)}{\ve_{3}}}+\right]^{\frac{1}{(\ve_{3}-1)}} - \lambda p_{m,t,C} = 0
      	 \end{split}
\end{equation}
    
   and
   
\begin{equation} \label{eqn:comp_output_foc_NC}
	\begin{split}
       	&  \frac{\partial \mathcal{L}}{\partial X_{m,t,NC}} = (1-\gamma_{m})^{\frac{1}{\ve_{3}}} X_{m,t,NC}^{\frac{-1}{\ve_{3}}} \left[\gamma_{m}^{\frac{1}{\ve_{3}}}X_{m,t,C}^{\frac{(\ve_{3}-1)}{\ve_{3}}}+(1-\gamma_{m})^{\frac{1}{\ve_{3}}}X_{m,t,NC}^{\frac{(\ve_{3}-1)}{\ve_{3}}}+\right]^{\frac{1}{(\ve_{3}-1)}} - \lambda p_{m,t,NC} = 0
	\end{split}
\end{equation}
    
    Solving the two necessary conditions, we can find the equations for the demand for the corporate and non-corporate output in industry $m$ as a function of the prices out output from each sector of industry $m$, price of the composite production good, the demand for the composite production good, and the parameters:
    
\begin{equation} \label{eqn:demand_XmtC}
	X_{m,t,C} = \frac{\gamma_{m}p_{m,t}X_{m,t}}{p_{m,t,C}^{\ve_{3}}\left[\gamma_{m}p_{m,t,C}^{1-\ve_{3}}+(1-\gamma_{m})p_{m,t,NC}^{1-\ve_{3}}\right]}
\end{equation}
    
    and 
    
    \begin{equation} \label{eqn:demand_XmtNC}
	X_{m,t,NC} = \frac{(1-\gamma_{m})p_{m,t}X_{m,t}}{p_{m,t,NC}^{\ve_{3}}\left[\gamma_{m}p_{m,t,C}^{1-\ve_{3}}+(1-\gamma_{m})p_{m,t,NC}^{1-\ve_{3}}\right]}
\end{equation}

To determine $p_{m,t}$, note that the CES subutility function defining preferences over corporate and non-corporate output within a production industry is linearly homogenous.  Because the subutility function is linearly homogenous, we know that the associated indirect utility function is homogenous of degree one in $X_{m,t}$.  Letting $V(\cdot)$ represent the indirect utility function, this means that $V(p_{m,t,C},p_{m,t,NC},\lambda X_{m,t}) = \lambda V(p_{m,t,C},p_{m,t,NC}, X_{m,t})$.  The linear homogeneity of the utility function also means that the indirect utility function is homogenous of degree -1 in prices.  That is, $V(\lambda p_{m,t,C},\lambda p_{m,t,NC}, X_{m,t}) = \frac{V(p_{m,t,C},p_{m,t,NC}, X_{m,t})}{\lambda}$. Linear homogeneity of the utility function means that: 

\begin{equation}
V(p_{m,t,C},p_{m,t,NC}, X_{m,t}) = \frac{p_{m,t}X_{m,t}}{e(p_{m,t,C},p_{m,t,NC})},
\end{equation}
    
    
\noindent\noindent where $e(p_{m,t,C},p_{m,t,NC})$ is the minimum expenditure for a unit of utility given prices.  Rearranging, we have: 
    
 \begin{equation}
 \label{eqn:price_comp}
 \begin{split}
& e(p_{m,t,C},p_{m,t,NC}) = \frac{p_{m,t}X_{m,t}}{V(p_{m,t,C},p_{m,t,NC}, X_{m,t})}\\
&\implies e(p_{m,t,C},p_{m,t,NC}) = p_{m,t}X_{m,t}/ \\
& {\left[\gamma_{m}^{\frac{1}{\ve_{3}}}\left( \frac{\gamma_{m}p_{m,t}X_{m,t}}{p_{m,t,C}^{\ve_{3}}\left[\gamma_{m}p_{m,t,C}^{1-\ve_{3}}+(1-\gamma_{m})p_{m,t,NC}^{1-\ve_{3}}\right]}\right)^{\frac{(\ve_{3}-1)}{\ve_{3}}}+(1-\gamma_{m})^{\frac{1}{\ve_{3}}}\left(\frac{(1-\gamma_{m})p_{m,t}X_{m,t}}{p_{m,t,NC}^{\ve_{3}}\left[\gamma_{m}p_{m,t,C}^{1-\ve_{3}}+(1-\gamma_{m})p_{m,t,NC}^{1-\ve_{3}}\right]}\right)^{\frac{(\ve_{3}-1)}{\ve_{3}}}+\right]^{\frac{\ve_{3}}{(\ve_{3}-1)}}}\\
&\implies e(p_{m,t,C},p_{m,t,NC}) = p_{m,t}X_{m,t}/ \\
& {p_{m,t}X_{m,t}\left[\gamma_{m}^{\frac{1}{\ve_{3}}}\left( \frac{\gamma_{m}}{p_{m,t,C}^{\ve_{3}}\left[\gamma_{m}p_{m,t,C}^{1-\ve_{3}}+(1-\gamma_{m})p_{m,t,NC}^{1-\ve_{3}}\right]}\right)^{\frac{(\ve_{3}-1)}{\ve_{3}}}+(1-\gamma_{m})^{\frac{1}{\ve_{3}}}\left(\frac{(1-\gamma_{m})}{p_{m,t,NC}^{\ve_{3}}\left[\gamma_{m}p_{m,t,C}^{1-\ve_{3}}+(1-\gamma_{m})p_{m,t,NC}^{1-\ve_{3}}\right]}\right)^{\frac{(\ve_{3}-1)}{\ve_{3}}}+\right]^{\frac{\ve_{3}}{(\ve_{3}-1)}}}\\
&\implies e(p_{m,t,C},p_{m,t,NC}) = 1/ \\
& \left[\gamma_{m}p_{m,t,C}^{1-\ve_{3}}+(1-\gamma_{m})p_{m,t,NC}^{1-\ve_{3}}\right]{\left[\gamma_{m}^{\frac{1}{\ve_{3}}}\left( \frac{\gamma_{m}}{p_{m,t,C}^{\ve_{3}}}\right)^{\frac{(\ve_{3}-1)}{\ve_{3}}}+(1-\gamma_{m})^{\frac{1}{\ve_{3}}}\left(\frac{(1-\gamma_{m})}{p_{m,t,NC}^{\ve_{3}}}\right)^{\frac{(\ve_{3}-1)}{\ve_{3}}}+\right]^{\frac{\ve_{3}}{(\ve_{3}-1)}}}\\
&\implies e(p_{m,t,C},p_{m,t,NC}) = 
\frac{\left[\gamma_{m}^{\frac{1}{\ve_{3}}}\left( \frac{\gamma_{m}}{p_{m,t,C}^{\ve_{3}}}\right)^{\frac{(\ve_{3}-1)}{\ve_{3}}}+(1-\gamma_{m})^{\frac{1}{\ve_{3}}}\left(\frac{(1-\gamma_{m})}{p_{m,t,NC}^{\ve_{3}}}\right)^{\frac{(\ve_{3}-1)}{\ve_{3}}}+\right]^{\frac{\ve_{3}}{(1-\ve_{3})}}}{\left[\gamma_{m}p_{m,t,C}^{1-\ve_{3}}+(1-\gamma_{m})p_{m,t,NC}^{1-\ve_{3}}\right]}\\
&\implies e(p_{m,t,C},p_{m,t,NC}) = 
\frac{\left[\gamma_{m}\left( \frac{1}{p_{m,t,C}^{\ve_{3}}}\right)^{\frac{(\ve_{3}-1)}{\ve_{3}}}+(1-\gamma_{m})\left(\frac{1}{p_{m,t,NC}^{\ve_{3}}}\right)^{\frac{(\ve_{3}-1)}{\ve_{3}}}+\right]^{\frac{\ve_{3}}{(1-\ve_{3})}}}{\left[\gamma_{m}p_{m,t,C}^{1-\ve_{3}}+(1-\gamma_{m})p_{m,t,NC}^{1-\ve_{3}}\right]}\\
&\implies e(p_{m,t,C},p_{m,t,NC}) = 
\frac{\left[\gamma_{m}p_{m,t,C}^{1-\ve_{3}}+(1-\gamma_{m})p_{m,t,NC}^{1-\ve_{3}}\right]^{\frac{\ve_{3}}{(1-\ve_{3})}}}{\left[\gamma_{m}p_{m,t,C}^{1-\ve_{3}}+(1-\gamma_{m})p_{m,t,NC}^{1-\ve_{3}}\right]}\\
&\implies e(p_{m,t,C},p_{m,t,NC}) = 
\left[\gamma_{m}p_{m,t,C}^{1-\ve_{3}}+(1-\gamma_{m})p_{m,t,NC}^{1-\ve_{3}}\right]^{\frac{1}{(1-\ve_{3})}}\\
\end{split}
\end{equation}   

\textcolor{red}{I'm not sure, but I think we then have  $e(p_{m,t,C},p_{m,t,NC})=p_{m,t}$ since the unit of ``utility" is really a unit of output of the composite production good, $X_{m,t}$}.

We can try to draw some nice tree structures as in \citet{FR1993}, but here's how the consumer side of the problem is solved;
\begin{enumerate}
\item The firm's problem determines prices of output from each sector and industry in a given year: $p_{m,t,C}$ and $p_{m,t,NC}$.
\item Given these, Equation \ref{eqn:price_comp} determines $p_{m,t}$, the price of composite output for sector $m$.
\item The price of composite output in each sector $m$ is related to the prices of the $I$ consumption goods through the matrix $Z$ as described in Equation \ref{eqn:mix_cons_price}.
\item Equation \ref{eqn:composite_price} describes how the composite consumption good price is determined from these individual consumption good prices, for each age $s$ composite consumption bundle.
\item With all prices in hand, we can solve for quantities demanded.  Using $p_{s,t}$ and $p_{i,t}$ in Equation \ref{EqBC}, we can then solve the consumers problem for the choice of $\tilde{c}_{j,s,t}$. 
\item These $\tilde{c}_{j,s,t}$ then determine demand for each individual consumption good, $c_{i,j,s,t}$ by Equation \ref{eqn:cons_solve}.
\item Demands for consumption goods are then translated into demands for output from each industry $m$ by Equation \ref{eqn:mix_cons}.
\item Demand for output from each sector (corporate or non-corporate) in industry $m$ are then determined by Equations \ref{eqn:demand_XmtC} and \ref{eqn:demand_XmtNC}.
\item These demands are then checked against the supply determined in (1).  If they match, we've found an eq'm. If not, we update the guess of $r$, $w$, $BQ$ and work from (1)-(8) again.
\end{enumerate}
    
    