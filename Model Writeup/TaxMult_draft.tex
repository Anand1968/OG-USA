	\documentclass[article,11pt,letterpaper,fleqn]{article}
\usepackage{graphicx,color}
\usepackage{array}
\usepackage{threeparttable}
\usepackage[format=hang,font=normalsize,labelfont=bf]{caption}
\usepackage{colortbl}
\usepackage{multirow}
\usepackage{geometry}
\usepackage{subfigure}
\geometry{letterpaper,tmargin=1in,bmargin=1in,lmargin=1.25in,rmargin=1.25in}
\usepackage{hyperref}
\hypersetup{colorlinks,%
citecolor=red,%
filecolor=red,%
linkcolor=red,%
urlcolor=blue,%
pdftex}
\usepackage{amsmath}
\usepackage{amssymb}
\usepackage{amsthm}
\usepackage{harvard}
\usepackage{setspace}
\usepackage{float,graphicx,color}
\usepackage{appendix}
\usepackage{longtable}
\newtheorem*{thm}{Theorem}
\theoremstyle{definition}
\numberwithin{equation}{section}
\newcommand{\cn}{\citeasnoun} % shortens command to cite as noun
\newcommand\ve{\varepsilon}



\author{Jason DeBacker\thanks{Tel: 770-289-0340, Email: \href{mailto: jason.debacker@gmail.com}{jason.debacker@gmail.com}} and Richard W. Evans\thanks{Brigham Young University,  Department of Economics, Email: \href{mailto: revans@byu.edu}{revans@byu.edu}}}
\title{Tax Multipliers in a DSGE Model\thanks{Preliminary-- Please do not cite without author's permission.}}
\date{First Draft: September 28, 2010\\ This Version: \today}


% make tables with smaller sized font 
\makeatletter
\def\table{\@ifnextchar[{\table@i}{\table@i[\fps@table]}}
\def\table@i[#1]{\@float{table}[#1]\footnotesize}
\makeatother



%\setlength{\topmargin}{-0.4in}
%\setlength{\topskip}{0.3in}    % between header and text
%\setlength{\textheight}{9.0in} % height of main text
%\setlength{\textwidth}{6in}    % width of text
%\setlength{\oddsidemargin}{39pt} %even side margin
%\setlength{\evensidemargin}{39pt} %odd side margin

\begin{document}
\bibliographystyle{aer}
\maketitle



\begin{abstract}
This paper analyzes the stabilization properties of tax cuts in a DSGE model.  
\end{abstract}






%\section{Introduction}
%\label{sec:intro}


Consumption taxes have long been favored by tax economists for their relative efficiency vis-a-vis taxes on capital and labor.  This paper attempts to argue that consumption taxes should also be favored by macroeconomists for their stabilization properties.  Cuts to consumption taxes can quickly and effectively stimulate aggregate demand. The following analysis studies the quantitative impact of tax cuts in a dynamic schocastic general equilibrium (DSGE) model.  In addition to consumption taxes, the model nests other tax policies, such as investment tax credits and changes to expensing and depreciation policies.  These policies target investment by lowering the after-tax cost of capital.  Historically, such instruments have been used by the U.S. to provide counter-cyclical fiscal policy (e.g. the Economic Recovery Act of 1981 increased the ability of businesses to take advantage of the investment tax credit).  Despite their use, such policies have not been studied in a DSGE environment.

The goal of this paper is to study specific tax policies in a DSGE environment.  In a sense, bring in the tax policies favored by tax economists into a model used by macroeconomists.  Recently, a number of researchers have used DSGE models to study the government spending multiplier (e.g., \cn{CER2010}), but less work has been done on tax multipliers.  Those who provide insights on tax multipliers (e.g., \cn{Zubairy2010}, \cn{IMF2010}) focus on multipliers associated with broad taxes on capital and labor income.  Such tax cuts are generally not the most effective counter-cyclical policies.  We are the first to provide quantitative insights into the the size of the multipliers associated with tax cuts that directly target consumption and investment.

Two recent, but related, events have spurred interest in consumption taxes; the 2007-2009 recession and the large federal budget deficits.  In a 2008 \emph{Financial Times} aricle \cn{LK_FT2008} argue that a national sales tax holiday would be an especially effective response to the dip in aggregate demand experienced during the 2007-2009 financial crisis.  While possible, it is difficult to implement such a policy without a national sales tax.  Recent debate about the fiscal sustainability of the federal government has brough much attention to value added taxes (VAT) as a way to increase revenue with minimal efficiency costs.  We hope to show that a tax on consumption has a second positive attribute - a non-zero tax on consumption allows policy makers important flexibility for macro stabilization policies.


Maybe cite some of the micro evidence of effects of sales tax holidays...

To cite:
- Romer and Romer (AER?) find tax multiplier of 3 with reduced form estimates
- More on multipliers: http://www.frbsf.org/publications/economics/letter/2009/el2009-20.html
- Nov 24, 2008 - UK cuts VAT from 17.5 to 15\% to respond to downturn - find more press on this.  And compare model results to actuals.  Note that 15\% is the lowest rate allowed by EU law.  VAT then raised to 20\% for a few years to cut budget deficits (part of 2010 budget plan).
- Note this for what is the Frisch labor elasticity: http://www.econ.umn.edu/~vr0j/ec8503-10/daolufrisch.pdf

Stress:
\begin{itemize}
\item lower dwl due to consumption tax than other forms of tax
\item faster impact of consumption tax cut
\item consumption beats spending because people get to choose what to spend it on (though we can't easily model this)
\item current focus on the VAT, which is a consumption tax.  Most of talk is about less distortionary way to raise revenue, but it would also allow to a good fiscal stabilizer.
\end{itemize}
Other novelties:
\begin{itemize}
\item build model do can handle other tax cuts like bonus depreciation (something simple so don't have to take account of vintage) and investment tax credits- see what the multiplier is for these in short and long run
\item Be best if could have heterogenous firms who vary in productivity and have price stickiness.  Build in corporate income tax, div tax, cap gains tax in more realistic detail (now all lumped in tax on capital).
\end{itemize}

\section{Environment}
\label{sec:model}

\subsection{Households}
Representative household.

HH maximizes:
\begin{equation}
U=E_{0}\sum_{t=0}^{\infty}\beta^{t}u(c_{i,t},l^{s}_{i,t},g_{i,t})
\end{equation}

We'll assume a per-period utility of the form:
\begin{equation}
u(c_{i,t},l^{s}_{i,t},g_{i,t}) = u^{b}_{t}\left( \frac{c^{1-\sigma}_{i,t}}{1-\sigma}-\theta\frac{(l^{s}_{i,t})^{1+\varphi}}{1+\varphi}+ \chi^{g}\frac{g_{t}^{1-\sigma_{g}}}{1-\sigma_{g}}\right)
\end{equation}

Where $u^{b}_{t}$ is a shock to the agent's impatience, $c$ is consumption of private good and services, $g$ consumption of public goods and services, and $l^{s}$ is labor supply.  The parameter $\sigma$ is the coefficient of relative risk aversion for consumption of private goods and services, and the parameter $\sigma_{g}$ is the analogue for consumption of public goods and services.  The parameter $\varphi$ is the inverse of the Frisch elasticity of labor supply.  $\theta$ and $\chi^{g}$ give the utility weight places on leisure and public goods consumption relative to private goods consumption.  

\begin{equation}
\label{patience_shock}
log(u^{b}_{t}) = \rho_{b}log(u^{b}_{t-1}) + \varepsilon^{b}_{t} 
\end{equation}

where $\varepsilon^{b}_{t} \sim N(0,\sigma_{b})$.

Households choose consumption, investment, and labor supply, subject to the real budget constraint:
\begin{equation}
\label{HH_BC}
\begin{split}
& c_{i,t}(1+\tau^{c}_{t}) + i_{i,t} + b_{i,t}   = \\ 
& (1-\tau^{k}_{t})r^{k}_{t}v_{i,t}k_{i,t-1} + \tau^{k}_{t}\delta^{tau}_{t}k^{\tau}_{i,t-1} +  \tau^{k}_{t}i_{i,t}e^{\tau}_{t} + \tau^{ic}_{t}i_{i,t} + (1-\tau^{l}_{t})w_{t}l^{s}_{i,t} + \frac{(1+(r_{t}(1-\tau^{i}_{t})))b_{i,t-1}}{\pi_{t}} + (1-\tau^{d}_{t})d_{i,t} + x_{t}  
\end{split}
\end{equation}

\noindent\noindent Where all variables pre-determined at time $t$ have subscripts strictly less than time $t$.  $c$ represents consumption, $i$ investment, $b$ government bond holdings, $\pi$ inflation ($\pi=\frac{P_{t}}{P{t-1}}$), $l$ labor supply, $k$ capital stock, $x$ government transfers, $d$ dividends, $w$ the nominal wage rate, $v$ intensity of capital utilization. Taxes $\tau^{i}$, $\tau^{l}$, $\tau^{k}$, $\tau^{d}$ are taxes on interest income, labor income, capital income, and dividend income.  $\tau_{c}$ is the tax rate on consumption.  $r$ is the nominal interest rate on government bonds and $r^{k}$ is the rental rate on capital.  $k^{\tau}$ is the tax basis for the household's capital stock, $\delta^{\tau}$ is the rate of depreciation for tax purposes, $e^{\tau}$ is the rate of expensing for tax purposes, and $\tau^{ic}$ is the investment tax credit.


The law of motion of the household's capital stock is:
\begin{equation}
\label{LOM_capital}
k_{i,t}=(1-\delta(v_{t}))k_{i,t-1} + i_{i,t}\left[1-S\left(\frac{i_{i,t}}{i_{i,t-1}}\right)\right]
\end{equation}

I'll assume that the investment adjustment cost function takes the form $S\left(\frac{i_{i,t}}{i_{i,t-1}}\right)=\frac{\kappa}{2}\left(\frac{u^{i}_{t}i_{i,t}}{i_{i,t-1}}-1\right)^2$.  Here, $\kappa$ is the adjustment cost parameter and $u^{i}_{t}$ is a investment specific efficiency shock.  Such a functional form is common in the literature (e.g., \cn{SW2003}, \cn{CER2010}, \cn{Zubairy2010}, et al) and has the nice properties that in the steady state $S=0,$ $S'=0$ and $S''>0$, and that in a deterministic steady state, the adjustment costs are zero. 

\begin{equation}
\label{invest_shock}
log(u^{i}_{t}) = \rho_{i}log(u^{i}_{t-1}) + \varepsilon^{i}_{t} 
\end{equation}

where $\varepsilon^{i}_{t} \sim N(0,\sigma_{i})$.

Assume also that $\delta(v_{t})$ takes the form: $\delta(v_{i,t})=\delta_{0}+\delta_{1}(v_{i,t}-1) + \frac{\delta_{2}}{2}(v_{i,t}-1)^2$.  This means that in the deterministic steady state, we can calibrate $\delta$ such that $v_{i,t}=1$, and $\delta(v)=\delta_{0}$ (see \cn{TY2010} on this point).  The steady state will also imply the relationship between $\delta_{0}$ and $\delta_{1}$, namely $\delta_{1} =  \frac{1}{\beta}-1+\delta_{0}$. 


The law of motion of the household's tax basis for it's capital stock is:
\begin{equation}
\label{LOM_tax_capital}
k_{i,t}^{\tau}=(1-\delta_{t}^{\tau})k_{i,t-1}^{\tau} + i_{i,t}(1-e_{t}^{\tau})
\end{equation}

Note that in general $\delta^{\tau}_{t}\neq \delta_{0}$, and typically tax depreciation is more accelerated than economic depreciation, lowering the user cost of capital.


\subsection{Firms}

\subsubsection{Final Goods Producers}

Final goods producers aggregate differentiated inputs to produce a homogenous output.  The Dixit-Stiglitz aggregation function is given by;

\begin{equation}
\label{aggY}
Y_{t}=\left(\int_{0}^{1} y_{i,t}^{\frac{1}{1+\eta_{p,t}}}di\right)^{1+\eta_{p,t}},
\end{equation}

\noindent\noindent where 
\begin{equation}
\label{markup_process}
log(\eta_{p,t})=\rho_{\eta}log(\eta_{p,t-1})+(1-\rho_{\eta})log(\eta_{p})+\varepsilon^{\eta}_{t}
\end{equation}
 is the stochastic price markup in the intermediate goods market.  We assume that $\varepsilon^{\eta}_{t}\sim N(0,\sigma_{\eta})$

Maximizing profits (or minimizing costs) subject to Equation \ref{aggY} results in demand for intermediate input $y_{i,t}$ of:

\begin{equation}
y_{i,t}=\left(\frac{p_{i,t}}{P_{t}}\right)^{-\frac{1+\eta_{p,t}}{\eta_{p,t}}}Y_{t} ,
\end{equation}

\noindent\noindent where the price of a unit of the final, homogenous output, $P_{t}$, is given by:

\begin{equation}
P_{t}=\left(\int_{0}^{1}p_{i,t}^{-\frac{1}{\eta_{p,t}}}di\right)^{-\eta_{p,t}}
\end{equation}

\subsubsection{Working Through the Final Goods Producer's problem}

Objective function:

\begin{equation}
\max_{y_{i,t}} P_{t}Y_{t}^{d} - \int_{0}^{1}p_{i,t}y_{i,t}di
\end{equation}

The FOC's are:

\begin{equation}
p_{t}(1+\eta_{p,t})\left(\int_{0}^{1}y_{i,t}^{\left(\frac{1}{1+\eta_{p,t}}\right)}di\right)^{\eta_{p,t}}\left(\frac{1}{1+\eta_{p,t}}\right)y_{i,t}^{\frac{1}{1+\eta_{p,t}}-1}-p_{i,t} = 0, \forall i
\end{equation}

Dividing the FOCs for intermediate inputs $i$ and $j$ $\implies$
\begin{equation}
\frac{p_{i,t}}{p_{j,t}} = \left(\frac{y_{i,t}}{y_{j,t}}\right)^{\frac{1}{1+\eta_{p,t}}-1} = \left(\frac{y_{i,t}}{y_{j,t}}\right)^{\frac{-\eta_{p,t}}{1+\eta_{p,t}}}
\end{equation}

Rearranging:
\begin{equation}
\begin{split}
&\implies p_{i,t} = \left(\frac{y_{j,t}}{y_{i,t}}\right)^{\frac{\eta_{p,t}}{1+\eta_{p,t}}}p_{j,t} \\
& \implies p_{i,t} = y_{j,t}^{\frac{\eta_{p,t}}{1+\eta_{p,t}}}p_{j,t}y_{i,t}^{\frac{-\eta_{p,t}}{1+\eta_{p,t}}} \\
& \implies p_{i,t}y_{i,t} = p_{j,t}y_{j,t}^{\left(\frac{\eta_{p,t}}{1+\eta_{p,t}}\right)}y_{i,t}^{\frac{1}{1+\eta_{p,t}}}
\end{split}
\end{equation}

Then integrate this condition to yield:
\begin{equation}
\begin{split}
\int_{0}^{1}p_{i,t}y_{i,t}di & = p_{j,t}y_{j,t}^{\left(\frac{\eta_{p,t}}{1+\eta_{p,t}}\right)}\int_{0}^{1}y_{i,t}^{\frac{1}{1+\eta_{p,t}}}di  \\
& = p_{j,t}y_{j,t}^{\left(\frac{\eta_{p,t}}{1+\eta_{p,t}}\right)}(Y_{t})^{\frac{1}{1+\eta_{p,t}}} 
\end{split}
\end{equation}

The zero profit condition implies $P_{t}Y_{t}=\int_{0}^{1}p_{i,t}y_{i,t}di$.  Plugging this into the above and we find:
\begin{equation}
\begin{split}
P_{t}Y_{t} &= p_{j,t}y_{j,t}^{\left(\frac{\eta_{p,t}}{1+\eta_{p,t}}\right)}(Y_{t})^{\frac{1}{1+\eta_{p,t}}} \\
\implies &P_{t}Y_{t}=p_{j,t}y_{j,t}^{\frac{\eta_{p,t}}{1+\eta_{p,t}}}(Y_{t})^{\frac{1}{1+\eta_{p,t}}} \\
\implies & P_{t} =  p_{j,t}y_{j,t}^{\frac{\eta_{p,t}}{1+\eta_{p,t}}}(Y_{t})^{\frac{-\eta_{p,t}}{1+\eta_{p,t}}}
\end{split}
\end{equation}

Which implies the demand function (just rearranging terms):
\begin{equation}
\begin{split}
y_{i,t} =  \left(\frac{p_{i,t}}{P_{t}}\right)^{\frac{-(1+\eta_{p,t})}{\eta_{p,t}}}Y_{t}, \forall i
\end{split}
\end{equation}

To find $P_{t}$ use the zero profit condition:
\begin{equation}
\begin{split}
& P_{t}Y_{t} = \int_{0}^{1}p_{i,t}y_{i,t}^{\left(\frac{\eta_{p,t}}{1+\eta_{p,t}}\right)}(Y_{t})^{\frac{1}{1+\eta_{p,t}}}di \\
\implies & P_{t}Y_{t} = \int_{0}^{1}p_{i,t}\left(\left(\frac{p_{i,t}}{P_{t}}\right)^{\frac{-(1+\eta_{p,t})}{\eta_{p,t}}}Y_{t}\right)^{\left(\frac{\eta_{p,t}}{1+\eta_{p,t}}\right)}(Y_{t})^{\frac{1}{1+\eta_{p,t}}}di \\
\implies & P_{t}Y_{t} = \int_{0}^{1}p_{i,t}\left(\frac{p_{i,t}}{P_{t}}\right)^{\frac{-(1+\eta_{p,t})}{\eta_{p,t}}}Y_{t}di \\
\implies & P_{t}Y_{t} = \int_{0}^{1}p_{i,t}p_{i,t}^{\frac{-(1+\eta_{p,t})}{\eta_{p,t}}}P_{t}^{\frac{(1+\eta_{p,t})}{\eta_{p,t}}}Y_{t}di \\
\implies & P_{t}Y_{t} = \int_{0}^{1}p_{i,t}^{\frac{-1}{\eta_{p,t}}}di \:  Y_{t}P_{t}^{\frac{1+\eta_{p,t}}{\eta_{p,t}}} \\
\implies & P_{t} = \int_{0}^{1}p_{i,t}^{\frac{-1}{\eta_{p,t}}}di \: P_{t}^{\frac{1+\eta_{p,t}}{\eta_{p,t}}} \\
\implies & P_{t}^{\frac{-1}{\eta_{p,t}}} = \int_{0}^{1}p_{i,t}^{\frac{-1}{\eta_{p,t}}}di \\
\implies & P_{t} = \left(\int_{0}^{1}p_{i,t}^{\frac{-1}{\eta_{p,t}}}di\right)^{-\eta_{p,t}} \\
\end{split}
\end{equation}



\subsubsection{Intermediate Goods Producers}

Each intermediate-goods producing firm has a monopoly on its heterogeneous output.  Following \cn{Calvo1983}, firms can set prices optimally each period with probability $(1-\omega_{p})$.  Firm's that cannot reset prices must rent labor and capital inputs to meet demand at last period's price, indexed to past inflation, $p_{i,t}=p_{i,t-1}\pi_{t-1}^{\chi_{p}}$.  Each intermediate good's producer has the same production function and technology:

\begin{equation}
\label{int_prodfunc}
y_{i,t}=z_{t}\tilde{k}_{i,t}^{\alpha}l_{i,t}^{1-\alpha} - \Phi z_{t},
\end{equation} 

\noindent\noindent where $\tilde{k}_{i,t}=v_{t}k_{i,t-1}$ are the effective units of capital rented from the households, $l_{i,t}$ is the effective labor rented from households, $z_{t}$ is a serially correlated shock to firm productivity (which affect all intermediate-goods producers equally). $\Phi$ are fixed costs of production that are proportional to the technology shock. 

We assume $z_{t}$ follows the following AR(1) process:
\begin{equation}\label{EqProdShock}
   \log(z_{t}) = \rho_z\log(z_{t-1}) + \varepsilon^{z}_{t} \quad\text{where}\quad\varepsilon^{z}_{t}\sim N(0,\sigma_z), \quad\text{and}\quad \rho_z\in[0,1)
\end{equation}

There is no entry or exit.

The problem of intermediate-goods producers who can optimally reset their prices can be solved in two stages.  In the first, the demand for factor inputs are found, taking the rental rate for capital, $r_{t}^{k}$, and the wage rate, $w_{t}$, as given.  This stage yields the relative factor demands and the marginal cost of production.  In the second stage, firm's take their marginal cost function as given, and choose the their, $p_{i,t}$ to maximize their expected, discounted profits.

The firms' per-period profit function takes the following form:
\begin{equation}
   d_{i,t} = p_{i,t}y_{i,t} - r_t^k\tilde{k}_{i,t} - w_t l_{i,t} \quad\forall i,t
\end{equation}

Thus, in the first stage, the firm solves:
\begin{equation}
\min_{\tilde{k}_{i,t},l_{i,t}} w_{t}l_{i,t} + r_{t}\tilde{k}_{i,t}
\end{equation}

subject to the supply curve: 
\begin{equation}
y_{i,t}=\begin{cases}
    z_{t}\tilde{k}_{i,t}^{\alpha}l_{i,t}^{1-\alpha} - \Phi z_{t}, & \text{if $z_{t}\tilde{k}_{i,t}^{\alpha}l_{i,t}^{1-\alpha}>\Phi z_{t}$}.\\
    0, & \text{otherwise}.
  \end{cases}
\end{equation}

FOC's:

\begin{equation}
\label{foc_int_k}
\frac{\partial E}{\partial \tilde{k}_{i,t}}:\frac{\alpha p_{i,t}y_{i,t}}{\tilde{k}_{i,t}}-r_{t}^{k}=0
\end{equation}

\begin{equation}
\label{foc_int_l}
\frac{\partial E}{\partial l_{i,t}}:\frac{(1-\alpha) p_{i,t}y_{i,t}}{l_{i,t}}-w_{t}=0
\end{equation}

Dividing Equation \ref{foc_int_k} by Equation \ref{foc_int_l} and rearranging some terms, one gets:

\begin{equation}
\label{k_l_ratio}
\frac{\tilde{k}_{i,t}}{l_{i,t}}=\left(\frac{\alpha}{1-\alpha}\right)\frac{w_{t}}{r_{t}^{k}}
\end{equation}


Plugging this ratio into Equation \ref{int_prodfunc} and setting it equal to 1, one can find the amount of labor needed to produce one unit of output in terms of $w_{t}, r{t}^{k}, z_{t}$, and the parameter $\alpha$.  Rearranging \ref{k_l_ratio} to put $\tilde{k}_{i,t}$ in terms of $l_{i,t}$, factor input prices, and parameters then writing the cost function ($r_{t+j}^{k}\tilde{k}_{i,t+j}+w_{t+j}l_{i,t+j}$) in terms of $l_{i,t}$, factor input prices, and parameters, one can substitute in the equation for $l_{i,t}$ This yields the marginal cost of production in real terms (this procedure is described in \cn{FVRR2006}):

\begin{equation}
\label{marg_cost}
mc_{t}=\frac{(r^{k}_{t})^{\alpha}w_{t}^{1-\alpha}}{z_{t}}\left(\frac{1}{1-\alpha}\right)^{1-\alpha}\left(\frac{1}{\alpha}\right)^{\alpha}
\end{equation}

\noindent\noindent N.B., the marginal cost function, $mc_{t}$ does not depend upon $i$.  This is because all firms receive the same technology shock and face the same factor input prices.


The second stage, were firms optimally choose price given the optimal choice of factor inputs, can be setup as a profit maximization problem, taking $mc_{t}$ as given.  Assuming an interior solution, the problem can be written as:

\begin{equation}
V=\max_{p_{i,t}}E_{t}\Biggl(\sum_{j=0}^{\infty}\beta^{j}\omega^{j}_{p}\frac{\lambda_{t+j}}{\lambda_{t}}\left[\left(\prod_{s=1}^{j}\pi^{\chi_{p}}_{t+s-1}\frac{p_{i,t}}{P_{t+j}}-mc_{t+j}\right)y_{i,t+j}\right]\Biggr),
\end{equation}


\noindent\noindent subject to: 
\begin{equation}
y_{i,t+j}=\left(\prod_{s=1}^{j}\pi^{\chi_{p}}_{t+s-1}\frac{p_{i,t}}{P_{t+j}}\right)^{-\left(\frac{1+\eta_{p,t+j}}{\eta_{p,t+j}}\right)}Y_{t+j}
\end{equation}

\noindent\noindent where $Y_{t+j}$ is the demand for final goods in period $t+j$.  N.B. there is no $p_{i,t+j}$ because prices are fixed at the value chosen in period $t$.  Note also that the value of profits are weighted by the shadow value of income for the households (since all profits are returned to households).


Rewriting the second-stage problem gives:

\begin{equation}
\begin{split}
 & V=\max_{p_{i,t}} E_{t}\left(\sum_{j=0}^{\infty}\beta^{j}\omega^{j}_{p}\frac{\lambda_{t+j}}{\lambda_{t}}\left[\left(\prod_{s=1}^{j}\pi^{\chi_{p}}_{t+s-1}\frac{p_{i,t}}{P_{t+j}}-mc_{t+j}\right)y_{i,t+j}\right]\right) 
\\
\implies & V=\max_{p_{i,t}} E_{t}\left(\sum_{j=0}^{\infty} \beta^{j}\omega^{j}_{p}\frac{\lambda_{t+j}}{\lambda_{t}}\left[\left(\prod_{s=1}^{j}\pi^{\chi_{p}}_{t+s-1}\frac{p_{i,t}}{P_{t+j}}-mc_{t+j}\right)\left(\prod_{s=1}^{j}\pi^{\chi_{p}}_{t+s-1}\frac{p_{i,t}}{P_{t+j}}\right)^{-\frac{1+\eta_{p,t+j}}{\eta_{p,t+j}}}Y_{t+j}\right]\right) \\
 \implies & V=\max_{p_{i,t}} E_{t}\left(\sum_{j=0}^{\infty} \beta^{j}\omega^{j}_{p}\frac{\lambda_{t+j}}{\lambda_{t}}\left[\left(\prod_{s=1}^{j}\pi^{\chi_{p}}_{t+s-1}\frac{p_{i,t}}{P_{t+j}}\right)^{\frac{-1}{\eta_{p,t+j}}}-\left(\prod_{s=1}^{j}\pi^{\chi_{p}}_{t+s-1}\frac{p_{i,t}}{P_{t+j}}\right)^{-\frac{1+\eta_{p,t+j}}{\eta_{p,t+j}}}mc_{t+j}\right]Y_{t+j}\right) \\
 \implies & V=\max_{p_{i,t}} E_{t}\left(\sum_{j=0}^{\infty} \beta^{j}\omega^{j}_{p}\frac{\lambda_{t+j}}{\lambda_{t}}\left[\left(\prod_{s=1}^{j}\pi^{\chi_{p}}_{t+s-1}\frac{p_{i,t}}{P_{t+j}}\right)^{\frac{-1}{\eta_{p,t+j}}}-\left(\prod_{s=1}^{j}\pi^{\chi_{p}}_{t+s-1}\frac{p_{i,t}}{P_{t+j}}\right)^{-\frac{1+\eta_{p,t+j}}{\eta_{p,t+j}}}mc_{t+j}\right]Y_{t+j}\right) \\
  \implies & V=\max_{p_{i,t}} E_{t}\left(\sum_{j=0}^{\infty} \beta^{j}\omega^{j}_{p}\frac{\lambda_{t+j}}{\lambda_{t}}\left[\left(\prod_{s=1}^{j}\frac{\pi^{\chi_{p}}_{t+s-1}}{\pi_{t+s}}\frac{p_{i,t}}{P_{t}}\right)^{\frac{-1}{\eta_{p,t+j}}}-\left(\prod_{s=1}^{j}\frac{\pi^{\chi_{p}}_{t+s-1}}{\pi_{t+s}}\frac{p_{i,t}}{P_{t}}\right)^{-\frac{1+\eta_{p,t+j}}{\eta_{p,t+j}}}mc_{t+j}\right]Y_{t+j}\right) 
\end{split}
\end{equation}

In the above, $\pi_{t}=\frac{P_{t}}{P_{t-1}}$.

The FOC for this problem (where the intermediate-goods producer is choosing $p_{i,t}$) is given by:

\begin{equation}
\begin{split}
 & \frac{\partial V}{\partial p_{i,t}}: E_{t}\left(\sum_{j=0}^{\infty} \beta^{j}\omega^{j}_{p}\frac{\lambda_{t+j}}{\lambda_{t}}\left[\left(\prod_{s=1}^{j}\frac{\pi^{\chi_{p}}_{t+s-1}}{\pi_{t+s}}\frac{1}{P_{t}}\right)^{\frac{-1}{\eta_{p,t+j}}}\frac{-1}{\eta_{p,t}}{p^{*}_{i,t}}^{\frac{-1}{\eta_{p,t}}-1}- \right. \right. \\
 & \left. \left. \left(\prod_{s=1}^{j}\frac{\pi^{\chi_{p}}_{t+s-1}}{\pi_{t+s}}\frac{1}{P_{t}}\right)^{-\frac{1+\eta_{p,t+j}}{\eta_{p,t+j}}}\left(\frac{-1-\eta_{p,t}}{\eta_{p,t}}\right){p^{*}_{i,t}}^{\frac{-1-\eta_{p,t}}{\eta_{p,t}}-1}mc_{t+j}\right]Y_{t+j}\right) = 0 \\
\implies & \frac{\partial V}{\partial p_{i,t}}: E_{t}\left(\sum_{j=0}^{\infty} \beta^{j}\omega^{j}_{p}\frac{\lambda_{t+j}}{\lambda_{t}}\left[\frac{-1}{\eta_{p,t}}\left(\prod_{s=1}^{j}\frac{\pi^{\chi_{p}}_{t+s-1}}{\pi_{t+s}}\frac{p^{*}_{i,t}}{P_{t}}\right)^{\frac{-1}{\eta_{p,t+j}}}{p^{*}_{i,t}}^{-1} +  \right. \right. \\
 & \left. \left. \left(\frac{1+\eta_{p,t}}{\eta_{p,t}}\right)\left(\prod_{s=1}^{j}\frac{\pi^{\chi_{p}}_{t+s-1}}{\pi_{t+s}}\frac{p^{*}_{i,t}}{P_{t}}\right)^{-\frac{1+\eta_{p,t+j}}{\eta_{p,t+j}}}{p^{*}_{i,t}}^{-1}mc_{t+j}\right]Y_{t+j}\right) = 0 \\
\end{split}
\end{equation}

Rearranging (and factoring out variables determined in period $t$ from the expectations and summation operations):
\begin{equation}
\begin{split}
\implies & \frac{\partial V}{\partial p_{i,t}}: E_{t}\left(\sum_{j=0}^{\infty} \beta^{j}\omega^{j}_{p}\lambda_{t+j}\left[\frac{-1}{\eta_{p,t}}\left(\prod_{s=1}^{j}\frac{\pi^{\chi_{p}}_{t+s-1}}{\pi_{t+s}}\right)^{\frac{-1}{\eta_{p,t+j}}} +  \right. \right. \\
 & \left. \left. \left(\frac{1+\eta_{p,t}}{\eta_{p,t}}\right)\left(\prod_{s=1}^{j}\frac{\pi^{\chi_{p}}_{t+s-1}}{\pi_{t+s}}\right)^{-\frac{1+\eta_{p,t+j}}{\eta_{p,t+j}}}\frac{p^{*}_{i,t}}{P_{t}}^{-1}mc_{t+j}\right]Y_{t+j}\right) = 0  
\end{split}
\end{equation}

Now, multiply both sides by $\frac{p^{*}_{i,t}}{P_{t}}$ and get

\begin{equation}
\begin{split}
\implies & \frac{\partial V}{\partial p_{i,t}}: E_{t}\left(\sum_{j=0}^{\infty} \beta^{j}\omega^{j}_{p}\lambda_{t+j}\left[\frac{-1}{\eta_{p,t}}\left(\prod_{s=1}^{j}\frac{\pi^{\chi_{p}}_{t+s-1}}{\pi_{t+s}}\right)^{\frac{-1}{\eta_{p,t+j}}}\frac{p^{*}_{i,t}}{P_{t}} +  \right. \right. \\
 & \left. \left. \left(\frac{1+\eta_{p,t}}{\eta_{p,t}}\right)\left(\prod_{s=1}^{j}\frac{\pi^{\chi_{p}}_{t+s-1}}{\pi_{t+s}}\right)^{-\frac{1+\eta_{p,t+j}}{\eta_{p,t+j}}}mc_{t+j}\right]Y_{t+j}\right) = 0  
\end{split}
\end{equation}

$p^{*}_{i,t}$ solves this equation.  We'll consider only symmetric equilibrium, thus write, $p^{*}_{i,t}=p^{*}_{t}$.  Also, define two terms to help define this relationship recursively:

Which implies:
\begin{equation}
g^{1}_{t}= E_{t}\sum_{j=0}^{\infty} \beta^{j}\omega^{j}_{p}\lambda_{t+j}\left(\prod_{s=1}^{j}\frac{\pi^{\chi_{p}}_{t+s-1}}{\pi_{t+s}}\right)^{-\frac{1+\eta_{p,t+j}}{\eta_{p,t+j}}}mc_{t+j}Y_{t+j}
\end{equation}

and

Which implies:
\begin{equation}
g^{2}_{t}= E_{t}\sum_{j=0}^{\infty} \beta^{j}\omega^{j}_{p}\lambda_{t+j}\left(\prod_{s=1}^{j}\frac{\pi^{\chi_{p}}_{t+s-1}}{\pi_{t+s}}\right)^{\frac{-1}{\eta_{p,t+j}}}\frac{p^{*}_{i,t}}{P_{t}}Y_{t+j}
\end{equation}

With these, we can write the FOC as: 
\begin{equation}
\label{price_rule}
\frac{1+\eta_{p,t+j}}{\eta_{p,t+j}}g^{1}_{t}={\frac{1}{\eta_{p,t+j}}}g^{2}_{t}
\end{equation}

We can then define these functions recursively:

\begin{equation}
\label{g1}
g^{1}_{t}= \lambda_{t}mc_{t}Y_{t} + \beta\omega_{p}E_{t}\left(\frac{\pi^{\chi_{p}}_{t}}{\pi_{t+1}}\right)^{-\frac{1+\eta_{p,t+j}}{\eta_{p,t+j}}}g^{1}_{t+1}
\end{equation}

\begin{equation}
\label{g2}
g^{2}_{t}= \lambda_{t}\pi^{*}_{t}Y_{t} + \beta\omega_{p}E_{t}\left(\frac{\pi^{\chi_{p}}_{t}}{\pi_{t+1}}\right)^{\frac{-1}{\eta_{p,t+j}}}\left(\frac{\pi^{*}_{t}}{\pi^{*}_{t+1}}\right)g^{2}_{t+1}
\end{equation}

where $\pi^{*}_{t}={p^{*}_{t}}{P_{t}}$.


Calvo pricing implies:
\begin{equation}
\label{Calvo_price}
P_{t}^{\frac{-1}{\eta_{p,t}}}= \omega_{p}\left(\pi^{\chi}_{t-1}\right)^{\frac{-1}{\eta_{p,t}}} P_{t-1}^{\frac{-1}{\eta_{p,t}}} +(1-\omega_{p})p_{t}^{*\frac{-1}{\eta_{p,t}}}
\end{equation}

 You can derive the above from the definition of the Calvo model and the aggregate price index from the final goods producers problem.  


Nominal profits from the intermediate-goods producers, $d_{i,t}$ are distributed as dividends to households.  



\subsection{Monetary Authority}

The monetary authority follows a Taylor rule, targeting the nominal, after-tax interest rate (I'm taking this from \cn{CER2010}, but feel free to modify to whatever- though I think it's important that we have the zero bound condition):

\begin{equation}
\label{Taylor_Rule}
r_{t+1} = \max\Biggl\{\frac{\frac{1}{\beta}(1+\pi_{t})^{\phi_{1}}\left(\frac{Y_{t}}{\bar{Y}}\right)^{\phi_{2}}\left(\frac{1+r_{t}}{1+\bar{r}}\right)^{\rho_{r}} - 1}{(1-\tau^{i}_{t})}+\varepsilon^{m}_{t},\: 0\Biggr\}
\end{equation}

Where $\pi_{t}=\frac{P_{t}-P_{t-1}}{P_{t-1}}$ and $\phi_{1} > 1$,  $\phi_{2}\in(0,1)$, and  $\rho_{r}\in(0,1)$.  $\varepsilon^{m}_{t}\sim N(0,\sigma^{m})$.

\subsection{Government/Fiscal Authority}

The government is described by a budget constraint and fiscal policy rules (like Taylor rules, but for fiscal policy).  Some, like \cn{Zubairy2010} and \cn{TY2010}, have equations like Taylor rules for fiscal policy, the parameters of which they estimate using historical data.  We will follow this.  To then get multiplier effects, you just look at how outcomes vary when the policy rules are shocked.

\subsection{Fiscal authority equations}\label{TAppSecFiscAuth}

Government budget constraint:
\begin{equation}\label{TAppEqGovtBC}
    G_{t} +  \frac{(1+r_{t})B_{t-1}}{\pi_{t}} + X_{t} = T_{t} + B_{t} 
\end{equation}

The left-hand side are expenditures ($G$ are expenditures on public goods, $X$ on transfers), the right hand side  are revenues from taxes, $T$, and bond sales.

Total tax revenue is given by:
\begin{equation}\label{tax_rev}
  T_{t} = \tau^{l}_{t}w_{t}L_{t} + \tau^{k}_{t}r^{k}_{t}\tilde{K}_{t} + \tau^{d}_{t}D_{t} + \tau^{i}_{t}r_{t}\frac{B_{t-1}}{\pi_{t}} + \tau^{c}_{t}C_{t} - \tau^{k}_{t}\delta^{\tau}_{t}K^{\tau}_{t-1} - \tau^{k}_{t}I_{t}e^{\tau}_{t} - \tau^{ic}_{t}I_{t}
\end{equation}

Government spending and tax policy will be given by fiscal policy rules.  These fiscal policy rules will react to the state of the economy and the amount of debt the government is carrying.  Thus, they will stabilize government debt around it's steady state level.  The amount of transfers, $X_{t}$, are thus determined as the residual in the government budget constraint.

The fiscal policy rules are as follows (where ``hats" denote percent deviations from the steady state values; i.e. $\hat{b}_{t-1}=\frac{B_{t-1}-\bar{B}}{\bar{B}}$):

\begin{equation}\label{gov_spend}
\hat{g}_{t} = \rho_{g}\hat{g}_{t-1} - \rho_{gb}\hat{b}_{t-1} + \rho_{gy}\hat{y}_{t-1} + e^{g}_{t} 
\end{equation}

\begin{equation}\label{transfers}
\hat{x}_{t} = \rho_{x}\hat{x}_{t-1} - \rho_{xb}\hat{b}_{t-1} + \rho_{xy}\hat{y}_{t-1} + e^{x}_{t}  
\end{equation}

\begin{equation}\label{labor_tax}
\hat{\tau}^{l}_{t} = \rho_{l}\hat{\tau}^{l}_{t-1} + \rho_{lb}\hat{b}_{t-1} + \rho_{ly}\hat{y}_{t-1} + e^{l}_{t} 
\end{equation}

\begin{equation}\label{capital_tax}
\hat{\tau}^{k}_{t} = \rho_{k}\hat{\tau}^{k}_{t-1} + \rho_{kb}\hat{b}_{t-1} + \rho_{ky}\hat{y}_{t-1} + e^{k}_{t} 
\end{equation}

\begin{equation}\label{interest_tax}
\hat{\tau}^{i}_{t} = \rho^{\tau}_{i}\hat{\tau}^{i}_{t-1} + \rho_{ib}\hat{b}_{t-1} + \rho_{iy}\hat{y}_{t-1} + e^{i}_{t} 
\end{equation}

\begin{equation}\label{dividend_tax}
\hat{\tau}^{d}_{t} = \rho_{d}\hat{\tau}^{d}_{t-1} + \rho_{db}\hat{b}_{t-1} + \rho_{dy}\hat{y}_{t-1} + e^{d}_{t} 
\end{equation}

\begin{equation}\label{consumption_tax}
\hat{\tau}^{c}_{t} = \rho_{c}\hat{\tau}^{c}_{t-1} +  \rho_{cb}\hat{b}_{t-1} + \rho_{cy}\hat{y}_{t-1} + e^{c}_{t}  
\end{equation}

The disturbances to these fiscal policy rules may have some correlation.  Currently, I only have correlation between capital and labor taxes (as in \cn{Zubairy2010}).

Namely:

$e^{k}_{t} = \phi_{\tau}\varepsilon^{l}_{t} + \varepsilon^{k}_{t}$\\

$e^{l}_{t} = \phi_{\tau}\varepsilon^{k}_{t} + \varepsilon^{l}_{t} $\\

Where, $\phi_{\tau}$ is correlation between labor and capital tax shocks.  The other variables are uncorrelated.

$e^{d}_{t} = \varepsilon^{d}_{t}$\\

$e^{i}_{t} = \varepsilon^{i}_{t}$\\

$e^{c}_{t} = \varepsilon^{c}_{t}$\\

$e^{g}_{t} = \varepsilon^{g}_{t}$\\

$e^{x}_{t} = \varepsilon^{x}_{t}$\\

Note that we assume $\varepsilon^{s}_{t} \sim i.i.d. \ N(0,\sigma_{s})$ for $s\in\{l, k, d, c, i, g, x\}$ (except the notation for the interest tax is $\sigma^{\tau}_{i}$ and $\varepsilon^{tau,i}_{t}$).

For simplicity, we assume that the tax depreciation rate, investment expensing rate, and investment tax credits are constant.  Thus, $\delta^{\tau}_{t}=\bar{\delta}^{\tau}$, $e^{\tau}_{t}=\bar{e}^{\tau}$, and $\tau^{ic}_{t}=\bar{\tau}^{ic}$, $\forall \ t$.

\subsection{Aggregation}


\subsubsection{Aggregate factor inputs}
Labor demand:
$L^{d}_{t} = \int_{0}^{1}l_{i,t}di $

Effective capital demand:

$\tilde{K}_{t} = \int_{0}^{1}\tilde{k}_{i,t}di=  \int_{0}^{1}v_{i,t}k_{i,t}di $

And note that since all firms have the same capital-labor ratio, it must be the case that: $\frac{\tilde{k}_{i,t}}{l_{i,t}}=\frac{\tilde{K}_{t}}{L_{t}}$

\subsubsection{Aggregate Output}
Supply from each $i$ intermediate goods producer is:
\begin{equation}
y^{s}_{i,t} = max\{0,z_{t}\tilde{k}^{\alpha}_{i,t}l^{1-\alpha}_{i,t}-z_{t}\Phi\}
\end{equation}

Demand from each $i$ intermediate goods producer is:
\begin{equation}
y^{d}_{i,t} = \left(\frac{p_{i,t}}{P_{t}}\right)^{\frac{-(1+\eta_{p,t})}{\eta_{p,t}}}Y_{t}
\end{equation}

In equilibrium, $y^{s}_{i,t}=y^{d}_{i,t}$ and in aggregate: $Y_{t}=\int_{0}^{1}y^{d}_{i,t}di = \int_{0}^{1}y^{s}_{i,t}di$.  Thus we have:

\begin{equation}
\begin{split}
\label{aggY}
& \int_{0}^{1}y^{d}_{i,t}di = \int_{0}^{1}y^{s}_{i,t}di \\
 & \implies  \int_{0}^{1}\left( \left(\frac{p_{i,t}}{P_{t}}\right)^{\frac{-(1+\eta_{p,t})}{\eta_{p,t}}}Y_{t}\right)di = \int_{0}^{1}\left(z_{t}\tilde{k}^{\alpha}_{i,t}l^{1-\alpha}_{i,t}-z_{t}\Phi\right)di \\
  & \implies  Y_{t} \int_{0}^{1} \underbrace{\left(\frac{p_{i,t}}{P_{t}}\right)^{\frac{-(1+\eta_{p,t})}{\eta_{p,t}}}}_{\nu^{1}_{p,t}}di = \int_{0}^{1}\left(z_{t}\left(\frac{\tilde{k}_{i,t}}{l_{i,t}}\right)^{\alpha}l_{i,t}\right)di-z_{t}\Phi \\
  & \implies  Y_{t} \nu^{1}_{p,t} = z_{t}\int_{0}^{1}\left(\frac{\tilde{K}_{t}}{L_{t}}\right)^{\alpha}l_{i,t}di-z_{t}\Phi \\
   & \implies  Y_{t} \nu^{1}_{p,t} = z_{t}\left(\frac{\tilde{K}_{t}}{L_{t}}\right)^{\alpha}\int_{0}^{1}l_{i,t}di-z_{t}\Phi \\
    & \implies  Y_{t} \nu^{1}_{p,t} = z_{t}\left(\frac{\tilde{K}_{t}}{L_{t}}\right)^{\alpha}L_{t}-z_{t}\Phi \\
    & \implies  Y_{t} \nu^{1}_{p,t} = z_{t}\tilde{K}_{t}^{\alpha}L_{t}^{1-\alpha}-z_{t}\Phi \\
    & \implies Y_{t} = \frac{z_{t}\tilde{K}_{i,t}^{\alpha}L_{i,t}^{1-\alpha}-z_{t}\Phi}{\nu^{1}_{p,t}} 
\end{split}
\end{equation}

Where,
\begin{equation}
\label{price_disp2}
\nu^{1}_{p,t}=\int_{0}^{1}\left(\frac{p_{i,t}}{P_{t}}\right)^{\frac{-(1+\eta_{p,t})}{\eta_{p,t}}}di
\end{equation}

We can use the Calvo Pricing to define this recursively: $\nu^{1}_{p,t}=\omega_{p}\left(\frac{\pi^{\chi}_{t-1}}{\pi_{t}}\right)^{\frac{-(1+\eta_{p,t})}{\eta_{p,t}}}\nu^{1}_{p,t-1}+ (1-\omega_{p}\left(\frac{p^{*}_{t}}{P_{t}}\right)^{\frac{-(1+\eta_{p,t})}{\eta_{p,t}}}$

\subsubsection{Aggregate Profits}
Real profits for firm $i$ in period $t$ are given by:
\begin{equation}
d_{i,t}=\begin{cases}
    \left(\frac{p_{i,t}}{P_{t}}-mc_{t}\right)y_{i,t}-z_{t}\Phi , & \text{if $y_{i,t}>0$} \\
    0, & \text{otherwise}
      \end{cases}
\end{equation}

If profits are positive, we can rewrite $d_{i,t}$:
\begin{equation}
\begin{split}
d_{i,t}& =  \left(\frac{p_{i,t}}{P_{t}}-mc_{t}\right)y_{i,t}-z_{t}\Phi \\
 & =  \left(\frac{p_{i,t}}{P_{t}}-mc_{t}\right)\left(\frac{p_{i,t}}{P_{t}}\right)^{\frac{-(1+\eta_{p,t})}{\eta_{p,t}}}Y_{t}-z_{t}\Phi \\
 & = Y_{t}\left[\left(\frac{p_{i,t}}{P_{t}}\right)^{\frac{-1}{\eta_{p,t}}}-mc_{t}\left(\frac{p_{i,t}}{P_{t}}\right)^{\frac{-(1+\eta_{p,t})}{\eta_{p,t}}}\right]-z_{t}\Phi \\
\end{split}
\end{equation}

Then aggregating:

\begin{equation}
\begin{split}
\label{agg_profits}
D_{t}= \int_{0}^{1}d_{i,t}di & = \int_{0}^{1}\left(Y_{t}\left[\left(\frac{p_{i,t}}{P_{t}}\right)^{\frac{-1}{\eta_{p,t}}}-mc_{t}\left(\frac{p_{i,t}}{P_{t}}\right)^{\frac{-(1+\eta_{p,t})}{\eta_{p,t}}}\right]-z_{t}\Phi\right)di \\
 & =   Y_{t}\underbrace{\int_{0}^{1}\left(\frac{p_{i,t}}{P_{t}}\right)^{\frac{-1}{\eta_{p,t}}}di}_{\nu^{2}_{p,t}} - mc_{t}Y_{t}\underbrace{\int_{0}^{1}\left(\frac{p_{i,t}}{P_{t}}\right)^{\frac{-(1+\eta_{p,t})}{\eta_{p,t}}}di}_{\nu^{1}_{p,t}}-z_{t}\Phi \\
 & = Y_{t}\left[\nu^{2}_{p,t} - mc_{t}\nu^{1}_{p,t}\right] - z_{t}\Phi
\end{split}
\end{equation}

Where: 
\begin{equation}
\label{price_disp2}
\nu^{2}_{p,t}=\int_{0}^{1}\left(\frac{p_{i,t}}{P_{t}}\right)^{\frac{-1}{\eta_{p,t}}}di
\end{equation}

 $\nu^{2}_{p,t}$ can further be written recursively: $\nu^{2}_{p,t} = \omega_{p}\left(\frac{\pi^{\chi}_{t-1}}{\pi_{t}}\right)^{\frac{-1}{\eta_{p,t}}}\nu^{2}_{p,t-1} + (1-\omega_{p})\left(\frac{p^{*}_{t}}{P_{t}}\right)^{\frac{-1}{\eta_{p,t}}}$


\subsubsection{Other aggregates:}

Consumption: $C_{t}=\int_{0}^{1} c_{i,t} di $\\

Investment: $I_{t} = \int_{0}^{1} i_{i,t} di $\\

Labor Supply: $L^{s}_{t} =  \int_{0}^{1} l^{s}_{i,t} di $\\

Tax Depreciable Basis: $K^{tau}_{t} = \int_{0}^{1} k^{tau}_{i,t} di $\\

Bond holdings : $B_{t} = \int_{0}^{1} b_{i,t} di $\\

Transfers: $X_{t} =  x_{t} di $ (Note, no $i$ subscript because lump sum, so all get same)\\

Public Goods: $G_{t}  =  g_{t} di $ (Note, no $i$ subscript because non-excludable, so all get same)\\

Prices: $P_{t}^{\frac{-1}{\eta_{p,t}}}= \omega_{p}\left(\pi^{\chi}_{t-1}\right)^{\frac{-1}{\eta_{p,t}}} P_{t-1}^{\frac{-1}{\eta_{p,t}}} +(1-\omega_{p})p_{t}^{*\frac{-1}{\eta_{p,t}}}$\\

\subsection{Equilibrium}

An equilibrium is where households and final good's producers optimize, taking prices as given.  Intermediate goods producers optimize taking factor input prices as given.  The government follows its fiscal policy rules and the monetary authority follows its Taylor rule.  Markets clear and the resource constraint is satisfied.

\subsubsection{Optimization}

Firm optimization and fiscal and monetary rules of given above.

We can write the Lagrangian for the household as (dropping $i$ subscripts since we'll assume a representative agent):
\begin{equation}
\label{HH_Lagrangian}
\begin{split}
\mathcal{L} & = E_{t} \sum_{t=0}^{\infty} \left[ u^{b}_{t}\left( \frac{c^{1-\sigma}_{t}}{1-\sigma}-\theta\frac{(l^{s}_{t})^{1+\varphi}}{1+\varphi}+ \chi^{g}\frac{g_{t}^{1-\sigma_{g}}}{1-\sigma_{g}}\right) + \right. \\
& \left.  \lambda_{t}\left((1-\tau^{k}_{t})r^{k}_{t}v_{t}k_{t-1} + \tau^{k}_{t}\delta^{tau}_{t}k^{\tau}_{t-1} +  \tau^{k}_{t}i_{t}e^{\tau}_{t} + \tau^{ic}_{t}i_{t} + (1-\tau^{l}_{t})w_{t}l^{s}_{t} + \frac{(1+(r_{t}(1-\tau^{i}_{t})))b_{t-1}}{\pi_{t}} + (1-\tau^{d}_{t})d_{t} \right. \right. \\
& \left. \left. + x_{t} - c_{t}(1+\tau^{c}_{t}) - i_{t} - b_{t}\right) + q_{t}\left((1-\delta(v_{t}))k_{t-1} + i_{t}\left[1-S\left(\frac{i_{t}}{i_{t-1}}\right)\right]-k_{t}\right)  \right]
\end{split}
\end{equation}

Where $\lambda_{t}$ and $q_{t}$ are the multipliers on the household's budget constraint and the law of motion for the capital stock, respectively.  

The necessary conditions of the household optimization problem are:

\begin{itemize}
\item HH FOC, consumption
\begin{equation}
\label{HH_FOC_cons}
\lambda_{t}*(1+\tau^{c}_{t}) = u^{b}_{t}c^{-\sigma}_{t} 
\end{equation}

\item HH FOC, labor supply
\begin{equation}
\label{HH_FOC_lab}
(1-\tau^{l}_{t})w_{t}\lambda_{t} = u^{b}_{t}\theta l^{\varphi}_{t} 
\end{equation}

\item HH FOC, investment
\begin{equation}
\label{HH_FOC_inv}
\begin{split}
\lambda_{t}(1-\tau^{k}_{t}e^{\tau}_{t}-\tau^{ic}_{t}) = & q_{t}\left(\left(1-\left(\frac{\kappa}{2}\left(\frac{u^{i}_{t}i_{t}}{i_{t-1}}-1\right)^{2}\right)\right) - \frac{u^{i}_{t}i_{t}}{i_{t-1}}\kappa\left(\frac{u^{i}_{t}i_{t}}{i_{t-1}}-1\right)\right) + \\
 &  \beta q_{t+1}i_{t+1}\kappa\left(\frac{u^{i}_{t+1}i_{t+1}}{i_{t}}-1\right)\frac{u^{i}_{t+1}i_{t+1}}{i^{2}_{t}}
\end{split}
\end{equation}

\item HH FOC, capital
\begin{equation}
\label{HH_FOC_cap}
q_{t} = \beta q_{t+1}\left(1-\delta_{0}-\delta_{1}(v_{t+1}-1)-\frac{\delta_{2}}{2}(v_{t+1}-1)^{2}\right) + \beta\lambda_{t+1}(1-\tau^{k}_{t+1})v_{t+1}r^{k}_{t+1} 
\end{equation}

\item HH FOC, capital utilization
\begin{equation}
\label{HH_FOC_util}
r^{k}_{t} = \left((\frac{q_{t}}{(1-\tau^{k}_{t})*\lambda_{t}}\right)\left(\delta_{1} + \delta_{2}(v_{t}-1)\right)
\end{equation}

\item HH FOC, capital utilization
\begin{equation}
\label{HH_FOC_bond}
\lambda_{t} = \beta\lambda_{t+1}\left(\frac{(1+r_{t+1})}{\pi_{t+1}}\right) ;
\end{equation}

\end{itemize}


\subsubsection{Market Clearing:}

\begin{itemize}
\item Bond Market: Household demand for bonds equals gov't supply: $\int_{0}^{1}b^{d}_{i,t}=B^{s}_{t}$ 
\item Labor Market: Household labor supply equals intermediate producer labor demand: $L^{s}_{t} = L^{d}_{t}$
\item Capital market: Household capital supply equals intermediate producer capital demand: $\int_{0}^{1} v_{i,t}k_{i,t} = \tilde{K}_{t}$
\item Goods market: Household and gov't consumption plus investment equals supply from the final goods producer (Resource Constraint): 
\begin{equation}
\label{Resource}
Y_{t} = C_{t} + I_{t} + g_{t}  
\end{equation}
\end{itemize}

The money market then clears by Walras' Law.


\subsubsection{Characterizing equations}
\label{TAppCharEqs}

Altogether, this model has 36 endogenous variables and 28 exogenous parameters. The 36 endogenous variables are characterized by the 37 equations listed in the sections below. By Walras' Law, one of the market clearing conditions is redundant. In the computation, we discard the resource constraint \eqref{TAppEqMarkClrResConst}.
\begin{align*}
   \text{Endogenous variables:}\quad &\left\{c_t,\lambda_t,Q_{t},l_t,i_t,b_t,v_t,k_t,k_t^\tau,\ve_t^b\right\} \\
   &\left\{Y_t,y_{i,t}^d,P_t,\lambda_{p,t}\right\} \\
   &\left\{y_{i,t},\tilde{k}_{i,t},d_{i,t},r_t^k,w_t,p_{i,t},z_t\right\} \\
   &\left\{G_t,X_t\right\} \\
   &\left\{r_t\right\} \\
   &\left\{p_t,d_t,x_t,g_t,k_{i,t},l_{i,t},C_t,L_t,B_t,K_t,K_t^\tau,D_t,I_t\right\} \\
   \text{Exogenous parameters:}\quad &\left\{\tau_t^c,\tau_t^i,\tau_t^l,\tau_t^k,\tau_t^{ic},\tau_t^d,\delta_t^\tau,e_t^\tau\right\} \\
   &\left\{\beta,\zeta,\sigma\right\} \\
   &\left\{\delta_0,\delta_1,\delta_2,\gamma\right\} \\
   &\left\{\rho_\ve,\mu_\ve,\sigma_\ve\right\} \\
   &\left\{\rho_\lambda,\mu_\lambda,\sigma_\lambda\right\} \\
   &\left\{\alpha,\theta\right\} \\
   &\left\{\gamma_{x,t}\right\} \\
   &\left\{\beta_r,\phi_1,\phi_2,\rho_r\right\}
\end{align*}


There are 21-23 unknown variables after solving for the aggregates from the MC conditions and assuming a symmetric equilibrium for intermediate goods producers : $i, k, b, c, k^{\tau}, \lambda, Q, G, X, d, P, p_{i,t}, w, z, \varepsilon^{b}, \lambda_{p,t}, r, r^{k}, v, l, Y, M$.  These breakdown into the following groups:
\begin{itemize}
\item 4 choice variables (Household): $i,v,l,c$
\item 4 prices: $P,w,r,r^{k}$
\item 2 choice variables (gov't): $g,x$, reallly $2+n$ where $n$ is the number tax instruments
\item 1 choice variable (monetary authority): $M$
\item 2 exogenous state: $z,\varepsilon$
\item 5 endogenous state variable: $k,b,d,k^{\tau},Y$
\end{itemize} 

Solve for these unknowns with the following equations:
\begin{itemize}
\item 8 FOCs, household $\implies i,v,l,\lambda,Q,b,c,???$
\item 3 from FOCs, intermediate goods producers $\implies r^{k},w,p_{i,t}$ (These FOC's give demand for effective capital and labor.  So together with the HH supply of these factors we can use our market clearing conditions to find the eq'm factor prices.)
\item 1 from the profit function of intermediate goods producers $\implies d$
\item 1 from FOCs, final goods producer $\implies Y$
\item 1 from the zero profit condition for the final goods producer $\implies P$ (This together with the definition of Calvo pricing gives you the aggregate price index.)
\item 2 Gov't budget constraint ($+n$ fiscal policy rules) and assumption of X being a certain fraction of spending $\implies B, X$, ($+n$ tax rates)
\item 1 Resource constraint $\implies G$
\item 2 laws of motion $\implies k^{\tau},k$
\item 3 exog shock process $\implies z,\varepsilon^{b},\lambda_{p}$
\item 1 Taylor rule for monetary authority $\implies r,M$ (There is the Taylor Rule plus the fact that $r=max(M,0)$ that determines both these.)
\end{itemize}

So we have 21-23 unknowns and 21-23 equations.  We should be able to plug this into Dynare and solve the model.


\subsection{Estimation}

Following many others (e.g., \cn{TY2010}, \cn{Zubairy2010}, \cn{SW2003}), we will use Bayesian estimation to estimate many parameters of the model.  We use data from the UK to do the estimation.  UK is probably a good bet since they have their own central bank and a VAT of about 20\%.

\begin{itemize}
\item $\delta_{0}$= annual depreciation rate ($\sim$10\%), equals the aggregate SS investment rate ($\frac{\bar{I}}{\bar{K}}$).
\item $\delta_{1}$ = 
\item $\delta_{2}$ = 
\item $\delta^{\tau}$ = Not sure how set- varies by asset type and front loaded, but probably just set to 12-14\%- something higher than econ depreciation
\item $\tau^{l}$ = See \cn{Jones2002} or just do statutory rate for median household or use tax calculator to get marginal rate for median household
\item $\tau^{i}$ = assume = labor income tax
\item $\tau^{d}$ = See \cn{Jones2002} or just do statutory rate for median household or use tax calculator to get marginal rate for median household
\item $\tau^{k}$ = See \cn{Jones2002} 
\item $\tau^{c}$ = See \cn{Jones2002} 
\item $\tau^{ic}$ = 0, most years for most investments this is zero 
\item $e^{\tau}$ = 0, in reality expensing for tax $> 0$ for many small businesses at most times, but we can set to zero   
\item $\gamma$ = 
\item $\zeta$ = I think this is the Frisch elasticity of labor.  \cn{CER2010} set to 0.29.
\item $\sigma$ = coeff of relative risk aversion.  Often set to 2.
\item $\chi^{g}$ = 
\item $\sigma_{g}$ = 
\item $\sigma_{\ve}$ = 
\item $\rho_{\ve}$ = 
\item $\sigma_{\lambda}$ =
\item $\lambda_{p}$ = mean markup = $\frac{1+\lambda_{p}}{\lambda_{p}}$, \cn{BF1995} give U.S. evidence of avg markup of 10-14\% which implies $\lambda_{p}\sim 8$.
\item $\beta$ = set to match (real?) after-tax interest rate (given tax on interest income)
\item $\alpha$ = capital share of output (30-35\%)
\item $\sigma_{z}$ = Estimate log of production function using aggregate data (typical business cycle accounting) to get residual.  Fit the residual to AR(1) to get $\sigma_{z}$ and $\rho_{z}$.
\item $\rho_{z}$ = Estimate log of production function using aggregate data (typical business cycle accounting) to get residual.  Fit the residual to AR(1) to get $\sigma_{z}$ and $\rho_{z}$.
\item $\theta$ = fraction of firms changing price
\item $\phi_{1}$ = estimate a log-log specification of the Taylor rule
\item $\phi_{2}$ = estimate a log-log specification of the Taylor rule
\item $\rho_{r}$ = estimate a log-log specification of the Taylor rule
\item $\bar{L}$ = fraction of hours worked ($0.3 \sim \frac{8}{24}$, 0.5 corresponds to Frisch elasticity = 1??)
\item $\frac{\bar{G}}{\bar{Y}}$ = historical average
\item $\frac{\bar{B}}{\bar{Y}}$ = historial average
\item $\frac{\bar{X}}{\bar{Y}}$ = historical average
\end{itemize}

TABLE: Estimated parameters (priors and posteriors from Bayesian Estimation)


\subsection{Solving the model}

We use Dynare to solve the model with a second order Taylor approximation.

\section{Short Run Multipliers}

\begin{enumerate}
\item FIGURES: impulse response functions for tax cuts (capital, income, consumption, investment) and gov't spending increase (\% changes in output, emp, cons, inv, int rates, inflation) to policy change
\item TABLE: table with size of multipliers
\end{enumerate}

\section{Long Run Multipliers}

FIGURE: graph of multiplier from each of fiscal policy measures over time 

\section{Sensitivity to Government Financing}

FIGURE: graph of multiplier over time given how gov't financing temporary tax cuts (with higher levels of debt going forward or with paying back by increasing taxes).  Maybe 3D graph with multiplier, years from cut, and year to pay back debt as axes.

\section{Sensitivity to Monetary Policy}

Talk about multipliers at and away from the zero-bound.  What happens if hold interest rates constant?  

\section{Dead-weight Loss and Fiscal Policy}

\begin{enumerate}
\item TABLE: table of DWL for each of policy responses
\item FIGURE: graph of DWL over time for policy response
\end{enumerate}

Do some revenue neutral changes in tax policy and calculate welfare....

\section{Sensitivity to Parameters}

How do results change with change to key parameters...


\section{Discussion}

Note that hard to know preferences for gov't spending, but imagine that people know better what want, cutting taxes may be politically easier than increasing spending.  Note other weaknesses of the modeling assumptions.

\section{Conclusion}



\section{Questions:}
\begin{enumerate}
\item Do those with models of the multiplier usually have transfers to households or gov't actually buying stuff?
\item Do I want to say anything about optimal consumption tax being non-zero (like want non-zero inflation) because can more easily do tax cut to respond to recession than could do negative tax rate (spending)?
\item Did Europe (or even US states) have any changes to consumption taxes in response to the recession? \emph{A: I don't think so.  In fact, I believe some states cancelled their sales tax holidays due to budget shortfalls.}
\item Do we want to have rules for fiscal policy as in \cn{Zubairy2010}?  Again, not sure why we need it if not trying to match data.  Just need to estimate SS levels and propose levels that will revert to after aggregate shock.
\end{enumerate}

\section{Does anyone do this?}
\begin{enumerate}
\item \cn{IMF2010} shows impulse response functions to a consumption tax cut.  This paper seems to find that consumption taxes do well compared to other taxes and transfers, but don't do as well as gov't consumption and gov't investment
\item \cn{LPT2010} have consumption taxes, but don't allow them to vary with gov't debt
\item \cn{Zubairy2010} doesn't not have consumption taxes, only capital and labor income taxes.  Says few even have distortionary taxes in models of multipliers.
\item \cn{TY2010} allow for consumption taxes and allow them to vary.  Not answering the same question as I want, but good to look at what they say about consumption taxes.  In particular, how with cons taxes the consumer and producer price index will vary and what crowding out happens with consumption tax changes.
\item \cn{SC2005} have consumption taxes, but don't use them as a policy instrument.  They don't say anything interesting about consumption taxes.
\item \cn{CER2010} don't have consumption taxes and only focus on spending multipliers.  I don't even think they have taxes.  Fiscal and monetary policy rules look simple though- might want to take these.
\end{enumerate}


\section{Links}
\begin{itemize}
\item \url{http://www.econbrowser.com/archives/2010/03/policy_in_dsges.html}
\item \cn{IMF2010} paper: \url{http://www.imf.org/external/pubs/ft/wp/2010/wp1073.pdf}
\item \url{http://delong.typepad.com/sdj/2009/07/cracking-chistiano-eichenbaum-and-rebelos-big-multipliers-without-coffee.html}
\item \url{http://www.econbrowser.com/archives/multipliers/index.html}
\item \cn{TY2010} paper: \url{http://www.nber.org/papers/w15160}
\end{itemize}




\section{MORE QUESTIONS:}
\begin{enumerate}
\item Why do \cn{CER2010} have a subsidy to correct for monopoly of int goods producers?  I don't see others with this.
\item What is capital tax equivalent to?  See what \cn{Zubairy2010} estimates from, but it's like a corp income tax, cap gains tax, div tax rolled into one.
\item Do Multiplier of: gov't spending (baseline to compare), cut in: cap tax, labor tax, cons tax, invest tax credit/bonus deprec/accelerated expensing
\item Can I do bonus deprec with my model- or do I need to keep track of vintage/basis since not partial expensing?
\end{enumerate}

\newpage

\bibliography{TaxMult_bib}



\newpage
\renewcommand{\theequation}{T.\arabic{section}.\arabic{equation}}
                                                 % redefine the command that creates the section number
\renewcommand{\thesection}{T-\arabic{section}}   % redefine the command that creates the equation number
\setcounter{equation}{0}                         % reset counter
\setcounter{section}{0}                          % reset section number
\section*{TECHNICAL APPENDIX}                    % use *-form to suppress numbering

\setcounter{equation}{0} % reset counter
\section{Derivation of monopolistic competition intermediate goods demand and aggregate price Equations}\label{TAppDemPrice}

The profit maximization approach to solving for each intermediate good demand equation $y_{i,t}$ and the aggregate price equation $P_t$ is the following.

\begin{equation}
\max_{y_{i,t}} P_{t}Y_{t}^{d} - \int_{0}^{1}p_{i,t}y_{i,t}di
\end{equation}

The FOC's are:

\begin{equation}
p_{t}(1+\lambda_{p,t})\left(\int_{0}^{1}y_{i,t}^{\left(\frac{1}{1+\lambda_{p,t}}\right)}di\right)^{\lambda_{p,t}}\left(\frac{1}{1+\lambda_{p,t}}\right)y_{i,t}^{\frac{1}{1+\lambda_{p,t}}-1}-p_{i,t} = 0, \forall i
\end{equation}

Dividing the FOCs for intermediate inputs $i$ and $j$ $\implies$
\begin{equation}
\frac{p_{i,t}}{p_{j,t}} = \left(\frac{y_{i,t}}{y_{j,t}}\right)^{\frac{1}{1+\lambda_{p,t}}-1} = \left(\frac{y_{i,t}}{y_{j,t}}\right)^{\frac{-\lambda_{p,t}}{1+\lambda_{p,t}}}
\end{equation}

Rearranging:
\begin{equation}
\begin{split}
&\implies p_{i,t} = \left(\frac{y_{j,t}}{y_{i,t}}\right)^{\frac{\lambda_{p,t}}{1+\lambda_{p,t}}}p_{j,t} \\
& \implies p_{i,t} = y_{j,t}^{\frac{\lambda_{p,t}}{1+\lambda_{p,t}}}p_{j,t}y_{i,t}^{\frac{-\lambda_{p,t}}{1+\lambda_{p,t}}} \\
& \implies p_{i,t}y_{i,t} = p_{j,t}y_{j,t}^{\left(\frac{\lambda_{p,t}}{1+\lambda_{p,t}}\right)}y_{i,t}^{\frac{1}{1+\lambda_{p,t}}}
\end{split}
\end{equation}

Then integrate this condition to yield:
\begin{equation}
\begin{split}
\int_{0}^{1}p_{i,t}y_{i,t}di & = p_{j,t}y_{j,t}^{\left(\frac{\lambda_{p,t}}{1+\lambda_{p,t}}\right)}\int_{0}^{1}y_{i,t}^{\frac{1}{1+\lambda_{p,t}}}di  \\
& = p_{j,t}y_{j,t}^{\left(\frac{\lambda_{p,t}}{1+\lambda_{p,t}}\right)}(Y_{t})^{\frac{1}{1+\lambda_{p,t}}} 
\end{split}
\end{equation}

The zero profit condition implies $P_{t}Y_{t}=\int_{0}^{1}p_{i,t}y_{i,t}di$.  Plugging this into the above and we find:
\begin{equation}
\begin{split}
P_{t}Y_{t} &= p_{j,t}y_{j,t}^{\left(\frac{\lambda_{p,t}}{1+\lambda_{p,t}}\right)}(Y_{t})^{\frac{1}{1+\lambda_{p,t}}} \\
\implies &P_{t}Y_{t}=p_{j,t}y_{j,t}^{\frac{\lambda_{p,t}}{1+\lambda_{p,t}}}(Y_{t})^{\frac{1}{1+\lambda_{p,t}}} \\
\implies & P_{t} =  p_{j,t}y_{j,t}^{\frac{\lambda_{p,t}}{1+\lambda_{p,t}}}(Y_{t})^{\frac{-\lambda_{p,t}}{1+\lambda_{p,t}}}
\end{split}
\end{equation}

Which implies the demand function (just rearranging terms):
\begin{equation}
\begin{split}
y_{i,t} =  \left(\frac{p_{i,t}}{P_{t}}\right)^{\frac{-(1+\lambda_{p,t})}{\lambda_{p,t}}}Y_{t}, \forall i
\end{split}
\end{equation}

To find $P_{t}$ use the zero profit condition:
\begin{equation}
\begin{split}
& P_{t}Y_{t} = \int_{0}^{1}p_{i,t}y_{i,t}^{\left(\frac{\lambda_{p,t}}{1+\lambda_{p,t}}\right)}(Y_{t})^{\frac{1}{1+\lambda_{p,t}}}di \\
\implies & P_{t}Y_{t} = \int_{0}^{1}p_{i,t}\left(\left(\frac{p_{i,t}}{P_{t}}\right)^{\frac{-(1+\lambda_{p,t})}{\lambda_{p,t}}}Y_{t}\right)^{\left(\frac{\lambda_{p,t}}{1+\lambda_{p,t}}\right)}(Y_{t})^{\frac{1}{1+\lambda_{p,t}}}di \\
\implies & P_{t}Y_{t} = \int_{0}^{1}p_{i,t}\left(\frac{p_{i,t}}{P_{t}}\right)^{\frac{-(1+\lambda_{p,t})}{\lambda_{p,t}}}Y_{t}di \\
\implies & P_{t}Y_{t} = \int_{0}^{1}p_{i,t}p_{i,t}^{\frac{-(1+\lambda_{p,t})}{\lambda_{p,t}}}P_{t}^{\frac{(1+\lambda_{p,t})}{\lambda_{p,t}}}Y_{t}di \\
\implies & P_{t}Y_{t} = \int_{0}^{1}p_{i,t}^{\frac{-1}{\lambda_{p,t}}}di \:  Y_{t}P_{t}^{\frac{1+\lambda_{p,t}}{\lambda_{p,t}}} \\
\implies & P_{t} = \int_{0}^{1}p_{i,t}^{\frac{-1}{\lambda_{p,t}}}di \: P_{t}^{\frac{1+\lambda_{p,t}}{\lambda_{p,t}}} \\
\implies & P_{t}^{\frac{-1}{\lambda_{p,t}}} = \int_{0}^{1}p_{i,t}^{\frac{-1}{\lambda_{p,t}}}di \\
\implies & P_{t} = \left(\int_{0}^{1}p_{i,t}^{\frac{-1}{\lambda_{p,t}}}di\right)^{-\lambda_{p,t}} \\
\end{split}
\end{equation}

The cost minimization approach to solving for each intermediate good demand equation $y_{i,t}$ and the aggregate price equation $P_t$ is the following.
\begin{equation}\label{TAppEqObjCostMin}
   \min_{y_{i,t}} \int_{0}^{1}p_{i,t}y_{i,t}di \quad\text{s.t.}\quad Y_t \leq \left(\int_{0}^{1} y_{i,t}^{\frac{1}{1+\lambda_{p,t}}}di\right)^{1+\lambda_{p,t}}
\end{equation}
The Lagrangian for this minimization problem is the following,
\begin{equation}\label{TAppEqLagrCostMin}
   \mathcal{L} = \int_{0}^{1}p_{i,t}y_{i,t}di + \lambda_y\left[Y_t - \left(\int_{0}^{1} y_{i,t}^{\frac{1}{1+\lambda_{p,t}}}di\right)^{1+\lambda_{p,t}}\right]
\end{equation}
in which the multiplier $\lambda_y$ has the interpretation of being the marginal cost of an extra unit of aggregated output. That is, $\lambda_y$ is the price of aggregate output $P_t$.
\begin{equation}\label{TAppEqLagrCostMin2}
   \mathcal{L} = \int_{0}^{1}p_{i,t}y_{i,t}di + P_t\left[Y_t - \left(\int_{0}^{1} y_{i,t}^{\frac{1}{1+\lambda_{p,t}}}di\right)^{1+\lambda_{p,t}}\right]
\end{equation}
We need to finish this....


\newpage
\setcounter{equation}{0} % reset counter
\section{Derivation of real marginal costs}\label{TAppRealMC}

Dividing the period profits equation \eqref{EqIntGoodDiv} by aggregate prices $P_t$ gives the following real total costs function.
\begin{equation}\label{TAppEqRealTC}
   tc_t = \frac{r_t^k}{P_t}\tilde{k}_{i,t} + \frac{w_t}{P_t}l_{i,t}
\end{equation}
Substituting in the expression for $\tilde{k}_{i,t}$ from the capital-labor ratio equation \eqref{Eqklratio} gives real total costs in terms of $\alpha$, $w_t$, $P_t$, and $l_{i,t}$.
\begin{equation}\label{TAppEqRealTCl}
   tc_t = \frac{1}{1-\alpha}\left(\frac{w_t}{P_t}\right)l_{i,t}
\end{equation}

Because the intermediate goods production exhibits constant returns to scale, we can find the real marginal cost $mc_t$ by finding the amount of labor necessary to produce 1 unit of output, and substituting that value into the real total costs equation \eqref{TAppEqRealTCl}.
\begin{equation}\label{TAppEqlunitprod1}
   z_t\tilde{k}_{i,t}^\alpha l_{i,t}^{1-\alpha} = 1
\end{equation}
Next, substitute in the equation for capital $\tilde{k}_{i,t}$ in terms of labor from the capital-labor ratio equation \eqref{Eqklratio}.
\begin{equation}\label{TAppEqlunitprod2}
   z_t\left(\frac{\alpha}{1-\alpha}\left[\frac{w_t}{r_t^k}\right]l_{i,t}\right)^\alpha l_{i,t}^{1-\alpha} = 1 \quad\Rightarrow\quad l_{i,t} = \frac{1}{z_t}\left(\frac{\alpha}{1-\alpha}\left[\frac{w_t}{r_t^k}\right]\right)^{-\alpha}
\end{equation}

Lastly, substitute the amount of labor necessary to produce one unit of output from \eqref{TAppEqlunitprod2} into the expression for real total costs \eqref{TAppEqRealTCl} to get real marginal costs.
\begin{equation}\label{TAppEqRealMC}
   mc_t = \frac{1}{1-\alpha}\left(\frac{w_t}{P_t}\right)\frac{1}{z_t}\left(\frac{\alpha}{1-\alpha}\left[\frac{w_t}{r_t^k}\right]\right)^{-\alpha} = \frac{w_t^{1-\alpha}(r_t^k)^\alpha}{z_t P_t \alpha^\alpha (1-\alpha)^{1-\alpha}}
\end{equation}




\end{document}
