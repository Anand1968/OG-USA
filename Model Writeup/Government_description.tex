\chapter{Government}
\index{Government%
@\emph{Government}}%



%\pagestyle{empty}


\section{Overview of Government in the Model}

Government will have four functions in our model:
\begin{enumerate}
\item The government runs a tax and social security system
	\begin{itemize}
	\item The tax system will be input by the user and/or determined by the current tax law (the default unless the user supplies changes)
	\end{itemize}
\item The government makes transfers to households outside of the tax/social security system
\item The government produces output that contributes to private consumption goods (e.g., education)
\item The government purchases capital and hires labor to produced a non-rival public good (e.g., national defense)
\end{enumerate}


    \section{Government budgeting}
      \begin{equation}
      \label{eqn:gbc}
      D_{t+1} + T^{\tau}_{t} = (1+r_{t})D_{t} + T^{H}_{t} + G^{subs}_{t} + G^{emp}_{t} + I^{G}_{t}
      \end{equation}

    \subsection{Rule for long-term fiscal stability}
      Let $D_t$ denote the government's outstanding real debt.  $T_t$ is total tax revenue, $T^H_t$ is total household transfers, $G_t$ is government purchases of goods, $L_t$ is the real value of purchases of labor services, and $S_t$ is subsidies to government run firms.

      \begin{equation}
        D_{t+1} = D_t(1+r_t) - T_t + T^H_t + G_t + L_t + S_t
      \end{equation}

      Letting a carat denote the ratio of a variable to GDP, we can rewrite this as follows:

      \begin{equation} \label{EqDhatlom}
        (1+g_{Yt}) \hat D_{t+1} = \hat D_t(1+r_t) - \hat T_t + \hat T^H_t + \hat G_t + \hat L_t + \hat S_t
      \end{equation}

      We need to adopt a government fiscal rule that determines how our residual expenditure $\hat G_t$ evolves over time.

      One way is to adopt a balanced budget rule which keeps the debt-to-GDP ratio constant at it's initial value of $\hat D_0$.

      \begin{align}
        (1+g_{Yt}) \hat D_0 & = \hat D_0(1+r_t) - \hat T_t + \hat T^H_t + \hat G_t + \hat L_t + \hat S_t \nonumber \\
        \hat G_t & = \hat D_0(g_{Yt}-r_t) + \hat T_t - \hat T^H_t -\hat L_t - \hat S_t \label{EqBalBudRule}
      \end{align}

      Another rule is to hold govenrment spending constant and let debt evolve as it will for several period.  Then in period $T$ impose fiscal austerity which forces $\hat G_t$ to adjust over time so that $\hat D_t$ goes to a steady value.

      \begin{equation}
        \hat G_t - \bar G = \rho_t (\hat D_t - \bar D);\quad \rho_t<0 \label{EqAdjRule}
      \end{equation}

      Substituting this into \eqref{EqDhatlom} gives:
      \begin{align}
        (1+g_{Yt}) \hat D_{t+1} & = \hat D_t(1+r_t) - \hat T_t + \hat T^H_t + \rho_t (\hat D_t - \bar D) + \bar G + \hat L_t + \hat S_t \nonumber \\
        \hat D_{t+1} & = \frac{\hat D_t(1+r_t) - \hat T_t + \hat T^H_t + \rho_t (\hat D_t - \bar D) + \bar G + \hat L_t + \hat S_t }{1+g_{Yt}} \label{EqDhatlom2}
      \end{align}

      Consider the steady state version of this.
      \begin{align}
        (1+\bar g_{Y}) \bar D & = \bar D(1+\bar r) + \bar T - \bar T^H + \rho_t (\bar D - \bar D) + \bar G + \bar L + \bar S_t  \nonumber \\
        \bar G & = \bar D(\bar g_{Y} -\bar r) + \bar T - \bar T^H - \bar L - \bar S  \label{EqGbardef}
      \end{align}

      This tells us the long-run value of government spending to GDP that will maintain the debt to GDP target.

      In order for \eqref{EqDhatlom2} to be a contraction mapping over $\hat D$ and thus converge to a steady state, we must put bounds on $\rho_t$.  Rearranging \eqref{EqDhatlom2} and using \eqref{EqGbardef}:

      \begin{align}
        \begin{split}
        (1+g_{Yt}) \hat D_{t+1} & = \hat D_t (1+r_t) - \hat T_t + \hat T^H_t + \rho_t (\hat D_t - \bar D) + \hat L_t + \hat S_t \\
        & + \bar D(\bar g_{Y} - \bar r) + \bar T - \bar T^H - \bar L - \bar S
        \end{split} \nonumber \\
        \begin{split}
        (1+g_{Yt}) \hat D_{t+1} & = \hat D_t (1+r_t) - \hat T_t + \hat T^H_t + \rho_t \hat D_t - \rho_t \bar D + \hat L_t + \hat S_t \\
        & + \bar g_Y \bar D - \bar r \bar D + \bar T - \bar T^H - \bar L - \bar S
        \end{split} \nonumber \\
        \begin{split}
        (1+g_{Yt}) \hat D_{t+1} & =  \hat D_t (1+r_t) + \rho_t (\hat D_t -\bar D) + (\bar g_Y - \bar r ) \bar D \\
        & - (\hat T_t - \bar T) + (\hat T^H_t -\bar T^H) + (\hat L_t -\bar L) + (\hat S_t -\bar S)
        \end{split} \nonumber \\
        \begin{split}
        \hat D_{t+1} - \bar D & = \hat D_t \frac{1+r_t}{1+g_{Yt}} + \frac{\rho_t}{1+g_{Yt}} (\hat D_t -\bar D) + \left( \frac{\bar g_Y - \bar r}{1+g_{Yt}} - 1 \right) \bar D \\
        & + \frac{-(\hat T_t - \bar T) + (\hat T^H_t -\bar T^H) + (\hat L_t -\bar L) + (\hat S_t -\bar S)}{1+g_{Yt}}
        \end{split}  \nonumber \\
        \begin{split} 
        \hat D_{t+1} - \bar D & = \frac{1+r_t+\rho_t}{1+g_{Yt}} (\hat D_t -\bar D) \\
        & + \frac{-(\hat T_t - \bar T) + (\hat T^H_t -\bar T^H) + (\hat L_t -\bar L) + (\hat S_t -\bar S)}{1+g_{Yt}}
        \end{split} 
        \label{EqStab}
      \end{align}

      We need $\frac{\hat D_{t+1} - \bar D}{\hat D_{t} - \bar D} < 1$ for stability.  Equation \eqref{EqStab} gives:
      \begin{align} 
        \frac{\hat D_{t+1} - \bar D}{\hat D_t -\bar D} & = \frac{1+r_t+\rho_t}{1+g_{Yt}}  + \frac{-(\hat T_t - \bar T) + (\hat T^H_t -\bar T^H) + (\hat L_t -\bar L) + (\hat S_t -\bar S)}{(1+g_{Yt})(\hat D_t -\bar D)} < 1 \nonumber \\
        & \frac{1+r_t+\rho_t}{1+g_{Yt}}  < \frac{(\hat T_t - \bar T) - (\hat T^H_t -\bar T^H) - (\hat L_t -\bar L) - (\hat S_t -\bar S)}{(1+g_{Yt})(\hat D_t -\bar D)} \nonumber \\
        & \rho_t  < (1+r_t)\frac{(\hat T_t - \bar T) - (\hat T^H_t -\bar T^H) - (\hat L_t -\bar L) - (\hat S_t -\bar S)}{\hat D_t -\bar D}
      \end{align} 

    \subsection{Transfer system}
      We'll need to estimate this.  Probably following \citet{FR1993}.  Or perhaps the micro simulation model calculates some of these.  Or ideally we get something like \citet{KotlikoffXXXX}.

\section{Government production of private goods}

\section{Government production of public goods}

\section{Steps for adding government to the dynamic model}

\begin{enumerate}
\item 1 firm
\item 1 firm + gov't
\item 2 firms + gov't
\item tax 2 firms + gov't
\item N firms with taxes + gov't
\end{enumerate}
