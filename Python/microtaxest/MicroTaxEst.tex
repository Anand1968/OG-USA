\documentclass[letterpaper,12pt]{article}

\usepackage{threeparttable}
\usepackage{geometry}
\geometry{letterpaper,tmargin=1in,bmargin=1in,lmargin=1.25in,rmargin=1.25in}
\usepackage[format=hang,font=normalsize,labelfont=bf]{caption}
\usepackage{amsmath}
\usepackage{multirow}
\usepackage{array}
\usepackage{delarray}
\usepackage{amssymb}
\usepackage{amsthm}
\usepackage{lscape}
\usepackage{natbib}
\usepackage{setspace}
\usepackage{float,color}
\usepackage[pdftex]{graphicx}
\usepackage{pdfsync}
\usepackage{verbatim}
\usepackage{placeins}
\usepackage{geometry}
\usepackage{pdflscape}
\synctex=1
\usepackage{hyperref}
\hypersetup{colorlinks,linkcolor=red,urlcolor=blue,citecolor=red}
\usepackage{bm}
\usepackage{color}
\usepackage{listings}
\lstset{frame=single,
  language=Python,
  showstringspaces=false,
  columns=flexible,
  basicstyle={\small\ttfamily},
  numbers=none,
  breaklines=true,
  breakatwhitespace=true
  tabsize=3
}


\theoremstyle{definition}
\newtheorem{theorem}{Theorem}
\newtheorem{acknowledgement}[theorem]{Acknowledgement}
\newtheorem{algorithm}[theorem]{Algorithm}
\newtheorem{axiom}[theorem]{Axiom}
\newtheorem{case}[theorem]{Case}
\newtheorem{claim}[theorem]{Claim}
\newtheorem{conclusion}[theorem]{Conclusion}
\newtheorem{condition}[theorem]{Condition}
\newtheorem{conjecture}[theorem]{Conjecture}
\newtheorem{corollary}[theorem]{Corollary}
\newtheorem{criterion}[theorem]{Criterion}
\newtheorem{definition}{Definition} % Number definitions on their own
\newtheorem{derivation}{Derivation} % Number derivations on their own
\newtheorem{example}[theorem]{Example}
\newtheorem{exercise}[theorem]{Exercise}
\newtheorem{lemma}[theorem]{Lemma}
\newtheorem{notation}[theorem]{Notation}
\newtheorem{problem}[theorem]{Problem}
\newtheorem{proposition}{Proposition} % Number propositions on their own
\newtheorem{remark}[theorem]{Remark}
\newtheorem{solution}[theorem]{Solution}
\newtheorem{summary}[theorem]{Summary}
\bibliographystyle{aer}
\newcommand\ve{\varepsilon}
\renewcommand\theenumi{\roman{enumi}}
\newcommand\norm[1]{\left\lVert#1\right\rVert}
\DeclareMathOperator*{\argmin}{argmin}

\begin{document}

\title{Integrating Microsimulation Tax Data into a Dynamic Scoring Model}
\author{Richard W. Evans}
\date{November 2015}
\maketitle


\section{Introduction}\label{SecIntro}

  This document shows how to integrate tax data from the Open Source Policy Center (OSPC) tax calculator microsimulation model into a dynamic scoring model. For this early version of the integration, we will use the dynamic general equilibrium model described in \citet{DeBackerEtAl2015}. This integration allows us to study the effects of a rich set of tax levers and household heterogeneity while at the same time being able to study the general equilibrium effects of changes in those levers on aggregate macroeconomic variables.


\section{Estimating Tax Functions from Microsimulation Data}\label{SecMicroTaxFunc}

  The microsimulation model of the Open Source Policy Center (OSPC) is called ``Tax Calculator''. The source code for this model is open source and available at \href{https://github.com/open-source-economics/Tax-Calculator}{https://github.com/open-source-economics/Tax-Calculator}.\footnote{Anyone can also access a web app at \href{http://www.ospc.org/taxbrain/}{http://www.ospc.org/taxbrain/} that serves as a simple interface for using the underlying source code.} This microsimulation model allows for....

  \begin{itemize}
    \item Explain what microsimulation model is and what it does.
    \item Describe the data we generate from it.
    \item Show the functional form estimation and how it summarizes the data.
    \item Describe how we integrate this into the model.
  \end{itemize}


  \subsection{Tax rate functional form}\label{SecFuncForm}

    Tax data has a robust negative exponential form that we want to estimate. A ratio of polynomials has the property of looking like the data and having strictly positive derivatives (marginal tax rates).
    \begin{equation}\label{EqSimpRatio}
      \tau(x) = \frac{x}{x+k}, \quad\text{where}\quad x,k>0
    \end{equation}
    This function has a shape where $\tau(0) = 0$, $\lim_{x\rightarrow\infty}\tau(x) = 1$, and $\tau'(x) > 0$. This seems ideal for fitting the data we want to fit.

    Further, this equation can be adjusted so that its minimum value $\tau(0)$ is negative and its maximum value $\lim_{x\rightarrow\infty}\tau(x)$ is less than or greater than zero. We change these upper and lower bounds by adding parameters $max$ and $min$ to the function.
    \begin{equation}\label{EqSimpRatio2}
      \tau(x) = (max - min)\left(\frac{x}{x+k}\right) + min, \quad\text{where}\quad x,k>0
    \end{equation}
    The tax function in equation \eqref{EqSimpRatio2} will have a maximum asymptote $\lim_{x\rightarrow\infty}\tau(x) = max$ and a minimum value of $\tau(0) = min$, while still preserving the negative exponential shape and the strictly positive derivative.

    In order to fit more nuanced transitions that the data could suggest, we replace the simple $x$ in \eqref{EqSimpRatio2} with a quadratic polynomial in $x$.
    \begin{equation}\label{EqSimpRatioQuad}
      \tau(x_i) = (max - min)\left(\frac{Ax_i^2 + Bx_i}{Ax_i^2 + Bx_i + C}\right) + min, \quad\text{where}\quad x_i,A,B,C>0
    \end{equation}
    This ratio of polynomials allows the rate of change in the slope to speed up or slow down depending on what the data look like.

    In our model, our tax function must be a function of both labor income $x$ and capital income $y$. The bivariate analogue to equation \ref{EqSimpRatioQuad} is the following equation with ratios of second order polynomials in $x$ and $y$.
    \begin{equation}\label{EqBivRatioQuad}
      \begin{split}
        \tau(x_i,y_i) = &(max - min)\left(\frac{Ax_i^2 + By_i^2 + Cx_iy_i + Dx_i + Ey_i}{Ax_i^2 + By_i^2 + Cx_iy_i + Dx_i + Ey_i + F}\right) + min, \\
        &\quad\text{where}\quad x,y,A,B,C,D,E,F>0
      \end{split}
    \end{equation}

    Lastly, we want to be able to have the maximum and minimum values of the tax rate fluctuate depending on whether an individual has more labor income $x_i$ or more capital income $y_i$. Let $\phi_i$ be the percent of total income $x+y$ that is labor income $x$.
    \begin{equation}\label{EqPhi}
      \phi_i = \frac{x_i}{x_i+y_i}
    \end{equation}
    Now we can adjust the maximum and minimum of the function depending on the relative amounts $\phi_i$ of labor income and capital income. Let $max_x$ and $min_x$ represent the maximum and minimum effective tax rates for any labor income $x$ when capital income is zero $y=0$. And let $max_y$ and $min_y$ represent the maximum and minimum effective tax rates for any capital income $y$ when labor income is zero $x=0$.
    \begin{equation}\label{EqBivRatioQuad2}
      \begin{split}
        \tau(x_i,y_i) = &\Bigl[\phi_i(max_x - min_x) + (1 - \phi_i)(max_x - min_x)\Bigr]\left(\frac{P(x_i,y_i)}{P(x_i,y_i) + F}\right) + ... \\
        &\quad \phi_i min_x + (1 - \phi_i)min_y, \\
        &\qquad\qquad\text{where}\quad P(x,y) = Ax^2 + By^2 + Cxy + Dx + Ey \\
        &\qquad\qquad\text{and}\quad x,y,A,B,C,D,E,F>0
      \end{split}
    \end{equation}


  \subsection{Estimation of tax rate functional form}\label{SecFuncFormEst}

    \begin{itemize}
      \item Transform $x_i$ and $y_i$ variables in $P(x,y)$ to percent deviations from their respective means to avoid scale issues.
      \item Do nonlinear constrained weighted least squares estimation.
      \item Save parameters for each age group and year in an array to be passed in to dynamic general equilibrium model.
    \end{itemize}


\bibliography{MicroTaxEst}


\end{document}
